\documentclass{acm_proc_article-sp}

\usepackage{url}
\usepackage{microtype}
\usepackage{appendix}

\begin{document}

% Without this, stuff overfills into the margin. TODO: Eventually, rewrite those
% paragraphs so that they don't.
%\sloppy

\title{Compromising Privacy with the FLUSH+RELOAD attack\titlenote{Submitted April 4, 2015}}
\subtitle{CPSC 502.04 Project Final Report}

\numberofauthors{1}

\author{
\alignauthor
Taylor Hornby (Supervised by Dr. John Aycock)\titlenote{John Aycock reviewed and
edited a draft of this document before it was submitted.}\\
\email{th@defuse.ca}
}

\maketitle

\begin{abstract}
    Side channel attacks are typically used to break implementations of
    cryptography. More recently, side channel attacks have been discovered in
    more general settings that violate user privacy. We continue this trend by
    applying the \textsc{Flush+Reload} L3 cache side channel from Yuval Yarom and Katrina
    Falkner \cite{yarom2013flush} to TrueCrypt and the Links web browser. We
    show that an attacker can reliably determine whether a TrueCrypt volume
    mounted by the victim contains a hidden volume or not. We also show that if
    the attacker knows the victim visited one of the top 100 Wikipedia pages in
    the Links browser, they can decide which page was visited with a success
    rate of over 80\%.
\end{abstract}

\section{Introduction}
\label{sec:intro}

Side channels are a well-known class of attack where an adversary learns
information about their victim through implicit, rather than explicit,
transmission of information.

Side channels are most frequently used to extract secret keys from
implementations of cryptography. For example, Thomas Messerges et al.\ were able
to extract the secret RSA key from a smart card by looking at the amount of
power it uses \cite{messerges1999power}. More recently, Genkin et al.\ were able
to extract a secret RSA key just by listening to the sounds a computer makes as
it is decrypting data \cite{genkin2013rsa}. 

Side channel attacks are also being used to compromise user privacy. For
example, Liang Cai and Hao Chen have shown that keystrokes (including typed PIN
numbers) can be reliably recovered from accelerometer data on smartphones
\cite{cai2012practicality}. Shuo Chen et al.\ showed that observing a web
application's behavior reveals information about the user's input. They gave
real-world examples where they could determine which medical conditions the user
was searching for, what their income range was, and how they allocated their
investment funds \cite{chen2010side}.

In this paper, we continue the trend of using side channels to attack the user's
privacy. We show that the generic L3 cache side channel called \textsc{Flush+Reload}
\cite{yarom2013flush} can be applied in non-cryptography settings to
meaningfully violate confidentiality. The attacks we present let an attacker on
a shared system (1) determine whether the TrueCrypt volume a victim mounted
contained a hidden volume or not, and (2) determine which of the top 100
Wikipedia pages the victim visited using the Links web browser.

To the best of our knowledge, this is the first time that a generic cache-based
side channel has been used to compromise privacy by attacking
a non-cryptographic application. The attacks we present show that cache side
channels have implications beyond just extracting secret keys from cryptography
software. We hope that the attacks will bring more attention to the privacy
issues caused by cache side channels, and the \textsc{Flush+Reload} attack in
particular.

We developed a robust \textsc{Flush+Reload} attack tool based on code the
original paper's authors gave us. Our code is published online under an
open-source license (see Appendix \ref{sec:reproducing}), so other researchers
can quickly reproduce our work and start developing better attacks. The source
code for our experiments and the data they produced is being made available as
well. We hope more researchers will follow our example of making code and data
available at the time of publication.

In the next Section, we summarize the \textsc{Flush+Reload} side channel that our attacks
are based on. Section 3 describes our implementation of the \textsc{Flush+Reload} attack
and explains our experimental methodology. The attacks and experimental results
are given in Section 4. Section 5 highlights recent work using side channels to
break privacy and explains how our contributions fit in with that work. Finally
we list some ideas for future work in Section 6, and conclude in Section 7.

\section{The Flush+Reload Attack}

The attacks we present use the \textsc{Flush+Reload} attack by Yuval Yarom and
Katrina Falkner \cite{yarom2013flush}. The side channel was first described by
Bangerter et al.\ \cite{gullasch2011cache} where it was used on AES lookup
tables to extract keys and plaintext during encryption. Yuval Yarom and Katrina
Falkner realized it could be applied more generally to spy on the code a process
executes. It was given the name \textsc{Flush+Reload} and has since been applied
to break GnuPG \cite{yarom2013flush} and OpenSSL \cite{benger2014ooh,
yarom2014recovering}.

Here we give a brief explanation of the \textsc{Flush+Reload} attack with enough detail
to understand our attacks. For all of the details, refer to Yuval Yarom and
Katrina Falkner's paper \cite{yarom2013flush}.

\textsc{Flush+Reload} is a generic L3 (or last-level, in case the system does
not have an L3 cache) cache side channel attack. It takes advantage of
executable code page sharing between processes. If Alice is the first to run
a program, the operating system will load the program into physical memory. When
Bob comes along and runs the same program, instead of loading a second copy into
memory, the operating system will set Bob's page tables to use the copy that was
loaded into memory for Alice. The result is that both Alice and Bob are working
off of the same physical memory. If Bob can determine which cache lines of the
shared memory are present in the cache over time, he can tell which cache lines
Alice's process accesses over time. This is what the \textsc{Flush+Reload}
attack does.

Suppose an attacker knows their victim is running a certain program and would
like to see which code the victim's process is running. \textsc{Flush+Reload}
lets the attacker select a handful of cache lines to watch, called ``probes.''
The attacker flushes those lines out of the cache, waits for a certain number of
clock cycles, then times how long it takes to read those lines. If the read is
fast (i.e. consistent with being in the L3 cache), it means the victim accessed
the line during the waiting period. If the read is slow, the victim did not
access the line. The attacker can repeat the process (flushing, then reloading)
to see which code the victim process is executing over time.

For example, the attack against GnuPG puts probes on the RSA modular
exponentiation routines and uses a waiting time of 2048 clock cycles.
Effectively, the attacker gets a 1.5 MHz resolution view into what the victim's
process is doing \cite{yarom2013flush}.

The attack assumes only one instance of the spied-on program is running at
a time. If multiple instances are running, an access to a probed cache line by
any instance will trigger the probe. Only one attacking process can run at
a time on a CPU core. If a pair of attacking processes have to contend for CPU
time, they will miss measurements. So it is assumed only one attacking process
can run at a time.

In summary, the attacker specifies a few (roughly 1 to 5, where adding more
probes makes the measurements less reliable) probe locations within a binary and
the amount of clock cycles to wait between measurements. As output, the attacker
learns the sequence of probes that were accessed during the waiting periods.

\section{Methods}
\label{sec:methods}

We set out to see if the \textsc{Flush+Reload} attack had privacy implications
beyond attacking cryptographic software. We did this by brainstorming possible
attacks, selecting a program to attack, manually implementing the attack as best
we could, and then running experiments to see how well the attack works.

The binaries we attacked were built from source. We left symbols in to make it
easier to find the probe locations in the binary. Symbols are not required for
the attack to work, but without them, the manual work of discovering the attack
would take longer.

To make sure that we did not cherry pick vulnerable programs, we kept track of
all the programs we looked at without developing any attacks. These are
disclosed in Section \ref{sec:failures}.

\subsection{Attack Implementation}

The attacks are built on the following tools that we built.

\begin{itemize}
    \item A \textsc{Flush+Reload} attack tool, written in C, takes a binary and probe
        addresses with single-character names as arguments. When probes are hit,
        their single-character names are printed to standard output, with pipe
        characters separating the waiting periods. This tool is
        a re-implementation of the tool the original paper's
        (\cite{yarom2013flush}) authors provided us.

    \item A Ruby class that lets Ruby scripts run the attack tool and monitor
        its output. It provides a callback whenever there is a burst of probe
        hits. This lets the attack script react to events (such as the entire
        execution of a command) without having to continually parse the attack
        tool's output. Our attacks are implemented as Ruby scripts that use this
        class.
\end{itemize}

\subsection{Experimental Evaluation}

We ran experiments on two systems. The first system is a laptop; the second
system is a dedicated server hosting a low-traffic website. The specifications
of these systems are given in Table \ref{table:specs}. We will refer to these
systems as System 1 and System 2, respectively.

\begin{table*}
    \centering
\begin{tabular}{|c|c|c|}
    \hline
    & \textbf{System 1} & \textbf{System 2} \\
    \hline
    \textbf{Use} & Laptop & Web server \\
    \hline
    \textbf{OS} & Arch Linux (February 2015) & Debian Wheezy \\
    \hline
    \textbf{CPU} & Intel Core2 Duo P8700 2.53GHz & Intel Xeon E3-1245 V2 3.40GHz  \\
    \hline
    \textbf{RAM} & 4GiB DDR2 800MHz & 32GiB DDR3 1333MHz \\
    \hline
    \textbf{L1 Cache} & 64KiB Split & 256KiB Unified \\
    \hline
    \textbf{L2 Cache} & 3MiB Unified & 1MiB Unified \\
    \hline
    \textbf{L3 Cache} & None & 8MiB Unified \\
    \hline
\end{tabular}
\caption{System specifications. Cache specifications were obtained by the
\texttt{dmidecode} utility and may not be accurate. System 1 does not have a L3
cache, but FLUSH+RELOAD works with its L2 cache as it is shared between cores.}
\label{table:specs}
\end{table*}

All of the experiments have been automated so that they are easy to repeat. The
source code and data from all experiment runs have been published so that other
researchers can verify and reproduce our work, see Appendix
\ref{sec:reproducing}.

All experiments were run with low system load.

\section{Results}
\label{sec:results}

We present two attacks. The first attack lets an attacker tell whether
a victim's TrueCrypt volume is normal or hidden. The second lets the the
attacker tell which of the top 100 Wikipedia pages the victim visited using the
Links web browser.

\subsection{TrueCrypt}

TrueCrypt is a popular disk encryption utility that supports storing encrypted
filesystems in files called TrueCrypt volumes. TrueCrypt gives users the option
to place a hidden volume inside a normal volume. Given the passphrase to the
outer normal volume, it is not supposed to be possible to determine whether
a hidden volume exists. This is to protect the user in case they are coerced
into revealing their passphrase. They can reveal the passphrase to the outer
volume and the contents of the hidden volume will be safe. 

Our attack watches the victim mount a TrueCrypt volume and determines whether
they mounted a hidden volume or a normal one.

\subsubsection{Attack Implementation}

TrueCrypt is written in C++ and has two classes defining the layout of hidden
and normal volumes. The normal volume layout is represented by the class
\texttt{VolumeLayoutV2Normal} and the \texttt{VolumeLayoutV2Hidden} class
represents the hidden volume layout. There are three other classes for the
operating system encryption layout and volume layouts from older versions,
making five volume layout classes in total.

When TrueCrypt mounts a volume, it instantiates all five layout classes and
tries to decrypt the volume using each one. It has to do this because, by
design, TrueCrypt volumes are supposed to be indistinguishable from random data,
so there is no way to tell which volume layout is the right one without trying
to decrypt the volume and seeing if it worked.

Each class implements a \texttt{GetDataSize()} method. This method is only
called on a layout object after the volume has been successfully decrypted using
that layout, so it is a good candidate for a \textsc{Flush+Reload} probe.

At first we tried to place probes on both \texttt{VolumeLayoutV2Normal}'s
\texttt{GetDataSize()} and \texttt{VolumeLayoutV2Hidden}'s
\texttt{GetDataSize()}. This did not work because in the binary, the hidden
method immediately follows the normal method, and instruction prefetching
triggers the hidden probe when the normal method gets executed.

To work around that problem, we placed two \textsc{Flush+Reload} probes. The
first is placed on the entry point to the TrueCrypt binary, so that we can tell
when the user runs a TrueCrypt command. The second is placed in
\texttt{VolumeLayoutV2Normal}'s \texttt{GetDataSize()}, which will get hit once
when the TrueCrypt binary is loaded, and then once again only if the volume is
normal.

The attacker records the probe sequence as the victim mounts the volume. They
check if the probe sequence ends in the \texttt{VolumeLayoutV2Normal} probe. If
it does, the volume was normal. If not, the volume was hidden (or what the
attacker captured was not the result of a mount command; we assume the attacker
knows that the user is mounting a volume).

\subsubsection{Experiment}

Our experiment creates two 1MB TrueCrypt volumes. One is a normal volume, one
contains a hidden volume; both are protected by the same passphrase.

The experiment assumes the attacker knows the TrueCrypt command was run to mount
a volume, and not some other task (like unmounting a volume). This could be done
in practice by looking at the process list to see which command-line options
were passed to TrueCrypt.

The experiment starts the attack tool on the TrueCrypt binary. After waiting for
the tool to start, it randomly mounts either the normal or hidden volume with
TrueCrypt's command-line tool. It stops the attack tool and checks if the last
probe hit was the \texttt{GetDataSize} probe. If it was, it guesses that the
volume was normal. If not, it guesses that the volume was hidden. This process
is repeated over and over $S$ times.

In one run\footnote{Run truecrypt/0001} of the experiment on System 1 with
$S=500$, the guess was right 416 times (83\% of the time). In a later
run\footnote{Run truecrypt/0003}, also on System 1, with $S=100$, the guess was
right 97\% of the time.

We could not reproduce the attack on System 2. We believe this is due to
differences in the way the two CPUs do instruction prefetching. The code for
normal volumes and hidden volumes is very close together, so the attack is
easily foiled by prefetching. System 2's processor seems to prefetch backwards
and reads the normal volume \texttt{GetDataSize()} code when the hidden volume
\texttt{GetDataSize()} is executing. This makes it impossible to distinguish the
two cases on System 2 with these probes.

We only tried the attack with TrueCrypt's command-line interface. The attack
should extend to the graphical interface, since both interfaces are frontends to
the same volume mounting code. 

\subsection{Links}

Links is a command-line web browser. We are interested in learning which web
page a user is visiting, out of a set of known pages. We found that it is
possible to reliably distinguish between the top 100 Wikipedia pages.

\subsubsection{Attack Implementation}

The attack works by placing probes on key locations in Links' HTML parsing code
(described below). An ad-hoc machine learning technique is used to recognize the
visited page from a single observed probe sequence.

First, the attacker performs a training phase on the victim's
machine\footnote{It is conceivable that the attacker could perform the training
phase on a different machine with similar hardware, but we did not try it.}.
After the training phase, the attacker observes the victim visiting a web page,
saving the probe sequence. Finally, there is a recovery phase to decide which
page the victim visited.

In the training phase, the attacker runs the attack tool while they themselves
visit each page in the set $T$ times. They save the observed probe sequences to
files, labeled with the page that they correspond to.

In the observation phase, the attacker spies on the Links program while the
victim visits one of the pages in the set (it is assumed the attacker knows the
page they visited is in the set). They write the observed sequence of probes to
a file.

Finally, in the recovery phase, the attacker attempts to decide which page the
victim visited. This is done by computing the Levenshtein distance
\cite{levenshtein1966binary} between the victim sequence and all of the training
sequences. The training sequence with the shortest Levenshtein distance from the
victim's sequence is assumed to be from the same page. To speed up the attack,
we only compute the distance on the first 1000 characters of the recorded
sequences\footnote{We made this change after the System 1 experiments, so this
only applies to the System 2 experiments. The System 1 experiments we report
computed the Levenshtein distance over the entire strings. We need to re-run the
System 1 experiments with the truncation.}. The recovery phase can be performed
offline.

The probe sequences are recorded by taking the attack tool's output as a string,
which is a sequence of single-character probe names. The first 200 bytes of
a training sample are shown in Figure \ref{figure:probetext}.

\begin{figure}
    \centering
\begin{verbatim}
HDRDSSSSDSDSDHDHDHDHDRDHDSSDSDSDSDHDHDHD
RDRDRDHSRDRDRDRDRDSSDHDRDRDSDSDRDRDRDHDH
DRDHDRDRDRDSDSDSDHSDSRDSDRSSDHSSDSSDSDRD
SDSDSDSDSDRDRDHSDSSSDSSRDSDRSSSDHDSDSDRS
DSDSDRDSDRDRDSSDSDSDSDRDSDSDSDSDSDSRDRDS
\end{verbatim}
\caption{The first 200 bytes (of 12,309) of a training sequence against the
Facebook Wikipedia page. From run links/0010.}
    \label{figure:probetext}
\end{figure}

We implemented this attack against the top 100 Wikipedia pages
\cite{wikitop2013}. The probe locations we chose are as follows.

\begin{enumerate}
    \item At the top of the \texttt{parse\_html()} function, which begins the
        process of parsing downloaded HTML.
    \item At the top of \texttt{html\_stack\_dup()}, which duplicates the element
        at the top of a stack used while parsing HTML tags.
    \item On \texttt{html\_h()}, the function that parses HTML header tags (h1,
        h2, h3...).
    \item On \texttt{html\_span()}, the function that parses span tags.
\end{enumerate}

These were chosen by trial and error to maximize the accuracy of the attack. We
aimed for locations in the parsing code that were far enough apart from each
other in the binary to reduce the amount of noise (e.g. from the CPU prefetching
code it won't actually execute), and whose order of execution was likely to vary
between Wikipedia pages.

\subsubsection{Experiment}

Our experiment carries out the attack described above. First the attacker takes
$T$ training samples of each page. Then, $S$ times, the victim visits
a particular page. The attacker tries to recover the page independently for each
of the $S$ observations of the victim. This is repeated for all of the pages in
the set. We used the top 100 Wikipedia pages of 2013 as our page set
\cite{wikitop2013}.

On System 1, with $T=10$ and $S=1$, the correct URL was recovered 91\% of the
time\footnote{Run links/0005}. With $T=5$ and $S=1$, the correct URL was
recovered 76\% of the time\footnote{Run links/0002}. So the attack is more
accurate when the attacker has more training samples.

We ran a longer experiment\footnote{Run links/0010} on System 2 with $T=5$ and
$S=10$. The attacker guessed correctly 879 times out of 1000, an 88\% accuracy.
Every page was correctly identified at least once except for Taylor Swift's
Wikipedia page. 43 of the pages where correctly identified all 10 times out of
10. This is plotted in Figure \ref{figure:identifycounts}, showing that most
pages can be reliably identified.

\begin{figure}
    \centering
    % GNUPLOT: LaTeX picture
\setlength{\unitlength}{0.240900pt}
\ifx\plotpoint\undefined\newsavebox{\plotpoint}\fi
\sbox{\plotpoint}{\rule[-0.200pt]{0.400pt}{0.400pt}}%
\begin{picture}(750,750)(0,0)
\sbox{\plotpoint}{\rule[-0.200pt]{0.400pt}{0.400pt}}%
\put(151.0,131.0){\rule[-0.200pt]{4.818pt}{0.400pt}}
\put(131,131){\makebox(0,0)[r]{ 0}}
\put(669.0,131.0){\rule[-0.200pt]{4.818pt}{0.400pt}}
\put(151.0,247.0){\rule[-0.200pt]{4.818pt}{0.400pt}}
\put(131,247){\makebox(0,0)[r]{ 10}}
\put(669.0,247.0){\rule[-0.200pt]{4.818pt}{0.400pt}}
\put(151.0,362.0){\rule[-0.200pt]{4.818pt}{0.400pt}}
\put(131,362){\makebox(0,0)[r]{ 20}}
\put(669.0,362.0){\rule[-0.200pt]{4.818pt}{0.400pt}}
\put(151.0,478.0){\rule[-0.200pt]{4.818pt}{0.400pt}}
\put(131,478){\makebox(0,0)[r]{ 30}}
\put(669.0,478.0){\rule[-0.200pt]{4.818pt}{0.400pt}}
\put(151.0,593.0){\rule[-0.200pt]{4.818pt}{0.400pt}}
\put(131,593){\makebox(0,0)[r]{ 40}}
\put(669.0,593.0){\rule[-0.200pt]{4.818pt}{0.400pt}}
\put(151.0,709.0){\rule[-0.200pt]{4.818pt}{0.400pt}}
\put(131,709){\makebox(0,0)[r]{ 50}}
\put(669.0,709.0){\rule[-0.200pt]{4.818pt}{0.400pt}}
\put(175,131){\usebox{\plotpoint}}
\put(175,709){\usebox{\plotpoint}}
\put(175,90){\makebox(0,0){10}}
\put(224,131){\usebox{\plotpoint}}
\put(224,709){\usebox{\plotpoint}}
\put(224,90){\makebox(0,0){9}}
\put(273,131){\usebox{\plotpoint}}
\put(273,709){\usebox{\plotpoint}}
\put(273,90){\makebox(0,0){8}}
\put(322,131){\usebox{\plotpoint}}
\put(322,709){\usebox{\plotpoint}}
\put(322,90){\makebox(0,0){7}}
\put(371,131){\usebox{\plotpoint}}
\put(371,709){\usebox{\plotpoint}}
\put(371,90){\makebox(0,0){6}}
\put(420,131){\usebox{\plotpoint}}
\put(420,709){\usebox{\plotpoint}}
\put(420,90){\makebox(0,0){5}}
\put(469,131){\usebox{\plotpoint}}
\put(469,709){\usebox{\plotpoint}}
\put(469,90){\makebox(0,0){4}}
\put(518,131){\usebox{\plotpoint}}
\put(518,709){\usebox{\plotpoint}}
\put(518,90){\makebox(0,0){3}}
\put(567,131){\usebox{\plotpoint}}
\put(567,709){\usebox{\plotpoint}}
\put(567,90){\makebox(0,0){2}}
\put(616,131){\usebox{\plotpoint}}
\put(616,709){\usebox{\plotpoint}}
\put(616,90){\makebox(0,0){1}}
\put(665,131){\usebox{\plotpoint}}
\put(665,709){\usebox{\plotpoint}}
\put(665,90){\makebox(0,0){0}}
\put(151.0,131.0){\rule[-0.200pt]{0.400pt}{139.240pt}}
\put(151.0,131.0){\rule[-0.200pt]{129.604pt}{0.400pt}}
\put(689.0,131.0){\rule[-0.200pt]{0.400pt}{139.240pt}}
\put(151.0,709.0){\rule[-0.200pt]{129.604pt}{0.400pt}}
\put(30,420){\makebox(0,0){\rotatebox{90}{Number of pages}}}
\put(420,29){\makebox(0,0){Times correctly identified of 10}}
\put(151.0,131.0){\rule[-0.200pt]{0.400pt}{119.727pt}}
\put(151.0,628.0){\rule[-0.200pt]{11.804pt}{0.400pt}}
\put(200.0,131.0){\rule[-0.200pt]{0.400pt}{119.727pt}}
\put(151.0,131.0){\rule[-0.200pt]{11.804pt}{0.400pt}}
\put(200.0,131.0){\rule[-0.200pt]{0.400pt}{91.783pt}}
\put(200.0,512.0){\rule[-0.200pt]{11.804pt}{0.400pt}}
\put(249.0,131.0){\rule[-0.200pt]{0.400pt}{91.783pt}}
\put(200.0,131.0){\rule[-0.200pt]{11.804pt}{0.400pt}}
\put(249.0,131.0){\rule[-0.200pt]{0.400pt}{22.163pt}}
\put(249.0,223.0){\rule[-0.200pt]{11.804pt}{0.400pt}}
\put(298.0,131.0){\rule[-0.200pt]{0.400pt}{22.163pt}}
\put(249.0,131.0){\rule[-0.200pt]{11.804pt}{0.400pt}}
\put(298.0,131.0){\rule[-0.200pt]{0.400pt}{19.513pt}}
\put(298.0,212.0){\rule[-0.200pt]{11.804pt}{0.400pt}}
\put(347.0,131.0){\rule[-0.200pt]{0.400pt}{19.513pt}}
\put(298.0,131.0){\rule[-0.200pt]{11.804pt}{0.400pt}}
\put(347.0,131.0){\rule[-0.200pt]{0.400pt}{11.081pt}}
\put(347.0,177.0){\rule[-0.200pt]{11.804pt}{0.400pt}}
\put(396.0,131.0){\rule[-0.200pt]{0.400pt}{11.081pt}}
\put(347.0,131.0){\rule[-0.200pt]{11.804pt}{0.400pt}}
\put(396.0,131.0){\rule[-0.200pt]{0.400pt}{5.541pt}}
\put(396.0,154.0){\rule[-0.200pt]{11.563pt}{0.400pt}}
\put(444.0,131.0){\rule[-0.200pt]{0.400pt}{5.541pt}}
\put(396.0,131.0){\rule[-0.200pt]{11.563pt}{0.400pt}}
\put(493.0,131.0){\rule[-0.200pt]{0.400pt}{2.891pt}}
\put(493.0,143.0){\rule[-0.200pt]{11.804pt}{0.400pt}}
\put(542.0,131.0){\rule[-0.200pt]{0.400pt}{2.891pt}}
\put(493.0,131.0){\rule[-0.200pt]{11.804pt}{0.400pt}}
\put(542.0,131.0){\rule[-0.200pt]{0.400pt}{2.891pt}}
\put(542.0,143.0){\rule[-0.200pt]{11.804pt}{0.400pt}}
\put(591.0,131.0){\rule[-0.200pt]{0.400pt}{2.891pt}}
\put(542.0,131.0){\rule[-0.200pt]{11.804pt}{0.400pt}}
\put(640.0,131.0){\rule[-0.200pt]{0.400pt}{2.891pt}}
\put(640.0,143.0){\rule[-0.200pt]{11.804pt}{0.400pt}}
\put(689.0,131.0){\rule[-0.200pt]{0.400pt}{2.891pt}}
\put(640.0,131.0){\rule[-0.200pt]{11.804pt}{0.400pt}}
\put(151.0,131.0){\rule[-0.200pt]{0.400pt}{139.240pt}}
\put(151.0,131.0){\rule[-0.200pt]{129.604pt}{0.400pt}}
\put(689.0,131.0){\rule[-0.200pt]{0.400pt}{139.240pt}}
\put(151.0,709.0){\rule[-0.200pt]{129.604pt}{0.400pt}}
\end{picture}

    \caption{The number of different
    pages that were identified a given number of times, out of the 10 victim
samples of that page. 43 pages were identified every time, and one page was
never correctly identified.}
    \label{figure:identifycounts}
\end{figure}

In that experiment, it took 70 seconds on average to recover one page from the
victim sample. The longest recovery took 80 seconds. Most of this time is spent
computing Levenshtein distances.

The plot of the Levenshtein distances betweeen a sample of the victim visiting
the YouTube Wikipedia page and all of the training samples is shown in Figure
\ref{figure:youtube}. In this case, the attacker makes the right decision since
the lowest distance corresponds to a training sample of the YouTube page.

The plot of a failed recovery attempt is shown in Figure \ref{figure:minaj}. The
attacker makes the wrong decision in this case because an Eminem training sample
has the shortest distance from the victim's sample, but the victim actually
visited the Nicki Minaj page (which is still identifiable as the second-lowest
point in the plot).

\begin{figure*}
    \centering
    % GNUPLOT: LaTeX picture
\setlength{\unitlength}{0.240900pt}
\ifx\plotpoint\undefined\newsavebox{\plotpoint}\fi
\sbox{\plotpoint}{\rule[-0.200pt]{0.400pt}{0.400pt}}%
\begin{picture}(1500,900)(0,0)
\sbox{\plotpoint}{\rule[-0.200pt]{0.400pt}{0.400pt}}%
\put(171.0,131.0){\rule[-0.200pt]{4.818pt}{0.400pt}}
\put(151,131){\makebox(0,0)[r]{ 0}}
\put(1419.0,131.0){\rule[-0.200pt]{4.818pt}{0.400pt}}
\put(171.0,222.0){\rule[-0.200pt]{4.818pt}{0.400pt}}
\put(151,222){\makebox(0,0)[r]{ 100}}
\put(1419.0,222.0){\rule[-0.200pt]{4.818pt}{0.400pt}}
\put(171.0,313.0){\rule[-0.200pt]{4.818pt}{0.400pt}}
\put(151,313){\makebox(0,0)[r]{ 200}}
\put(1419.0,313.0){\rule[-0.200pt]{4.818pt}{0.400pt}}
\put(171.0,404.0){\rule[-0.200pt]{4.818pt}{0.400pt}}
\put(151,404){\makebox(0,0)[r]{ 300}}
\put(1419.0,404.0){\rule[-0.200pt]{4.818pt}{0.400pt}}
\put(171.0,495.0){\rule[-0.200pt]{4.818pt}{0.400pt}}
\put(151,495){\makebox(0,0)[r]{ 400}}
\put(1419.0,495.0){\rule[-0.200pt]{4.818pt}{0.400pt}}
\put(171.0,586.0){\rule[-0.200pt]{4.818pt}{0.400pt}}
\put(151,586){\makebox(0,0)[r]{ 500}}
\put(1419.0,586.0){\rule[-0.200pt]{4.818pt}{0.400pt}}
\put(171.0,677.0){\rule[-0.200pt]{4.818pt}{0.400pt}}
\put(151,677){\makebox(0,0)[r]{ 600}}
\put(1419.0,677.0){\rule[-0.200pt]{4.818pt}{0.400pt}}
\put(171.0,768.0){\rule[-0.200pt]{4.818pt}{0.400pt}}
\put(151,768){\makebox(0,0)[r]{ 700}}
\put(1419.0,768.0){\rule[-0.200pt]{4.818pt}{0.400pt}}
\put(171.0,859.0){\rule[-0.200pt]{4.818pt}{0.400pt}}
\put(151,859){\makebox(0,0)[r]{ 800}}
\put(1419.0,859.0){\rule[-0.200pt]{4.818pt}{0.400pt}}
\put(171.0,131.0){\rule[-0.200pt]{0.400pt}{4.818pt}}
\put(171,90){\makebox(0,0){ 0}}
\put(171.0,839.0){\rule[-0.200pt]{0.400pt}{4.818pt}}
\put(298.0,131.0){\rule[-0.200pt]{0.400pt}{4.818pt}}
\put(298,90){\makebox(0,0){ 10}}
\put(298.0,839.0){\rule[-0.200pt]{0.400pt}{4.818pt}}
\put(425.0,131.0){\rule[-0.200pt]{0.400pt}{4.818pt}}
\put(425,90){\makebox(0,0){ 20}}
\put(425.0,839.0){\rule[-0.200pt]{0.400pt}{4.818pt}}
\put(551.0,131.0){\rule[-0.200pt]{0.400pt}{4.818pt}}
\put(551,90){\makebox(0,0){ 30}}
\put(551.0,839.0){\rule[-0.200pt]{0.400pt}{4.818pt}}
\put(678.0,131.0){\rule[-0.200pt]{0.400pt}{4.818pt}}
\put(678,90){\makebox(0,0){ 40}}
\put(678.0,839.0){\rule[-0.200pt]{0.400pt}{4.818pt}}
\put(805.0,131.0){\rule[-0.200pt]{0.400pt}{4.818pt}}
\put(805,90){\makebox(0,0){ 50}}
\put(805.0,839.0){\rule[-0.200pt]{0.400pt}{4.818pt}}
\put(932.0,131.0){\rule[-0.200pt]{0.400pt}{4.818pt}}
\put(932,90){\makebox(0,0){ 60}}
\put(932.0,839.0){\rule[-0.200pt]{0.400pt}{4.818pt}}
\put(1059.0,131.0){\rule[-0.200pt]{0.400pt}{4.818pt}}
\put(1059,90){\makebox(0,0){ 70}}
\put(1059.0,839.0){\rule[-0.200pt]{0.400pt}{4.818pt}}
\put(1185.0,131.0){\rule[-0.200pt]{0.400pt}{4.818pt}}
\put(1185,90){\makebox(0,0){ 80}}
\put(1185.0,839.0){\rule[-0.200pt]{0.400pt}{4.818pt}}
\put(1312.0,131.0){\rule[-0.200pt]{0.400pt}{4.818pt}}
\put(1312,90){\makebox(0,0){ 90}}
\put(1312.0,839.0){\rule[-0.200pt]{0.400pt}{4.818pt}}
\put(1439.0,131.0){\rule[-0.200pt]{0.400pt}{4.818pt}}
\put(1439,90){\makebox(0,0){ 100}}
\put(1439.0,839.0){\rule[-0.200pt]{0.400pt}{4.818pt}}
\put(171.0,131.0){\rule[-0.200pt]{0.400pt}{175.375pt}}
\put(171.0,131.0){\rule[-0.200pt]{305.461pt}{0.400pt}}
\put(30,495){\makebox(0,0){\rotatebox{90}{Levenshtein Distance}}}
\put(805,29){\makebox(0,0){Page}}
\put(171,338){\makebox(0,0){$+$}}
\put(184,359){\makebox(0,0){$+$}}
\put(196,400){\makebox(0,0){$+$}}
\put(209,412){\makebox(0,0){$+$}}
\put(222,385){\makebox(0,0){$+$}}
\put(234,380){\makebox(0,0){$+$}}
\put(247,432){\makebox(0,0){$+$}}
\put(260,483){\makebox(0,0){$+$}}
\put(272,434){\makebox(0,0){$+$}}
\put(285,341){\makebox(0,0){$+$}}
\put(298,431){\makebox(0,0){$+$}}
\put(310,425){\makebox(0,0){$+$}}
\put(323,365){\makebox(0,0){$+$}}
\put(336,417){\makebox(0,0){$+$}}
\put(349,361){\makebox(0,0){$+$}}
\put(361,393){\makebox(0,0){$+$}}
\put(374,450){\makebox(0,0){$+$}}
\put(387,402){\makebox(0,0){$+$}}
\put(399,345){\makebox(0,0){$+$}}
\put(412,380){\makebox(0,0){$+$}}
\put(425,400){\makebox(0,0){$+$}}
\put(437,409){\makebox(0,0){$+$}}
\put(450,439){\makebox(0,0){$+$}}
\put(463,453){\makebox(0,0){$+$}}
\put(475,353){\makebox(0,0){$+$}}
\put(488,405){\makebox(0,0){$+$}}
\put(501,480){\makebox(0,0){$+$}}
\put(513,371){\makebox(0,0){$+$}}
\put(526,783){\makebox(0,0){$+$}}
\put(539,424){\makebox(0,0){$+$}}
\put(551,409){\makebox(0,0){$+$}}
\put(564,396){\makebox(0,0){$+$}}
\put(577,429){\makebox(0,0){$+$}}
\put(589,351){\makebox(0,0){$+$}}
\put(602,391){\makebox(0,0){$+$}}
\put(615,417){\makebox(0,0){$+$}}
\put(627,346){\makebox(0,0){$+$}}
\put(640,471){\makebox(0,0){$+$}}
\put(653,452){\makebox(0,0){$+$}}
\put(666,350){\makebox(0,0){$+$}}
\put(678,434){\makebox(0,0){$+$}}
\put(691,360){\makebox(0,0){$+$}}
\put(704,359){\makebox(0,0){$+$}}
\put(716,402){\makebox(0,0){$+$}}
\put(729,361){\makebox(0,0){$+$}}
\put(742,403){\makebox(0,0){$+$}}
\put(754,471){\makebox(0,0){$+$}}
\put(767,371){\makebox(0,0){$+$}}
\put(780,349){\makebox(0,0){$+$}}
\put(792,460){\makebox(0,0){$+$}}
\put(805,356){\makebox(0,0){$+$}}
\put(818,439){\makebox(0,0){$+$}}
\put(830,466){\makebox(0,0){$+$}}
\put(843,462){\makebox(0,0){$+$}}
\put(856,444){\makebox(0,0){$+$}}
\put(868,389){\makebox(0,0){$+$}}
\put(881,420){\makebox(0,0){$+$}}
\put(894,450){\makebox(0,0){$+$}}
\put(906,387){\makebox(0,0){$+$}}
\put(919,565){\makebox(0,0){$+$}}
\put(932,419){\makebox(0,0){$+$}}
\put(944,392){\makebox(0,0){$+$}}
\put(957,376){\makebox(0,0){$+$}}
\put(970,420){\makebox(0,0){$+$}}
\put(983,392){\makebox(0,0){$+$}}
\put(995,346){\makebox(0,0){$+$}}
\put(1008,375){\makebox(0,0){$+$}}
\put(1021,285){\makebox(0,0){$+$}}
\put(1033,440){\makebox(0,0){$+$}}
\put(1046,404){\makebox(0,0){$+$}}
\put(1059,367){\makebox(0,0){$+$}}
\put(1071,409){\makebox(0,0){$+$}}
\put(1084,411){\makebox(0,0){$+$}}
\put(1097,381){\makebox(0,0){$+$}}
\put(1109,424){\makebox(0,0){$+$}}
\put(1122,376){\makebox(0,0){$+$}}
\put(1135,361){\makebox(0,0){$+$}}
\put(1147,329){\makebox(0,0){$+$}}
\put(1160,505){\makebox(0,0){$+$}}
\put(1173,343){\makebox(0,0){$+$}}
\put(1185,348){\makebox(0,0){$+$}}
\put(1198,416){\makebox(0,0){$+$}}
\put(1211,410){\makebox(0,0){$+$}}
\put(1223,388){\makebox(0,0){$+$}}
\put(1236,347){\makebox(0,0){$+$}}
\put(1249,371){\makebox(0,0){$+$}}
\put(1261,414){\makebox(0,0){$+$}}
\put(1274,499){\makebox(0,0){$+$}}
\put(1287,402){\makebox(0,0){$+$}}
\put(1300,426){\makebox(0,0){$+$}}
\put(1312,370){\makebox(0,0){$+$}}
\put(1325,403){\makebox(0,0){$+$}}
\put(1338,530){\makebox(0,0){$+$}}
\put(1350,437){\makebox(0,0){$+$}}
\put(1363,357){\makebox(0,0){$+$}}
\put(1376,342){\makebox(0,0){$+$}}
\put(1388,427){\makebox(0,0){$+$}}
\put(1401,382){\makebox(0,0){$+$}}
\put(1414,447){\makebox(0,0){$+$}}
\put(1426,346){\makebox(0,0){$+$}}
\put(171,406){\makebox(0,0){$\times$}}
\put(184,362){\makebox(0,0){$\times$}}
\put(196,410){\makebox(0,0){$\times$}}
\put(209,409){\makebox(0,0){$\times$}}
\put(222,409){\makebox(0,0){$\times$}}
\put(234,399){\makebox(0,0){$\times$}}
\put(247,458){\makebox(0,0){$\times$}}
\put(260,487){\makebox(0,0){$\times$}}
\put(272,418){\makebox(0,0){$\times$}}
\put(285,343){\makebox(0,0){$\times$}}
\put(298,428){\makebox(0,0){$\times$}}
\put(310,421){\makebox(0,0){$\times$}}
\put(323,368){\makebox(0,0){$\times$}}
\put(336,423){\makebox(0,0){$\times$}}
\put(349,377){\makebox(0,0){$\times$}}
\put(361,356){\makebox(0,0){$\times$}}
\put(374,441){\makebox(0,0){$\times$}}
\put(387,405){\makebox(0,0){$\times$}}
\put(399,345){\makebox(0,0){$\times$}}
\put(412,428){\makebox(0,0){$\times$}}
\put(425,375){\makebox(0,0){$\times$}}
\put(437,418){\makebox(0,0){$\times$}}
\put(450,430){\makebox(0,0){$\times$}}
\put(463,446){\makebox(0,0){$\times$}}
\put(475,372){\makebox(0,0){$\times$}}
\put(488,402){\makebox(0,0){$\times$}}
\put(501,473){\makebox(0,0){$\times$}}
\put(513,415){\makebox(0,0){$\times$}}
\put(526,781){\makebox(0,0){$\times$}}
\put(539,434){\makebox(0,0){$\times$}}
\put(551,388){\makebox(0,0){$\times$}}
\put(564,396){\makebox(0,0){$\times$}}
\put(577,398){\makebox(0,0){$\times$}}
\put(589,330){\makebox(0,0){$\times$}}
\put(602,385){\makebox(0,0){$\times$}}
\put(615,411){\makebox(0,0){$\times$}}
\put(627,356){\makebox(0,0){$\times$}}
\put(640,477){\makebox(0,0){$\times$}}
\put(653,457){\makebox(0,0){$\times$}}
\put(666,343){\makebox(0,0){$\times$}}
\put(678,388){\makebox(0,0){$\times$}}
\put(691,376){\makebox(0,0){$\times$}}
\put(704,368){\makebox(0,0){$\times$}}
\put(716,409){\makebox(0,0){$\times$}}
\put(729,366){\makebox(0,0){$\times$}}
\put(742,405){\makebox(0,0){$\times$}}
\put(754,465){\makebox(0,0){$\times$}}
\put(767,379){\makebox(0,0){$\times$}}
\put(780,374){\makebox(0,0){$\times$}}
\put(792,479){\makebox(0,0){$\times$}}
\put(805,370){\makebox(0,0){$\times$}}
\put(818,421){\makebox(0,0){$\times$}}
\put(830,480){\makebox(0,0){$\times$}}
\put(843,476){\makebox(0,0){$\times$}}
\put(856,450){\makebox(0,0){$\times$}}
\put(868,372){\makebox(0,0){$\times$}}
\put(881,429){\makebox(0,0){$\times$}}
\put(894,437){\makebox(0,0){$\times$}}
\put(906,367){\makebox(0,0){$\times$}}
\put(919,554){\makebox(0,0){$\times$}}
\put(932,421){\makebox(0,0){$\times$}}
\put(944,392){\makebox(0,0){$\times$}}
\put(957,370){\makebox(0,0){$\times$}}
\put(970,419){\makebox(0,0){$\times$}}
\put(983,408){\makebox(0,0){$\times$}}
\put(995,329){\makebox(0,0){$\times$}}
\put(1008,373){\makebox(0,0){$\times$}}
\put(1021,333){\makebox(0,0){$\times$}}
\put(1033,494){\makebox(0,0){$\times$}}
\put(1046,418){\makebox(0,0){$\times$}}
\put(1059,381){\makebox(0,0){$\times$}}
\put(1071,399){\makebox(0,0){$\times$}}
\put(1084,400){\makebox(0,0){$\times$}}
\put(1097,384){\makebox(0,0){$\times$}}
\put(1109,404){\makebox(0,0){$\times$}}
\put(1122,361){\makebox(0,0){$\times$}}
\put(1135,369){\makebox(0,0){$\times$}}
\put(1147,383){\makebox(0,0){$\times$}}
\put(1160,513){\makebox(0,0){$\times$}}
\put(1173,339){\makebox(0,0){$\times$}}
\put(1185,355){\makebox(0,0){$\times$}}
\put(1198,415){\makebox(0,0){$\times$}}
\put(1211,429){\makebox(0,0){$\times$}}
\put(1223,383){\makebox(0,0){$\times$}}
\put(1236,345){\makebox(0,0){$\times$}}
\put(1249,405){\makebox(0,0){$\times$}}
\put(1261,447){\makebox(0,0){$\times$}}
\put(1274,494){\makebox(0,0){$\times$}}
\put(1287,430){\makebox(0,0){$\times$}}
\put(1300,415){\makebox(0,0){$\times$}}
\put(1312,368){\makebox(0,0){$\times$}}
\put(1325,358){\makebox(0,0){$\times$}}
\put(1338,534){\makebox(0,0){$\times$}}
\put(1350,413){\makebox(0,0){$\times$}}
\put(1363,369){\makebox(0,0){$\times$}}
\put(1376,348){\makebox(0,0){$\times$}}
\put(1388,407){\makebox(0,0){$\times$}}
\put(1401,392){\makebox(0,0){$\times$}}
\put(1414,438){\makebox(0,0){$\times$}}
\put(1426,378){\makebox(0,0){$\times$}}
\sbox{\plotpoint}{\rule[-0.400pt]{0.800pt}{0.800pt}}%
\put(171,390){\makebox(0,0){$\ast$}}
\put(184,384){\makebox(0,0){$\ast$}}
\put(196,384){\makebox(0,0){$\ast$}}
\put(209,406){\makebox(0,0){$\ast$}}
\put(222,411){\makebox(0,0){$\ast$}}
\put(234,406){\makebox(0,0){$\ast$}}
\put(247,431){\makebox(0,0){$\ast$}}
\put(260,484){\makebox(0,0){$\ast$}}
\put(272,433){\makebox(0,0){$\ast$}}
\put(285,338){\makebox(0,0){$\ast$}}
\put(298,422){\makebox(0,0){$\ast$}}
\put(310,421){\makebox(0,0){$\ast$}}
\put(323,375){\makebox(0,0){$\ast$}}
\put(336,421){\makebox(0,0){$\ast$}}
\put(349,372){\makebox(0,0){$\ast$}}
\put(361,381){\makebox(0,0){$\ast$}}
\put(374,464){\makebox(0,0){$\ast$}}
\put(387,394){\makebox(0,0){$\ast$}}
\put(399,329){\makebox(0,0){$\ast$}}
\put(412,389){\makebox(0,0){$\ast$}}
\put(425,395){\makebox(0,0){$\ast$}}
\put(437,403){\makebox(0,0){$\ast$}}
\put(450,440){\makebox(0,0){$\ast$}}
\put(463,451){\makebox(0,0){$\ast$}}
\put(475,365){\makebox(0,0){$\ast$}}
\put(488,406){\makebox(0,0){$\ast$}}
\put(501,470){\makebox(0,0){$\ast$}}
\put(513,400){\makebox(0,0){$\ast$}}
\put(526,767){\makebox(0,0){$\ast$}}
\put(539,446){\makebox(0,0){$\ast$}}
\put(551,403){\makebox(0,0){$\ast$}}
\put(564,366){\makebox(0,0){$\ast$}}
\put(577,392){\makebox(0,0){$\ast$}}
\put(589,364){\makebox(0,0){$\ast$}}
\put(602,383){\makebox(0,0){$\ast$}}
\put(615,409){\makebox(0,0){$\ast$}}
\put(627,361){\makebox(0,0){$\ast$}}
\put(640,465){\makebox(0,0){$\ast$}}
\put(653,441){\makebox(0,0){$\ast$}}
\put(666,350){\makebox(0,0){$\ast$}}
\put(678,416){\makebox(0,0){$\ast$}}
\put(691,371){\makebox(0,0){$\ast$}}
\put(704,371){\makebox(0,0){$\ast$}}
\put(716,394){\makebox(0,0){$\ast$}}
\put(729,365){\makebox(0,0){$\ast$}}
\put(742,404){\makebox(0,0){$\ast$}}
\put(754,467){\makebox(0,0){$\ast$}}
\put(767,359){\makebox(0,0){$\ast$}}
\put(780,372){\makebox(0,0){$\ast$}}
\put(792,468){\makebox(0,0){$\ast$}}
\put(805,359){\makebox(0,0){$\ast$}}
\put(818,423){\makebox(0,0){$\ast$}}
\put(830,476){\makebox(0,0){$\ast$}}
\put(843,473){\makebox(0,0){$\ast$}}
\put(856,448){\makebox(0,0){$\ast$}}
\put(868,364){\makebox(0,0){$\ast$}}
\put(881,415){\makebox(0,0){$\ast$}}
\put(894,450){\makebox(0,0){$\ast$}}
\put(906,370){\makebox(0,0){$\ast$}}
\put(919,550){\makebox(0,0){$\ast$}}
\put(932,431){\makebox(0,0){$\ast$}}
\put(944,389){\makebox(0,0){$\ast$}}
\put(957,392){\makebox(0,0){$\ast$}}
\put(970,425){\makebox(0,0){$\ast$}}
\put(983,399){\makebox(0,0){$\ast$}}
\put(995,346){\makebox(0,0){$\ast$}}
\put(1008,381){\makebox(0,0){$\ast$}}
\put(1021,317){\makebox(0,0){$\ast$}}
\put(1033,464){\makebox(0,0){$\ast$}}
\put(1046,410){\makebox(0,0){$\ast$}}
\put(1059,383){\makebox(0,0){$\ast$}}
\put(1071,390){\makebox(0,0){$\ast$}}
\put(1084,418){\makebox(0,0){$\ast$}}
\put(1097,373){\makebox(0,0){$\ast$}}
\put(1109,424){\makebox(0,0){$\ast$}}
\put(1122,372){\makebox(0,0){$\ast$}}
\put(1135,364){\makebox(0,0){$\ast$}}
\put(1147,371){\makebox(0,0){$\ast$}}
\put(1160,520){\makebox(0,0){$\ast$}}
\put(1173,343){\makebox(0,0){$\ast$}}
\put(1185,359){\makebox(0,0){$\ast$}}
\put(1198,412){\makebox(0,0){$\ast$}}
\put(1211,398){\makebox(0,0){$\ast$}}
\put(1223,389){\makebox(0,0){$\ast$}}
\put(1236,353){\makebox(0,0){$\ast$}}
\put(1249,383){\makebox(0,0){$\ast$}}
\put(1261,449){\makebox(0,0){$\ast$}}
\put(1274,494){\makebox(0,0){$\ast$}}
\put(1287,384){\makebox(0,0){$\ast$}}
\put(1300,417){\makebox(0,0){$\ast$}}
\put(1312,367){\makebox(0,0){$\ast$}}
\put(1325,345){\makebox(0,0){$\ast$}}
\put(1338,528){\makebox(0,0){$\ast$}}
\put(1350,439){\makebox(0,0){$\ast$}}
\put(1363,376){\makebox(0,0){$\ast$}}
\put(1376,353){\makebox(0,0){$\ast$}}
\put(1388,401){\makebox(0,0){$\ast$}}
\put(1401,373){\makebox(0,0){$\ast$}}
\put(1414,436){\makebox(0,0){$\ast$}}
\put(1426,376){\makebox(0,0){$\ast$}}
\sbox{\plotpoint}{\rule[-0.500pt]{1.000pt}{1.000pt}}%
\put(171,359){\raisebox{-.8pt}{\makebox(0,0){$\Box$}}}
\put(184,376){\raisebox{-.8pt}{\makebox(0,0){$\Box$}}}
\put(196,396){\raisebox{-.8pt}{\makebox(0,0){$\Box$}}}
\put(209,415){\raisebox{-.8pt}{\makebox(0,0){$\Box$}}}
\put(222,406){\raisebox{-.8pt}{\makebox(0,0){$\Box$}}}
\put(234,400){\raisebox{-.8pt}{\makebox(0,0){$\Box$}}}
\put(247,439){\raisebox{-.8pt}{\makebox(0,0){$\Box$}}}
\put(260,492){\raisebox{-.8pt}{\makebox(0,0){$\Box$}}}
\put(272,430){\raisebox{-.8pt}{\makebox(0,0){$\Box$}}}
\put(285,379){\raisebox{-.8pt}{\makebox(0,0){$\Box$}}}
\put(298,426){\raisebox{-.8pt}{\makebox(0,0){$\Box$}}}
\put(310,405){\raisebox{-.8pt}{\makebox(0,0){$\Box$}}}
\put(323,370){\raisebox{-.8pt}{\makebox(0,0){$\Box$}}}
\put(336,419){\raisebox{-.8pt}{\makebox(0,0){$\Box$}}}
\put(349,357){\raisebox{-.8pt}{\makebox(0,0){$\Box$}}}
\put(361,367){\raisebox{-.8pt}{\makebox(0,0){$\Box$}}}
\put(374,429){\raisebox{-.8pt}{\makebox(0,0){$\Box$}}}
\put(387,480){\raisebox{-.8pt}{\makebox(0,0){$\Box$}}}
\put(399,345){\raisebox{-.8pt}{\makebox(0,0){$\Box$}}}
\put(412,393){\raisebox{-.8pt}{\makebox(0,0){$\Box$}}}
\put(425,406){\raisebox{-.8pt}{\makebox(0,0){$\Box$}}}
\put(437,410){\raisebox{-.8pt}{\makebox(0,0){$\Box$}}}
\put(450,433){\raisebox{-.8pt}{\makebox(0,0){$\Box$}}}
\put(463,448){\raisebox{-.8pt}{\makebox(0,0){$\Box$}}}
\put(475,350){\raisebox{-.8pt}{\makebox(0,0){$\Box$}}}
\put(488,410){\raisebox{-.8pt}{\makebox(0,0){$\Box$}}}
\put(501,481){\raisebox{-.8pt}{\makebox(0,0){$\Box$}}}
\put(513,408){\raisebox{-.8pt}{\makebox(0,0){$\Box$}}}
\put(526,776){\raisebox{-.8pt}{\makebox(0,0){$\Box$}}}
\put(539,423){\raisebox{-.8pt}{\makebox(0,0){$\Box$}}}
\put(551,415){\raisebox{-.8pt}{\makebox(0,0){$\Box$}}}
\put(564,383){\raisebox{-.8pt}{\makebox(0,0){$\Box$}}}
\put(577,359){\raisebox{-.8pt}{\makebox(0,0){$\Box$}}}
\put(589,339){\raisebox{-.8pt}{\makebox(0,0){$\Box$}}}
\put(602,372){\raisebox{-.8pt}{\makebox(0,0){$\Box$}}}
\put(615,411){\raisebox{-.8pt}{\makebox(0,0){$\Box$}}}
\put(627,351){\raisebox{-.8pt}{\makebox(0,0){$\Box$}}}
\put(640,472){\raisebox{-.8pt}{\makebox(0,0){$\Box$}}}
\put(653,440){\raisebox{-.8pt}{\makebox(0,0){$\Box$}}}
\put(666,378){\raisebox{-.8pt}{\makebox(0,0){$\Box$}}}
\put(678,381){\raisebox{-.8pt}{\makebox(0,0){$\Box$}}}
\put(691,370){\raisebox{-.8pt}{\makebox(0,0){$\Box$}}}
\put(704,398){\raisebox{-.8pt}{\makebox(0,0){$\Box$}}}
\put(716,394){\raisebox{-.8pt}{\makebox(0,0){$\Box$}}}
\put(729,368){\raisebox{-.8pt}{\makebox(0,0){$\Box$}}}
\put(742,401){\raisebox{-.8pt}{\makebox(0,0){$\Box$}}}
\put(754,466){\raisebox{-.8pt}{\makebox(0,0){$\Box$}}}
\put(767,419){\raisebox{-.8pt}{\makebox(0,0){$\Box$}}}
\put(780,413){\raisebox{-.8pt}{\makebox(0,0){$\Box$}}}
\put(792,468){\raisebox{-.8pt}{\makebox(0,0){$\Box$}}}
\put(805,375){\raisebox{-.8pt}{\makebox(0,0){$\Box$}}}
\put(818,426){\raisebox{-.8pt}{\makebox(0,0){$\Box$}}}
\put(830,463){\raisebox{-.8pt}{\makebox(0,0){$\Box$}}}
\put(843,467){\raisebox{-.8pt}{\makebox(0,0){$\Box$}}}
\put(856,445){\raisebox{-.8pt}{\makebox(0,0){$\Box$}}}
\put(868,359){\raisebox{-.8pt}{\makebox(0,0){$\Box$}}}
\put(881,404){\raisebox{-.8pt}{\makebox(0,0){$\Box$}}}
\put(894,465){\raisebox{-.8pt}{\makebox(0,0){$\Box$}}}
\put(906,373){\raisebox{-.8pt}{\makebox(0,0){$\Box$}}}
\put(919,532){\raisebox{-.8pt}{\makebox(0,0){$\Box$}}}
\put(932,405){\raisebox{-.8pt}{\makebox(0,0){$\Box$}}}
\put(944,399){\raisebox{-.8pt}{\makebox(0,0){$\Box$}}}
\put(957,374){\raisebox{-.8pt}{\makebox(0,0){$\Box$}}}
\put(970,460){\raisebox{-.8pt}{\makebox(0,0){$\Box$}}}
\put(983,401){\raisebox{-.8pt}{\makebox(0,0){$\Box$}}}
\put(995,366){\raisebox{-.8pt}{\makebox(0,0){$\Box$}}}
\put(1008,380){\raisebox{-.8pt}{\makebox(0,0){$\Box$}}}
\put(1021,301){\raisebox{-.8pt}{\makebox(0,0){$\Box$}}}
\put(1033,486){\raisebox{-.8pt}{\makebox(0,0){$\Box$}}}
\put(1046,419){\raisebox{-.8pt}{\makebox(0,0){$\Box$}}}
\put(1059,390){\raisebox{-.8pt}{\makebox(0,0){$\Box$}}}
\put(1071,380){\raisebox{-.8pt}{\makebox(0,0){$\Box$}}}
\put(1084,365){\raisebox{-.8pt}{\makebox(0,0){$\Box$}}}
\put(1097,380){\raisebox{-.8pt}{\makebox(0,0){$\Box$}}}
\put(1109,404){\raisebox{-.8pt}{\makebox(0,0){$\Box$}}}
\put(1122,386){\raisebox{-.8pt}{\makebox(0,0){$\Box$}}}
\put(1135,362){\raisebox{-.8pt}{\makebox(0,0){$\Box$}}}
\put(1147,331){\raisebox{-.8pt}{\makebox(0,0){$\Box$}}}
\put(1160,514){\raisebox{-.8pt}{\makebox(0,0){$\Box$}}}
\put(1173,352){\raisebox{-.8pt}{\makebox(0,0){$\Box$}}}
\put(1185,372){\raisebox{-.8pt}{\makebox(0,0){$\Box$}}}
\put(1198,407){\raisebox{-.8pt}{\makebox(0,0){$\Box$}}}
\put(1211,371){\raisebox{-.8pt}{\makebox(0,0){$\Box$}}}
\put(1223,389){\raisebox{-.8pt}{\makebox(0,0){$\Box$}}}
\put(1236,338){\raisebox{-.8pt}{\makebox(0,0){$\Box$}}}
\put(1249,390){\raisebox{-.8pt}{\makebox(0,0){$\Box$}}}
\put(1261,449){\raisebox{-.8pt}{\makebox(0,0){$\Box$}}}
\put(1274,486){\raisebox{-.8pt}{\makebox(0,0){$\Box$}}}
\put(1287,381){\raisebox{-.8pt}{\makebox(0,0){$\Box$}}}
\put(1300,432){\raisebox{-.8pt}{\makebox(0,0){$\Box$}}}
\put(1312,366){\raisebox{-.8pt}{\makebox(0,0){$\Box$}}}
\put(1325,361){\raisebox{-.8pt}{\makebox(0,0){$\Box$}}}
\put(1338,528){\raisebox{-.8pt}{\makebox(0,0){$\Box$}}}
\put(1350,440){\raisebox{-.8pt}{\makebox(0,0){$\Box$}}}
\put(1363,425){\raisebox{-.8pt}{\makebox(0,0){$\Box$}}}
\put(1376,358){\raisebox{-.8pt}{\makebox(0,0){$\Box$}}}
\put(1388,399){\raisebox{-.8pt}{\makebox(0,0){$\Box$}}}
\put(1401,369){\raisebox{-.8pt}{\makebox(0,0){$\Box$}}}
\put(1414,430){\raisebox{-.8pt}{\makebox(0,0){$\Box$}}}
\put(1426,395){\raisebox{-.8pt}{\makebox(0,0){$\Box$}}}
\sbox{\plotpoint}{\rule[-0.600pt]{1.200pt}{1.200pt}}%
\put(171,367){\makebox(0,0){$\blacksquare$}}
\put(184,359){\makebox(0,0){$\blacksquare$}}
\put(196,403){\makebox(0,0){$\blacksquare$}}
\put(209,410){\makebox(0,0){$\blacksquare$}}
\put(222,413){\makebox(0,0){$\blacksquare$}}
\put(234,394){\makebox(0,0){$\blacksquare$}}
\put(247,470){\makebox(0,0){$\blacksquare$}}
\put(260,465){\makebox(0,0){$\blacksquare$}}
\put(272,431){\makebox(0,0){$\blacksquare$}}
\put(285,340){\makebox(0,0){$\blacksquare$}}
\put(298,433){\makebox(0,0){$\blacksquare$}}
\put(310,422){\makebox(0,0){$\blacksquare$}}
\put(323,372){\makebox(0,0){$\blacksquare$}}
\put(336,417){\makebox(0,0){$\blacksquare$}}
\put(349,365){\makebox(0,0){$\blacksquare$}}
\put(361,347){\makebox(0,0){$\blacksquare$}}
\put(374,410){\makebox(0,0){$\blacksquare$}}
\put(387,406){\makebox(0,0){$\blacksquare$}}
\put(399,346){\makebox(0,0){$\blacksquare$}}
\put(412,386){\makebox(0,0){$\blacksquare$}}
\put(425,386){\makebox(0,0){$\blacksquare$}}
\put(437,414){\makebox(0,0){$\blacksquare$}}
\put(450,431){\makebox(0,0){$\blacksquare$}}
\put(463,460){\makebox(0,0){$\blacksquare$}}
\put(475,332){\makebox(0,0){$\blacksquare$}}
\put(488,399){\makebox(0,0){$\blacksquare$}}
\put(501,475){\makebox(0,0){$\blacksquare$}}
\put(513,392){\makebox(0,0){$\blacksquare$}}
\put(526,776){\makebox(0,0){$\blacksquare$}}
\put(539,420){\makebox(0,0){$\blacksquare$}}
\put(551,406){\makebox(0,0){$\blacksquare$}}
\put(564,395){\makebox(0,0){$\blacksquare$}}
\put(577,411){\makebox(0,0){$\blacksquare$}}
\put(589,338){\makebox(0,0){$\blacksquare$}}
\put(602,386){\makebox(0,0){$\blacksquare$}}
\put(615,419){\makebox(0,0){$\blacksquare$}}
\put(627,376){\makebox(0,0){$\blacksquare$}}
\put(640,480){\makebox(0,0){$\blacksquare$}}
\put(653,462){\makebox(0,0){$\blacksquare$}}
\put(666,348){\makebox(0,0){$\blacksquare$}}
\put(678,411){\makebox(0,0){$\blacksquare$}}
\put(691,363){\makebox(0,0){$\blacksquare$}}
\put(704,347){\makebox(0,0){$\blacksquare$}}
\put(716,394){\makebox(0,0){$\blacksquare$}}
\put(729,356){\makebox(0,0){$\blacksquare$}}
\put(742,393){\makebox(0,0){$\blacksquare$}}
\put(754,473){\makebox(0,0){$\blacksquare$}}
\put(767,411){\makebox(0,0){$\blacksquare$}}
\put(780,371){\makebox(0,0){$\blacksquare$}}
\put(792,482){\makebox(0,0){$\blacksquare$}}
\put(805,374){\makebox(0,0){$\blacksquare$}}
\put(818,423){\makebox(0,0){$\blacksquare$}}
\put(830,466){\makebox(0,0){$\blacksquare$}}
\put(843,479){\makebox(0,0){$\blacksquare$}}
\put(856,445){\makebox(0,0){$\blacksquare$}}
\put(868,408){\makebox(0,0){$\blacksquare$}}
\put(881,403){\makebox(0,0){$\blacksquare$}}
\put(894,458){\makebox(0,0){$\blacksquare$}}
\put(906,379){\makebox(0,0){$\blacksquare$}}
\put(919,582){\makebox(0,0){$\blacksquare$}}
\put(932,412){\makebox(0,0){$\blacksquare$}}
\put(944,389){\makebox(0,0){$\blacksquare$}}
\put(957,387){\makebox(0,0){$\blacksquare$}}
\put(970,417){\makebox(0,0){$\blacksquare$}}
\put(983,407){\makebox(0,0){$\blacksquare$}}
\put(995,353){\makebox(0,0){$\blacksquare$}}
\put(1008,380){\makebox(0,0){$\blacksquare$}}
\put(1021,299){\makebox(0,0){$\blacksquare$}}
\put(1033,467){\makebox(0,0){$\blacksquare$}}
\put(1046,411){\makebox(0,0){$\blacksquare$}}
\put(1059,367){\makebox(0,0){$\blacksquare$}}
\put(1071,415){\makebox(0,0){$\blacksquare$}}
\put(1084,419){\makebox(0,0){$\blacksquare$}}
\put(1097,388){\makebox(0,0){$\blacksquare$}}
\put(1109,411){\makebox(0,0){$\blacksquare$}}
\put(1122,400){\makebox(0,0){$\blacksquare$}}
\put(1135,363){\makebox(0,0){$\blacksquare$}}
\put(1147,339){\makebox(0,0){$\blacksquare$}}
\put(1160,492){\makebox(0,0){$\blacksquare$}}
\put(1173,338){\makebox(0,0){$\blacksquare$}}
\put(1185,365){\makebox(0,0){$\blacksquare$}}
\put(1198,422){\makebox(0,0){$\blacksquare$}}
\put(1211,428){\makebox(0,0){$\blacksquare$}}
\put(1223,386){\makebox(0,0){$\blacksquare$}}
\put(1236,347){\makebox(0,0){$\blacksquare$}}
\put(1249,389){\makebox(0,0){$\blacksquare$}}
\put(1261,427){\makebox(0,0){$\blacksquare$}}
\put(1274,486){\makebox(0,0){$\blacksquare$}}
\put(1287,401){\makebox(0,0){$\blacksquare$}}
\put(1300,412){\makebox(0,0){$\blacksquare$}}
\put(1312,361){\makebox(0,0){$\blacksquare$}}
\put(1325,360){\makebox(0,0){$\blacksquare$}}
\put(1338,533){\makebox(0,0){$\blacksquare$}}
\put(1350,438){\makebox(0,0){$\blacksquare$}}
\put(1363,374){\makebox(0,0){$\blacksquare$}}
\put(1376,373){\makebox(0,0){$\blacksquare$}}
\put(1388,440){\makebox(0,0){$\blacksquare$}}
\put(1401,369){\makebox(0,0){$\blacksquare$}}
\put(1414,447){\makebox(0,0){$\blacksquare$}}
\put(1426,371){\makebox(0,0){$\blacksquare$}}
\sbox{\plotpoint}{\rule[-0.200pt]{0.400pt}{0.400pt}}%
\put(171.0,131.0){\rule[-0.200pt]{0.400pt}{175.375pt}}
\put(171.0,131.0){\rule[-0.200pt]{305.461pt}{0.400pt}}
\end{picture}

    \caption{A successful recovery. The Levenshtein distance between the
    training samples and a recording of the victim visiting the YouTube
    Wikipedia page. The shortest distance is visible at mark 68 on the page axis
    which corresponds to a YouTube training sample. The outlier at mark 29
    corresponds to a disambiguation page that has a different format from the
usual Wikipedia page. The different shapes in a column represent the five
training samples of that page. The order on the page axis is not meaningful.}
    \label{figure:youtube}
\end{figure*}

\begin{figure*}
    \centering
    % GNUPLOT: LaTeX picture
\setlength{\unitlength}{0.240900pt}
\ifx\plotpoint\undefined\newsavebox{\plotpoint}\fi
\sbox{\plotpoint}{\rule[-0.200pt]{0.400pt}{0.400pt}}%
\begin{picture}(1500,900)(0,0)
\sbox{\plotpoint}{\rule[-0.200pt]{0.400pt}{0.400pt}}%
\put(171.0,131.0){\rule[-0.200pt]{4.818pt}{0.400pt}}
\put(151,131){\makebox(0,0)[r]{ 0}}
\put(1419.0,131.0){\rule[-0.200pt]{4.818pt}{0.400pt}}
\put(171.0,277.0){\rule[-0.200pt]{4.818pt}{0.400pt}}
\put(151,277){\makebox(0,0)[r]{ 100}}
\put(1419.0,277.0){\rule[-0.200pt]{4.818pt}{0.400pt}}
\put(171.0,422.0){\rule[-0.200pt]{4.818pt}{0.400pt}}
\put(151,422){\makebox(0,0)[r]{ 200}}
\put(1419.0,422.0){\rule[-0.200pt]{4.818pt}{0.400pt}}
\put(171.0,568.0){\rule[-0.200pt]{4.818pt}{0.400pt}}
\put(151,568){\makebox(0,0)[r]{ 300}}
\put(1419.0,568.0){\rule[-0.200pt]{4.818pt}{0.400pt}}
\put(171.0,713.0){\rule[-0.200pt]{4.818pt}{0.400pt}}
\put(151,713){\makebox(0,0)[r]{ 400}}
\put(1419.0,713.0){\rule[-0.200pt]{4.818pt}{0.400pt}}
\put(171.0,859.0){\rule[-0.200pt]{4.818pt}{0.400pt}}
\put(151,859){\makebox(0,0)[r]{ 500}}
\put(1419.0,859.0){\rule[-0.200pt]{4.818pt}{0.400pt}}
\put(171.0,131.0){\rule[-0.200pt]{0.400pt}{4.818pt}}
\put(171,90){\makebox(0,0){ 0}}
\put(171.0,839.0){\rule[-0.200pt]{0.400pt}{4.818pt}}
\put(298.0,131.0){\rule[-0.200pt]{0.400pt}{4.818pt}}
\put(298,90){\makebox(0,0){ 10}}
\put(298.0,839.0){\rule[-0.200pt]{0.400pt}{4.818pt}}
\put(425.0,131.0){\rule[-0.200pt]{0.400pt}{4.818pt}}
\put(425,90){\makebox(0,0){ 20}}
\put(425.0,839.0){\rule[-0.200pt]{0.400pt}{4.818pt}}
\put(551.0,131.0){\rule[-0.200pt]{0.400pt}{4.818pt}}
\put(551,90){\makebox(0,0){ 30}}
\put(551.0,839.0){\rule[-0.200pt]{0.400pt}{4.818pt}}
\put(678.0,131.0){\rule[-0.200pt]{0.400pt}{4.818pt}}
\put(678,90){\makebox(0,0){ 40}}
\put(678.0,839.0){\rule[-0.200pt]{0.400pt}{4.818pt}}
\put(805.0,131.0){\rule[-0.200pt]{0.400pt}{4.818pt}}
\put(805,90){\makebox(0,0){ 50}}
\put(805.0,839.0){\rule[-0.200pt]{0.400pt}{4.818pt}}
\put(932.0,131.0){\rule[-0.200pt]{0.400pt}{4.818pt}}
\put(932,90){\makebox(0,0){ 60}}
\put(932.0,839.0){\rule[-0.200pt]{0.400pt}{4.818pt}}
\put(1059.0,131.0){\rule[-0.200pt]{0.400pt}{4.818pt}}
\put(1059,90){\makebox(0,0){ 70}}
\put(1059.0,839.0){\rule[-0.200pt]{0.400pt}{4.818pt}}
\put(1185.0,131.0){\rule[-0.200pt]{0.400pt}{4.818pt}}
\put(1185,90){\makebox(0,0){ 80}}
\put(1185.0,839.0){\rule[-0.200pt]{0.400pt}{4.818pt}}
\put(1312.0,131.0){\rule[-0.200pt]{0.400pt}{4.818pt}}
\put(1312,90){\makebox(0,0){ 90}}
\put(1312.0,839.0){\rule[-0.200pt]{0.400pt}{4.818pt}}
\put(1439.0,131.0){\rule[-0.200pt]{0.400pt}{4.818pt}}
\put(1439,90){\makebox(0,0){ 100}}
\put(1439.0,839.0){\rule[-0.200pt]{0.400pt}{4.818pt}}
\put(171.0,131.0){\rule[-0.200pt]{0.400pt}{175.375pt}}
\put(171.0,131.0){\rule[-0.200pt]{305.461pt}{0.400pt}}
\put(30,495){\makebox(0,0){\rotatebox{90}{Levenshtein Distance}}}
\put(805,29){\makebox(0,0){Page}}
\put(171,457){\makebox(0,0){$+$}}
\put(184,495){\makebox(0,0){$+$}}
\put(196,600){\makebox(0,0){$+$}}
\put(209,620){\makebox(0,0){$+$}}
\put(222,464){\makebox(0,0){$+$}}
\put(234,563){\makebox(0,0){$+$}}
\put(247,546){\makebox(0,0){$+$}}
\put(260,690){\makebox(0,0){$+$}}
\put(272,627){\makebox(0,0){$+$}}
\put(285,463){\makebox(0,0){$+$}}
\put(298,598){\makebox(0,0){$+$}}
\put(310,575){\makebox(0,0){$+$}}
\put(323,547){\makebox(0,0){$+$}}
\put(336,601){\makebox(0,0){$+$}}
\put(349,531){\makebox(0,0){$+$}}
\put(361,445){\makebox(0,0){$+$}}
\put(374,582){\makebox(0,0){$+$}}
\put(387,611){\makebox(0,0){$+$}}
\put(399,488){\makebox(0,0){$+$}}
\put(412,561){\makebox(0,0){$+$}}
\put(425,540){\makebox(0,0){$+$}}
\put(437,582){\makebox(0,0){$+$}}
\put(450,619){\makebox(0,0){$+$}}
\put(463,613){\makebox(0,0){$+$}}
\put(475,507){\makebox(0,0){$+$}}
\put(488,547){\makebox(0,0){$+$}}
\put(501,660){\makebox(0,0){$+$}}
\put(513,486){\makebox(0,0){$+$}}
\put(539,577){\makebox(0,0){$+$}}
\put(551,508){\makebox(0,0){$+$}}
\put(564,533){\makebox(0,0){$+$}}
\put(577,533){\makebox(0,0){$+$}}
\put(589,478){\makebox(0,0){$+$}}
\put(602,555){\makebox(0,0){$+$}}
\put(615,552){\makebox(0,0){$+$}}
\put(627,514){\makebox(0,0){$+$}}
\put(640,658){\makebox(0,0){$+$}}
\put(653,593){\makebox(0,0){$+$}}
\put(666,549){\makebox(0,0){$+$}}
\put(678,609){\makebox(0,0){$+$}}
\put(691,499){\makebox(0,0){$+$}}
\put(704,533){\makebox(0,0){$+$}}
\put(716,489){\makebox(0,0){$+$}}
\put(729,521){\makebox(0,0){$+$}}
\put(742,575){\makebox(0,0){$+$}}
\put(754,654){\makebox(0,0){$+$}}
\put(767,501){\makebox(0,0){$+$}}
\put(780,496){\makebox(0,0){$+$}}
\put(792,639){\makebox(0,0){$+$}}
\put(805,578){\makebox(0,0){$+$}}
\put(818,593){\makebox(0,0){$+$}}
\put(830,600){\makebox(0,0){$+$}}
\put(843,582){\makebox(0,0){$+$}}
\put(856,514){\makebox(0,0){$+$}}
\put(868,424){\makebox(0,0){$+$}}
\put(881,593){\makebox(0,0){$+$}}
\put(894,584){\makebox(0,0){$+$}}
\put(906,533){\makebox(0,0){$+$}}
\put(932,671){\makebox(0,0){$+$}}
\put(944,534){\makebox(0,0){$+$}}
\put(957,476){\makebox(0,0){$+$}}
\put(970,677){\makebox(0,0){$+$}}
\put(983,528){\makebox(0,0){$+$}}
\put(995,547){\makebox(0,0){$+$}}
\put(1008,518){\makebox(0,0){$+$}}
\put(1021,483){\makebox(0,0){$+$}}
\put(1033,574){\makebox(0,0){$+$}}
\put(1046,530){\makebox(0,0){$+$}}
\put(1059,496){\makebox(0,0){$+$}}
\put(1071,616){\makebox(0,0){$+$}}
\put(1084,534){\makebox(0,0){$+$}}
\put(1097,555){\makebox(0,0){$+$}}
\put(1109,594){\makebox(0,0){$+$}}
\put(1122,478){\makebox(0,0){$+$}}
\put(1135,562){\makebox(0,0){$+$}}
\put(1147,534){\makebox(0,0){$+$}}
\put(1160,638){\makebox(0,0){$+$}}
\put(1173,552){\makebox(0,0){$+$}}
\put(1185,545){\makebox(0,0){$+$}}
\put(1198,526){\makebox(0,0){$+$}}
\put(1211,536){\makebox(0,0){$+$}}
\put(1223,539){\makebox(0,0){$+$}}
\put(1236,496){\makebox(0,0){$+$}}
\put(1249,534){\makebox(0,0){$+$}}
\put(1261,578){\makebox(0,0){$+$}}
\put(1274,690){\makebox(0,0){$+$}}
\put(1287,527){\makebox(0,0){$+$}}
\put(1300,514){\makebox(0,0){$+$}}
\put(1312,463){\makebox(0,0){$+$}}
\put(1325,451){\makebox(0,0){$+$}}
\put(1338,756){\makebox(0,0){$+$}}
\put(1350,680){\makebox(0,0){$+$}}
\put(1363,459){\makebox(0,0){$+$}}
\put(1376,539){\makebox(0,0){$+$}}
\put(1388,467){\makebox(0,0){$+$}}
\put(1401,409){\makebox(0,0){$+$}}
\put(1414,582){\makebox(0,0){$+$}}
\put(1426,492){\makebox(0,0){$+$}}
\put(171,594){\makebox(0,0){$\times$}}
\put(184,494){\makebox(0,0){$\times$}}
\put(196,614){\makebox(0,0){$\times$}}
\put(209,585){\makebox(0,0){$\times$}}
\put(222,514){\makebox(0,0){$\times$}}
\put(234,578){\makebox(0,0){$\times$}}
\put(247,565){\makebox(0,0){$\times$}}
\put(260,725){\makebox(0,0){$\times$}}
\put(272,607){\makebox(0,0){$\times$}}
\put(285,496){\makebox(0,0){$\times$}}
\put(298,585){\makebox(0,0){$\times$}}
\put(310,594){\makebox(0,0){$\times$}}
\put(323,549){\makebox(0,0){$\times$}}
\put(336,625){\makebox(0,0){$\times$}}
\put(349,533){\makebox(0,0){$\times$}}
\put(361,432){\makebox(0,0){$\times$}}
\put(374,569){\makebox(0,0){$\times$}}
\put(387,574){\makebox(0,0){$\times$}}
\put(399,517){\makebox(0,0){$\times$}}
\put(412,584){\makebox(0,0){$\times$}}
\put(425,537){\makebox(0,0){$\times$}}
\put(437,563){\makebox(0,0){$\times$}}
\put(450,629){\makebox(0,0){$\times$}}
\put(463,609){\makebox(0,0){$\times$}}
\put(475,534){\makebox(0,0){$\times$}}
\put(488,505){\makebox(0,0){$\times$}}
\put(501,661){\makebox(0,0){$\times$}}
\put(513,566){\makebox(0,0){$\times$}}
\put(539,578){\makebox(0,0){$\times$}}
\put(551,435){\makebox(0,0){$\times$}}
\put(564,531){\makebox(0,0){$\times$}}
\put(577,534){\makebox(0,0){$\times$}}
\put(589,482){\makebox(0,0){$\times$}}
\put(602,572){\makebox(0,0){$\times$}}
\put(615,550){\makebox(0,0){$\times$}}
\put(627,565){\makebox(0,0){$\times$}}
\put(640,648){\makebox(0,0){$\times$}}
\put(653,595){\makebox(0,0){$\times$}}
\put(666,523){\makebox(0,0){$\times$}}
\put(678,542){\makebox(0,0){$\times$}}
\put(691,542){\makebox(0,0){$\times$}}
\put(704,520){\makebox(0,0){$\times$}}
\put(716,492){\makebox(0,0){$\times$}}
\put(729,539){\makebox(0,0){$\times$}}
\put(742,577){\makebox(0,0){$\times$}}
\put(754,660){\makebox(0,0){$\times$}}
\put(767,510){\makebox(0,0){$\times$}}
\put(780,507){\makebox(0,0){$\times$}}
\put(792,657){\makebox(0,0){$\times$}}
\put(805,584){\makebox(0,0){$\times$}}
\put(818,593){\makebox(0,0){$\times$}}
\put(830,610){\makebox(0,0){$\times$}}
\put(843,603){\makebox(0,0){$\times$}}
\put(856,508){\makebox(0,0){$\times$}}
\put(868,397){\makebox(0,0){$\times$}}
\put(881,610){\makebox(0,0){$\times$}}
\put(894,534){\makebox(0,0){$\times$}}
\put(906,521){\makebox(0,0){$\times$}}
\put(919,820){\makebox(0,0){$\times$}}
\put(932,662){\makebox(0,0){$\times$}}
\put(944,539){\makebox(0,0){$\times$}}
\put(957,495){\makebox(0,0){$\times$}}
\put(970,668){\makebox(0,0){$\times$}}
\put(983,563){\makebox(0,0){$\times$}}
\put(995,526){\makebox(0,0){$\times$}}
\put(1008,530){\makebox(0,0){$\times$}}
\put(1021,473){\makebox(0,0){$\times$}}
\put(1033,728){\makebox(0,0){$\times$}}
\put(1046,575){\makebox(0,0){$\times$}}
\put(1059,476){\makebox(0,0){$\times$}}
\put(1071,611){\makebox(0,0){$\times$}}
\put(1084,448){\makebox(0,0){$\times$}}
\put(1097,569){\makebox(0,0){$\times$}}
\put(1109,558){\makebox(0,0){$\times$}}
\put(1122,495){\makebox(0,0){$\times$}}
\put(1135,566){\makebox(0,0){$\times$}}
\put(1147,480){\makebox(0,0){$\times$}}
\put(1160,633){\makebox(0,0){$\times$}}
\put(1173,518){\makebox(0,0){$\times$}}
\put(1185,539){\makebox(0,0){$\times$}}
\put(1198,526){\makebox(0,0){$\times$}}
\put(1211,639){\makebox(0,0){$\times$}}
\put(1223,536){\makebox(0,0){$\times$}}
\put(1236,480){\makebox(0,0){$\times$}}
\put(1249,558){\makebox(0,0){$\times$}}
\put(1261,619){\makebox(0,0){$\times$}}
\put(1274,686){\makebox(0,0){$\times$}}
\put(1287,658){\makebox(0,0){$\times$}}
\put(1300,486){\makebox(0,0){$\times$}}
\put(1312,445){\makebox(0,0){$\times$}}
\put(1325,473){\makebox(0,0){$\times$}}
\put(1338,754){\makebox(0,0){$\times$}}
\put(1350,655){\makebox(0,0){$\times$}}
\put(1363,416){\makebox(0,0){$\times$}}
\put(1376,491){\makebox(0,0){$\times$}}
\put(1388,512){\makebox(0,0){$\times$}}
\put(1401,447){\makebox(0,0){$\times$}}
\put(1414,566){\makebox(0,0){$\times$}}
\put(1426,459){\makebox(0,0){$\times$}}
\sbox{\plotpoint}{\rule[-0.400pt]{0.800pt}{0.800pt}}%
\put(171,475){\makebox(0,0){$\ast$}}
\put(184,517){\makebox(0,0){$\ast$}}
\put(196,591){\makebox(0,0){$\ast$}}
\put(209,595){\makebox(0,0){$\ast$}}
\put(222,571){\makebox(0,0){$\ast$}}
\put(234,588){\makebox(0,0){$\ast$}}
\put(247,553){\makebox(0,0){$\ast$}}
\put(260,713){\makebox(0,0){$\ast$}}
\put(272,623){\makebox(0,0){$\ast$}}
\put(285,450){\makebox(0,0){$\ast$}}
\put(298,588){\makebox(0,0){$\ast$}}
\put(310,582){\makebox(0,0){$\ast$}}
\put(323,543){\makebox(0,0){$\ast$}}
\put(336,638){\makebox(0,0){$\ast$}}
\put(349,547){\makebox(0,0){$\ast$}}
\put(361,422){\makebox(0,0){$\ast$}}
\put(374,565){\makebox(0,0){$\ast$}}
\put(387,604){\makebox(0,0){$\ast$}}
\put(399,495){\makebox(0,0){$\ast$}}
\put(412,600){\makebox(0,0){$\ast$}}
\put(425,555){\makebox(0,0){$\ast$}}
\put(437,563){\makebox(0,0){$\ast$}}
\put(450,604){\makebox(0,0){$\ast$}}
\put(463,616){\makebox(0,0){$\ast$}}
\put(475,595){\makebox(0,0){$\ast$}}
\put(488,504){\makebox(0,0){$\ast$}}
\put(501,651){\makebox(0,0){$\ast$}}
\put(513,507){\makebox(0,0){$\ast$}}
\put(539,587){\makebox(0,0){$\ast$}}
\put(551,453){\makebox(0,0){$\ast$}}
\put(564,480){\makebox(0,0){$\ast$}}
\put(577,552){\makebox(0,0){$\ast$}}
\put(589,457){\makebox(0,0){$\ast$}}
\put(602,574){\makebox(0,0){$\ast$}}
\put(615,553){\makebox(0,0){$\ast$}}
\put(627,559){\makebox(0,0){$\ast$}}
\put(640,658){\makebox(0,0){$\ast$}}
\put(653,591){\makebox(0,0){$\ast$}}
\put(666,512){\makebox(0,0){$\ast$}}
\put(678,547){\makebox(0,0){$\ast$}}
\put(691,510){\makebox(0,0){$\ast$}}
\put(704,523){\makebox(0,0){$\ast$}}
\put(716,473){\makebox(0,0){$\ast$}}
\put(729,540){\makebox(0,0){$\ast$}}
\put(742,579){\makebox(0,0){$\ast$}}
\put(754,660){\makebox(0,0){$\ast$}}
\put(767,501){\makebox(0,0){$\ast$}}
\put(780,488){\makebox(0,0){$\ast$}}
\put(792,655){\makebox(0,0){$\ast$}}
\put(805,572){\makebox(0,0){$\ast$}}
\put(818,588){\makebox(0,0){$\ast$}}
\put(830,610){\makebox(0,0){$\ast$}}
\put(843,581){\makebox(0,0){$\ast$}}
\put(856,508){\makebox(0,0){$\ast$}}
\put(868,429){\makebox(0,0){$\ast$}}
\put(881,579){\makebox(0,0){$\ast$}}
\put(894,572){\makebox(0,0){$\ast$}}
\put(906,520){\makebox(0,0){$\ast$}}
\put(919,826){\makebox(0,0){$\ast$}}
\put(932,677){\makebox(0,0){$\ast$}}
\put(944,526){\makebox(0,0){$\ast$}}
\put(957,505){\makebox(0,0){$\ast$}}
\put(970,680){\makebox(0,0){$\ast$}}
\put(983,547){\makebox(0,0){$\ast$}}
\put(995,565){\makebox(0,0){$\ast$}}
\put(1008,526){\makebox(0,0){$\ast$}}
\put(1021,479){\makebox(0,0){$\ast$}}
\put(1033,696){\makebox(0,0){$\ast$}}
\put(1046,523){\makebox(0,0){$\ast$}}
\put(1059,480){\makebox(0,0){$\ast$}}
\put(1071,611){\makebox(0,0){$\ast$}}
\put(1084,533){\makebox(0,0){$\ast$}}
\put(1097,577){\makebox(0,0){$\ast$}}
\put(1109,601){\makebox(0,0){$\ast$}}
\put(1122,489){\makebox(0,0){$\ast$}}
\put(1135,581){\makebox(0,0){$\ast$}}
\put(1147,505){\makebox(0,0){$\ast$}}
\put(1160,645){\makebox(0,0){$\ast$}}
\put(1173,527){\makebox(0,0){$\ast$}}
\put(1185,550){\makebox(0,0){$\ast$}}
\put(1198,523){\makebox(0,0){$\ast$}}
\put(1211,483){\makebox(0,0){$\ast$}}
\put(1223,531){\makebox(0,0){$\ast$}}
\put(1236,498){\makebox(0,0){$\ast$}}
\put(1249,534){\makebox(0,0){$\ast$}}
\put(1261,623){\makebox(0,0){$\ast$}}
\put(1274,696){\makebox(0,0){$\ast$}}
\put(1287,486){\makebox(0,0){$\ast$}}
\put(1300,536){\makebox(0,0){$\ast$}}
\put(1312,448){\makebox(0,0){$\ast$}}
\put(1325,520){\makebox(0,0){$\ast$}}
\put(1338,694){\makebox(0,0){$\ast$}}
\put(1350,677){\makebox(0,0){$\ast$}}
\put(1363,427){\makebox(0,0){$\ast$}}
\put(1376,517){\makebox(0,0){$\ast$}}
\put(1388,489){\makebox(0,0){$\ast$}}
\put(1401,384){\makebox(0,0){$\ast$}}
\put(1414,558){\makebox(0,0){$\ast$}}
\put(1426,467){\makebox(0,0){$\ast$}}
\sbox{\plotpoint}{\rule[-0.500pt]{1.000pt}{1.000pt}}%
\put(171,453){\raisebox{-.8pt}{\makebox(0,0){$\Box$}}}
\put(184,498){\raisebox{-.8pt}{\makebox(0,0){$\Box$}}}
\put(196,619){\raisebox{-.8pt}{\makebox(0,0){$\Box$}}}
\put(209,620){\raisebox{-.8pt}{\makebox(0,0){$\Box$}}}
\put(222,507){\raisebox{-.8pt}{\makebox(0,0){$\Box$}}}
\put(234,581){\raisebox{-.8pt}{\makebox(0,0){$\Box$}}}
\put(247,546){\raisebox{-.8pt}{\makebox(0,0){$\Box$}}}
\put(260,716){\raisebox{-.8pt}{\makebox(0,0){$\Box$}}}
\put(272,614){\raisebox{-.8pt}{\makebox(0,0){$\Box$}}}
\put(285,459){\raisebox{-.8pt}{\makebox(0,0){$\Box$}}}
\put(298,572){\raisebox{-.8pt}{\makebox(0,0){$\Box$}}}
\put(310,581){\raisebox{-.8pt}{\makebox(0,0){$\Box$}}}
\put(323,545){\raisebox{-.8pt}{\makebox(0,0){$\Box$}}}
\put(336,616){\raisebox{-.8pt}{\makebox(0,0){$\Box$}}}
\put(349,491){\raisebox{-.8pt}{\makebox(0,0){$\Box$}}}
\put(361,435){\raisebox{-.8pt}{\makebox(0,0){$\Box$}}}
\put(374,641){\raisebox{-.8pt}{\makebox(0,0){$\Box$}}}
\put(387,646){\raisebox{-.8pt}{\makebox(0,0){$\Box$}}}
\put(399,515){\raisebox{-.8pt}{\makebox(0,0){$\Box$}}}
\put(412,600){\raisebox{-.8pt}{\makebox(0,0){$\Box$}}}
\put(425,547){\raisebox{-.8pt}{\makebox(0,0){$\Box$}}}
\put(437,584){\raisebox{-.8pt}{\makebox(0,0){$\Box$}}}
\put(450,642){\raisebox{-.8pt}{\makebox(0,0){$\Box$}}}
\put(463,607){\raisebox{-.8pt}{\makebox(0,0){$\Box$}}}
\put(475,472){\raisebox{-.8pt}{\makebox(0,0){$\Box$}}}
\put(488,531){\raisebox{-.8pt}{\makebox(0,0){$\Box$}}}
\put(501,665){\raisebox{-.8pt}{\makebox(0,0){$\Box$}}}
\put(513,527){\raisebox{-.8pt}{\makebox(0,0){$\Box$}}}
\put(539,588){\raisebox{-.8pt}{\makebox(0,0){$\Box$}}}
\put(551,488){\raisebox{-.8pt}{\makebox(0,0){$\Box$}}}
\put(564,520){\raisebox{-.8pt}{\makebox(0,0){$\Box$}}}
\put(577,508){\raisebox{-.8pt}{\makebox(0,0){$\Box$}}}
\put(589,502){\raisebox{-.8pt}{\makebox(0,0){$\Box$}}}
\put(602,543){\raisebox{-.8pt}{\makebox(0,0){$\Box$}}}
\put(615,553){\raisebox{-.8pt}{\makebox(0,0){$\Box$}}}
\put(627,549){\raisebox{-.8pt}{\makebox(0,0){$\Box$}}}
\put(640,678){\raisebox{-.8pt}{\makebox(0,0){$\Box$}}}
\put(653,591){\raisebox{-.8pt}{\makebox(0,0){$\Box$}}}
\put(666,524){\raisebox{-.8pt}{\makebox(0,0){$\Box$}}}
\put(678,553){\raisebox{-.8pt}{\makebox(0,0){$\Box$}}}
\put(691,498){\raisebox{-.8pt}{\makebox(0,0){$\Box$}}}
\put(704,543){\raisebox{-.8pt}{\makebox(0,0){$\Box$}}}
\put(716,495){\raisebox{-.8pt}{\makebox(0,0){$\Box$}}}
\put(729,547){\raisebox{-.8pt}{\makebox(0,0){$\Box$}}}
\put(742,547){\raisebox{-.8pt}{\makebox(0,0){$\Box$}}}
\put(754,665){\raisebox{-.8pt}{\makebox(0,0){$\Box$}}}
\put(767,581){\raisebox{-.8pt}{\makebox(0,0){$\Box$}}}
\put(780,492){\raisebox{-.8pt}{\makebox(0,0){$\Box$}}}
\put(792,673){\raisebox{-.8pt}{\makebox(0,0){$\Box$}}}
\put(805,594){\raisebox{-.8pt}{\makebox(0,0){$\Box$}}}
\put(818,574){\raisebox{-.8pt}{\makebox(0,0){$\Box$}}}
\put(830,591){\raisebox{-.8pt}{\makebox(0,0){$\Box$}}}
\put(843,582){\raisebox{-.8pt}{\makebox(0,0){$\Box$}}}
\put(856,512){\raisebox{-.8pt}{\makebox(0,0){$\Box$}}}
\put(868,451){\raisebox{-.8pt}{\makebox(0,0){$\Box$}}}
\put(881,591){\raisebox{-.8pt}{\makebox(0,0){$\Box$}}}
\put(894,639){\raisebox{-.8pt}{\makebox(0,0){$\Box$}}}
\put(906,520){\raisebox{-.8pt}{\makebox(0,0){$\Box$}}}
\put(919,809){\raisebox{-.8pt}{\makebox(0,0){$\Box$}}}
\put(932,593){\raisebox{-.8pt}{\makebox(0,0){$\Box$}}}
\put(944,552){\raisebox{-.8pt}{\makebox(0,0){$\Box$}}}
\put(957,479){\raisebox{-.8pt}{\makebox(0,0){$\Box$}}}
\put(970,633){\raisebox{-.8pt}{\makebox(0,0){$\Box$}}}
\put(983,550){\raisebox{-.8pt}{\makebox(0,0){$\Box$}}}
\put(995,457){\raisebox{-.8pt}{\makebox(0,0){$\Box$}}}
\put(1008,533){\raisebox{-.8pt}{\makebox(0,0){$\Box$}}}
\put(1021,505){\raisebox{-.8pt}{\makebox(0,0){$\Box$}}}
\put(1033,715){\raisebox{-.8pt}{\makebox(0,0){$\Box$}}}
\put(1046,585){\raisebox{-.8pt}{\makebox(0,0){$\Box$}}}
\put(1059,491){\raisebox{-.8pt}{\makebox(0,0){$\Box$}}}
\put(1071,584){\raisebox{-.8pt}{\makebox(0,0){$\Box$}}}
\put(1084,459){\raisebox{-.8pt}{\makebox(0,0){$\Box$}}}
\put(1097,539){\raisebox{-.8pt}{\makebox(0,0){$\Box$}}}
\put(1109,566){\raisebox{-.8pt}{\makebox(0,0){$\Box$}}}
\put(1122,502){\raisebox{-.8pt}{\makebox(0,0){$\Box$}}}
\put(1135,569){\raisebox{-.8pt}{\makebox(0,0){$\Box$}}}
\put(1147,510){\raisebox{-.8pt}{\makebox(0,0){$\Box$}}}
\put(1160,627){\raisebox{-.8pt}{\makebox(0,0){$\Box$}}}
\put(1173,536){\raisebox{-.8pt}{\makebox(0,0){$\Box$}}}
\put(1185,559){\raisebox{-.8pt}{\makebox(0,0){$\Box$}}}
\put(1198,501){\raisebox{-.8pt}{\makebox(0,0){$\Box$}}}
\put(1211,460){\raisebox{-.8pt}{\makebox(0,0){$\Box$}}}
\put(1223,528){\raisebox{-.8pt}{\makebox(0,0){$\Box$}}}
\put(1236,491){\raisebox{-.8pt}{\makebox(0,0){$\Box$}}}
\put(1249,543){\raisebox{-.8pt}{\makebox(0,0){$\Box$}}}
\put(1261,606){\raisebox{-.8pt}{\makebox(0,0){$\Box$}}}
\put(1274,687){\raisebox{-.8pt}{\makebox(0,0){$\Box$}}}
\put(1287,504){\raisebox{-.8pt}{\makebox(0,0){$\Box$}}}
\put(1300,511){\raisebox{-.8pt}{\makebox(0,0){$\Box$}}}
\put(1312,453){\raisebox{-.8pt}{\makebox(0,0){$\Box$}}}
\put(1325,470){\raisebox{-.8pt}{\makebox(0,0){$\Box$}}}
\put(1338,759){\raisebox{-.8pt}{\makebox(0,0){$\Box$}}}
\put(1350,684){\raisebox{-.8pt}{\makebox(0,0){$\Box$}}}
\put(1363,489){\raisebox{-.8pt}{\makebox(0,0){$\Box$}}}
\put(1376,523){\raisebox{-.8pt}{\makebox(0,0){$\Box$}}}
\put(1388,486){\raisebox{-.8pt}{\makebox(0,0){$\Box$}}}
\put(1401,399){\raisebox{-.8pt}{\makebox(0,0){$\Box$}}}
\put(1414,609){\raisebox{-.8pt}{\makebox(0,0){$\Box$}}}
\put(1426,478){\raisebox{-.8pt}{\makebox(0,0){$\Box$}}}
\sbox{\plotpoint}{\rule[-0.600pt]{1.200pt}{1.200pt}}%
\put(171,470){\makebox(0,0){$\blacksquare$}}
\put(184,514){\makebox(0,0){$\blacksquare$}}
\put(196,597){\makebox(0,0){$\blacksquare$}}
\put(209,591){\makebox(0,0){$\blacksquare$}}
\put(222,507){\makebox(0,0){$\blacksquare$}}
\put(234,555){\makebox(0,0){$\blacksquare$}}
\put(247,582){\makebox(0,0){$\blacksquare$}}
\put(260,699){\makebox(0,0){$\blacksquare$}}
\put(272,595){\makebox(0,0){$\blacksquare$}}
\put(285,453){\makebox(0,0){$\blacksquare$}}
\put(298,601){\makebox(0,0){$\blacksquare$}}
\put(310,593){\makebox(0,0){$\blacksquare$}}
\put(323,574){\makebox(0,0){$\blacksquare$}}
\put(336,609){\makebox(0,0){$\blacksquare$}}
\put(349,537){\makebox(0,0){$\blacksquare$}}
\put(361,416){\makebox(0,0){$\blacksquare$}}
\put(374,553){\makebox(0,0){$\blacksquare$}}
\put(387,559){\makebox(0,0){$\blacksquare$}}
\put(399,499){\makebox(0,0){$\blacksquare$}}
\put(412,600){\makebox(0,0){$\blacksquare$}}
\put(425,523){\makebox(0,0){$\blacksquare$}}
\put(437,575){\makebox(0,0){$\blacksquare$}}
\put(450,641){\makebox(0,0){$\blacksquare$}}
\put(463,638){\makebox(0,0){$\blacksquare$}}
\put(475,511){\makebox(0,0){$\blacksquare$}}
\put(488,527){\makebox(0,0){$\blacksquare$}}
\put(501,668){\makebox(0,0){$\blacksquare$}}
\put(513,479){\makebox(0,0){$\blacksquare$}}
\put(539,593){\makebox(0,0){$\blacksquare$}}
\put(551,437){\makebox(0,0){$\blacksquare$}}
\put(564,536){\makebox(0,0){$\blacksquare$}}
\put(577,502){\makebox(0,0){$\blacksquare$}}
\put(589,518){\makebox(0,0){$\blacksquare$}}
\put(602,581){\makebox(0,0){$\blacksquare$}}
\put(615,566){\makebox(0,0){$\blacksquare$}}
\put(627,578){\makebox(0,0){$\blacksquare$}}
\put(640,655){\makebox(0,0){$\blacksquare$}}
\put(653,595){\makebox(0,0){$\blacksquare$}}
\put(666,520){\makebox(0,0){$\blacksquare$}}
\put(678,568){\makebox(0,0){$\blacksquare$}}
\put(691,495){\makebox(0,0){$\blacksquare$}}
\put(704,547){\makebox(0,0){$\blacksquare$}}
\put(716,496){\makebox(0,0){$\blacksquare$}}
\put(729,518){\makebox(0,0){$\blacksquare$}}
\put(742,540){\makebox(0,0){$\blacksquare$}}
\put(754,664){\makebox(0,0){$\blacksquare$}}
\put(767,572){\makebox(0,0){$\blacksquare$}}
\put(780,501){\makebox(0,0){$\blacksquare$}}
\put(792,662){\makebox(0,0){$\blacksquare$}}
\put(805,584){\makebox(0,0){$\blacksquare$}}
\put(818,595){\makebox(0,0){$\blacksquare$}}
\put(830,603){\makebox(0,0){$\blacksquare$}}
\put(843,611){\makebox(0,0){$\blacksquare$}}
\put(856,531){\makebox(0,0){$\blacksquare$}}
\put(868,572){\makebox(0,0){$\blacksquare$}}
\put(881,566){\makebox(0,0){$\blacksquare$}}
\put(894,579){\makebox(0,0){$\blacksquare$}}
\put(906,536){\makebox(0,0){$\blacksquare$}}
\put(932,671){\makebox(0,0){$\blacksquare$}}
\put(944,523){\makebox(0,0){$\blacksquare$}}
\put(957,492){\makebox(0,0){$\blacksquare$}}
\put(970,681){\makebox(0,0){$\blacksquare$}}
\put(983,563){\makebox(0,0){$\blacksquare$}}
\put(995,559){\makebox(0,0){$\blacksquare$}}
\put(1008,518){\makebox(0,0){$\blacksquare$}}
\put(1021,521){\makebox(0,0){$\blacksquare$}}
\put(1033,689){\makebox(0,0){$\blacksquare$}}
\put(1046,539){\makebox(0,0){$\blacksquare$}}
\put(1059,502){\makebox(0,0){$\blacksquare$}}
\put(1071,638){\makebox(0,0){$\blacksquare$}}
\put(1084,563){\makebox(0,0){$\blacksquare$}}
\put(1097,600){\makebox(0,0){$\blacksquare$}}
\put(1109,569){\makebox(0,0){$\blacksquare$}}
\put(1122,518){\makebox(0,0){$\blacksquare$}}
\put(1135,581){\makebox(0,0){$\blacksquare$}}
\put(1147,540){\makebox(0,0){$\blacksquare$}}
\put(1160,697){\makebox(0,0){$\blacksquare$}}
\put(1173,523){\makebox(0,0){$\blacksquare$}}
\put(1185,555){\makebox(0,0){$\blacksquare$}}
\put(1198,496){\makebox(0,0){$\blacksquare$}}
\put(1211,526){\makebox(0,0){$\blacksquare$}}
\put(1223,527){\makebox(0,0){$\blacksquare$}}
\put(1236,507){\makebox(0,0){$\blacksquare$}}
\put(1249,523){\makebox(0,0){$\blacksquare$}}
\put(1261,533){\makebox(0,0){$\blacksquare$}}
\put(1274,700){\makebox(0,0){$\blacksquare$}}
\put(1287,543){\makebox(0,0){$\blacksquare$}}
\put(1300,512){\makebox(0,0){$\blacksquare$}}
\put(1312,443){\makebox(0,0){$\blacksquare$}}
\put(1325,464){\makebox(0,0){$\blacksquare$}}
\put(1338,756){\makebox(0,0){$\blacksquare$}}
\put(1350,674){\makebox(0,0){$\blacksquare$}}
\put(1363,444){\makebox(0,0){$\blacksquare$}}
\put(1376,523){\makebox(0,0){$\blacksquare$}}
\put(1388,489){\makebox(0,0){$\blacksquare$}}
\put(1401,413){\makebox(0,0){$\blacksquare$}}
\put(1414,598){\makebox(0,0){$\blacksquare$}}
\put(1426,456){\makebox(0,0){$\blacksquare$}}
\sbox{\plotpoint}{\rule[-0.200pt]{0.400pt}{0.400pt}}%
\put(171.0,131.0){\rule[-0.200pt]{0.400pt}{175.375pt}}
\put(171.0,131.0){\rule[-0.200pt]{305.461pt}{0.400pt}}
\end{picture}

    \caption{A failed recovery. The Levenshtein distance between the training samples and
        a recording of the victim visiting the Nicki Minaj Wikipedia page. The
        shortest distance (97 on the page axis) corresponds to a training sample
        of the Eminem Wikipedia page. The Nicki Minaj training samples still
        stand out (55 on the page axis). The different shapes in a column
        represent the five training samples of that page. The order on the page
        axis is not meaningful.}
    \label{figure:minaj}
\end{figure*}

\subsection{Failures}
\label{sec:failures}

Here we disclose the programs we looked at without finding any conclusive
attacks. This section can be taken as evidence that we did not need to look
``too hard'' to find attacks, but it also shows that attacks are not completely
trivial to find.

Early on in our investigations we tried extract content from the Evince PDF
viewer. We got to the point where \textsc{Flush+Reload} appeared (by visual
inspection) to be partially recovering the sequence of PDF commands in
a hand-crafted PDF file. Due to lack of time, we did not perform a quantitative
analysis. 

While looking for an attack against a web browser, we spent some time with
Firefox, Lynx, and ELinks. When we found the Links attack we stopped looking at
the other browsers. We only spent a short amount of time (less than a day)
looking at the code of the other browsers, and we never ran \textsc{Flush+Reload} against
them.

The fact that we did not find an attack does not mean that none exist. We were
working under extremely limited time constraints and chose to go with the first
attacks we found. We conjecture that similar attacks will work against the other
browsers.

We tried the Links attack on an Amazon EC2 instance and it did not work. Looking
at the attack tool's output, it appears to be because the VM is running out of
CPU credit which is making it miss too many measurements. We did not have time
to investigate further.

\section{Related Work}
\label{sec:relwork}

Our attacks are based on the \textsc{Flush+Reload} attack \cite{yarom2013flush}.
The authors provided us with their implementation of the attack, which we
improved by turning it into a command-line utility and making it accessible to
Ruby scripts.

The \textsc{Flush+Reload} attack is based on work by Bangerter et al.\ where it was used
to break an implementation of AES \cite{gullasch2011cache}. It has since been
applied to GnuPG \cite{yarom2013flush} and OpenSSL \cite{benger2014ooh,
yarom2014recovering}. Our work applies it in a non-cryptographic setting.

There is a vast body of literature on attacking cryptographic software and
hardware with side channels. There are far too many examples to list here; we
refer to the interesting cases of extracting RSA keys via power analysis
\cite{messerges1999power} and sound recording \cite{genkin2013rsa} as well as
the \textsc{Flush+Reload} attacks \cite{yarom2013flush, benger2014ooh,
yarom2014recovering}.

Our work shows that cache side channel attacks have privacy implications beyond
breaking cryptography software. We believe we are the first to apply a cache
side channel against non-cryptographic software to compromise privacy. Cache
side channels have already been applied once in a non-cryptographic context: to
break kernel address space layout randomization \cite{hund2013practical}.

There is a growing body of work showing that other kinds of side channels are
successfully breaking privacy. Here we highlight some of that work.

\begin{itemize}
    \item[--] Some web applications leak information about the user's input through
          their behavior in communicating with the web server (even over
          encrypted connections) \cite{bortz2007exposing, chen2010side}.
    \item[--] Malicious web pages can extract browser history through a side-channel attack
          \cite{weinberg2011still}. 
    \item[--] Timing variations in databases make it possible to extract
          indexed records \cite{futoransky2007nd2db}.
    \item[--] Variable bit rate encoding can leak words spoken over encrypted
        VoIP links \cite{white2011phonotactic}.
    \item[--] In an Android app, the UI state can be inferred through side channels
          in the GUI framework \cite{chen2014peeking}.
    \item[--] On Linux, keystroke timings can be learned by watching the size of
          a victim process's stack \cite{zhang2009peeping} and by observing an
          SSH connection \cite{song2001timing}. Knowing the timing of keystrokes
          reveals information about what is being typed.
    \item[--] Malicious web pages can use ``red pill'' side channels to find out if
          they are running in a virtual machine (i.e.\ for malware analysis)
          \cite{ho2014tick}.
    \item[--] An attacker can learn a victim's Internet traffic volume, as well as
          individual packet times by exploiting a side channel in router
          scheduling algorithms. \cite{kadloor2010low}.
    \item[--] A smartphone app can infer what the user is typing from accelerometer
          and gyroscope measurements \cite{owusu2012accessory,
          cai2012practicality}.
    \item[--] Profiling a server's power use can reveal which virtual machines
        are running on it \cite{hlavacs2011energy}.
    \item[--] Text can be recovered from the sounds dot-matrix printers make
          \cite{backes2010acoustic}.
\end{itemize}

The list above shows that the days of side channels only being a problem for
cryptographic software are long gone. 

The attacks we presented show that at least one generic cache side channel
attack can be used to compromise the user's privacy. We believe these attacks
represent baby steps towards understanding the full implication of the
\textsc{Flush+Reload} attack, and we hope they will motivate more research on
the topic.

\section{Future Work}
\label{sec:future}

We found the probe locations for our attacks manually. This involved reading the
source code to get an understanding of how the program works, then making
educated guesses about the best probe locations. This can easily amount to a few
days of work per attack. Therefore, our attacks were restricted to simple parts
of simple programs, i.e. we attacked Links instead of a modern web browser like
Firefox.

In the future, we plan to automate as much of the manual work as possible. We
hypothesize that input distinguishing attacks can be found automatically,
without needing access to the source code. Given memory access traces of
a program running on different inputs, it should be possible to automatically
infer which probe locations are the most useful for distinguishing the inputs
(e.g. by taking the ones that maximize the Levenshtein distance).

If we can successfully automate attack-finding, we plan on attacking programs
with larger user bases and developing a Metasploit-like exploitation framework.

Quantitatively, our attacks only need a small amount of information from the
target programs. For TrueCrypt, we are only extracting one bit of information.
The Links attack only needs to extract a number of bits logarithmic in the
number of pages we want to distinguish between (6.6 bits for 100 pages). We
should find out how much information \textsc{Flush+Reload} can extract from
non-cryptographic programs. In particular, can \textsc{Flush+Reload} extract
previously-unknown user input from a program in any plausible scenario?

Knowing that \textsc{Flush+Reload} has privacy implications to regular users and
not just users of cryptography, there will be a higher demand to know if one's
system is vulnerable. A database of vulnerable processors should be created, and
an easy-to-use test tool should be developed, so that users can determine if
their processors are vulnerable. With this knowledge, consumers can make better
purchasing decisions.

The \textsc{Flush+Reload} attack does not work on AMD processors. It has been
speculated that this is because ``AMD caches are non-inclusive, i.e. data in L1
does not need to also be in L2 or L3, as is the case with the Intel caches''
\cite{yarom2013flush}. The question of why \textsc{Flush+Reload} does not work
on AMD should be answered conclusively with empirical data. Without a complete
understanding, we cannot say for sure that AMD processors are not vulnerable. 

Some virtual machine hypervisors deduplicate pages between virtual machines. If
two physical pages have identical content, the hypervisor will change the page
tables so that both virtual machines access the same physical page. Two isolated
virtual machines on the same hardware can communicate by checking whether their
pages got deduplicated. The resulting covert channel can move 80 bits per
second, but with latencies of up to five minutes \cite{xiao2013security}. Using
\textsc{Flush+Reload} to communicate through the deduplicated pages could yield
similar data rates with less latency.

When memory deduplication is accomplished by matching page contents, an attacker
can force a deduplication to happen on any page whose contents they know.
Therefore, if an attacker knows or can predict some data inside another process,
they can spy on it with \textsc{Flush+Reload}. This might reveal more
information than spying on the program code alone.

\section{Conclusion}
\label{sec:conclusion}

Classically, side channel attacks are used to extract encryption keys from
software and hardware implementations of cryptography. More recently, side
channel attacks are being used to compromise privacy in more general settings.
We presented two attacks that continue this trend. 

Our attacks let an attacker (1) determine whether a TrueCrypt volume contains
a hidden volume and (2) determine which of the top 100 Wikipedia pages a victim
visited with the Links web browser.

These attacks are not very damaging on their own, but they are part of a growing
body of work that applies side channels in a much broader setting than breaking
cryptography. We have proposed future work to evaluate the automated discovery
of more powerful attacks based on \textsc{Flush+Reload}.

In addition to presenting a novel use of the \textsc{Flush+Reload} side channel,
we hope to have set an example by releasing our attack implementations,
experiment implementations, and all of our experiment data at the time of
publication.

\bibliographystyle{abbrv}
\bibliography{proposal}

\begin{appendices}
    \section{Reproducing Our Work}
    \label{sec:reproducing}

    The attack tools, experiment implementations, and experiment data are all
    available for download on the author's website.

    \texttt{https://defuse.ca/compromising-privacy-flush-reload.htm}

%    For longevity, a second copy of everything is archived in the Internet
%    Archive\footnote{Actually, it is not on the Internet Archive yet}.
%
%    \texttt{https://archive.org/\textbf{URL-TO-BE-DETERMINED}}
%
    The Links experiment should work without modification on vulnerable
    hardware. Give it a try!

    Throughout this paper we have referred to experiment runs by the name of the
    experiment followed by a four-digit run number. The experiment name
    corresponds to a directory name in the archive, and the run number
    corresponds to a subdirectory of that directory. For example, the
    truecrypt/0003 data can be found in
    \texttt{experiments/truecrypt/runs/0003}.
\end{appendices}

\end{document}

