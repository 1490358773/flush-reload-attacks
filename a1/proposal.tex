\documentclass{acm_proc_article-sp}

\usepackage{url}

\begin{document}

\title{CPSC 502.04 Project Proposal}
\subtitle{Side Channel Attacks Against Non-Cryptographic Applications}

\numberofauthors{1}

\author{
\alignauthor
Taylor Hornby
}

\maketitle

\section{Introduction}

This is a proposal for an undergraduate research project on applying
side-channel attacks to non-cryptographic applications. Section 2, describes
what side-channel attacks are and why we care about them. In section 3, we
present the research questions that will be the focus of this project. Section
4 explains how the research will proceed and what the results should look like.
Section 5 concludes.

% TODO: Use actual references for the sections.

\section{Side Channel Attacks}

Computers are physical realizations of abstract machines. Therefore, when we
program computers, it is easiest to think abstractly. Systems are organized into
layers of abstraction with the highest-level ideas and operations at the top,
the bits and bytes further down, and the physical implementation \textendash
wires and voltages on them \textendash still further down. Complexity is hidden
behind interfaces and APIs. We can use the lower layers without having to
understand all of the details about how they work. Likewise, we can write
libraries without having to consider and understand all of the possible uses.
Thinking about computers this way makes them manageable. It is impossible to
remember all of the details. If the details can be divided into parts with clear
boundaries, we can remember how each part is to be used, without knowing how
that part works.

Because we think about computers as abstract machines, we are trained to think
in terms of explicit communication paths. The explicit communication channels
are the ones that matter: Reading input and writing output. Implicit information
flow gets much less attention. The user does not care exactly how much power
their computer is using, how much noise it makes, or how long it takes to
perform an operation, so long as the values stay within reasonable bounds.
Variations like these are called \emph{side channels}, and if we look closely,
they tell us a little bit about what is going on inside the computer.

Side channels sometimes leak information that the programmer never intended to
leave the program. For example, research \cite{messerges1999power} has shown
that by taking a close look at the amount of power used by smart card, we can
steal the closely-guarded RSA secret keys inside them. This is possible because the
device performs a longer operation when the bit is 1 and a shorter operation
when the bit is 0. So by looking at the amount of power it uses over time, it is
possible to tell which bits are 0 and which are 1. Attacks like these, which use
side channels to learn information about a system are called \emph{side channel
attacks}. Many more examples are known:

\begin{itemize}
\item Genkin et al. \cite{genkin2013rsa} showed that by listening to the sounds
made by a computer performing RSA decryptions, it is possible to extract the
secret key.

\item SSH sends one packet per keystroke, immediately after the key is pressed.
Song et al. \cite{song2001timing} showed that some information about what the
user is typing can be inferred from the timing of the packets.

\item There is a lot of research \cite{bernstein2005cache, osvik2006cache,
weiss2012cache, aciiccmez2006cache} showing that it is very hard to write
a side-channel-free implementation of the AES cipher.

\item Timing side channels were shown to work even across a local network by
Brumley and Boneh \cite{brumley2005remote}.

\item More famously, the BREACH attack on SSL \cite{gluck2013breach} uses the
compression ratio as a side channel to learn secret tokens embedded in an
encrypted web page.
\end{itemize}

Side channels are a challenge because a computer program can be logically
correct and still be vulnerable to side channel attacks. The program can respond
to all inputs in all the right ways, always getting the right answer, but still
it leaks its secrets out through the time it takes to run or the amount of power
it uses.

The list above shows how diverse and widespread side channel attacks are.
Despite their prevalence, we still don't understand them very well. The next
section presents some open questions that this project will try to answer.

\section{Research Problem}

In the last section, we saw that side channel attacks are everywhere, and that
they can be devastating for a system's security. In order to build defenses
against channel attacks, we need to understand them. Our ignorance about side
channel attacks boils down to two central questions: Under what conditions are
side-channel attacks possible? and When a side-channel attack is found, how
reliably can it be exploited?

Many examples of side-channel attacks have been presented in the literature.
Most research is focused on developing new attack vectors. Acoustic
cryptanalysis [CITE] and extracting keys from ground potential fluctuations
[CITE, CHECK] are recent examples. The new styles of attack are almost
universally applied to cryptography first. This makes sense, because leaking
a small number of bits (the key) from a cryptgoraphic application completely
breaks the security. Compounding the issue, cryptography makes heavy, repeated,
use of the few bits that it does need to keep secret, making them more likely to
influence the side channels.

It is unclear if this focus on cryptography because non-cryptographic programs
are hard to attack, or if cryptography is just the low-hanging fruit that gets
attacked first. This is an important question to resolve, since if
non-cryptographic programs \emph{are} vulnerable to side-channels, we might be
overlooking serious vulnerabilities in the programs we use every day. An attack
against a regular program can be just as devastating as an attack against an
implementation of cryptography. For example, if it is possible for one user to
spy on another user as they compose an email, that is just as serious an attack
as if the first user could steal the second user's PGP private key.

This project will try to answer that question. In particular, we ask:

\begin{enumerate}
\item What impact do side-channel attacks have on non-cryptographic systems?

\item Under what conditions are side channels a threat? Under what conditions
      are they not?

\item Does every developer have to be aware of side channels and design their
      code to resist attacks? Or is it enough to care about side channels only
      for sensitive applications like cryptography?

\item Are there automated technological defenses against side-channel attacks?
      Are there easy-to-follow coding guidelines that reduce the impact of side
      channels?

\item Can side channels be used for good? Could we, perhaps, use side channels
      monitoring as a mechanism for spotting attacks?
\end{enumerate}

If this project finds that non-crypto attacks represent real threats, it will
motivate the research community to apply their attacks and defenses beyond
cryptography. If the result find that side-channels cannot be applied to regular
programs, we can learn \emph{why not}, and begin to apply that knowledge to
build defenses.

The next section discusses the methods we will use to answer these questions,
and what the results of the project should look like.

\section{Steps Forward}

% HERE, NOT above, note FLUSH+RELOAD as being particularly applicable to non-crypto attacks.
% 
% 
% Attack Non-crypto apps with FLUSH+RELOAD.
% Strong experimental / reproducible  (scientific) basis.
% Learn from the attacks to see if there are effective defenses.

To find out whether side channel attacks are a significant threat to
non-cryptographic programs, we will start by trying to apply known side-channel
attacks to these programs.

A recently discovered generic side-channel attack called \textsc{Flush+Reload}
\cite{yarom2013flush} is particularly applicable. Using \textsc{Flush+Reload},
a spy process can monitor which code a victim process runs, as long as it can
map the victim process' binary into its address space. The attack works by
timing memory access to determine whether the code is present in the x86
architecture's L3 cache.

The authors of the \textsc{Flush+Reload} attack originally applied it to extract
the secret RSA key from GnuPG. I have speculated \cite{hornby2013flush} that the
attack would also be devastating against non-cryptographic applications, since
it is so general.

TODO: Address project failure here too (i.e. what happens if no attacks are
found)?

\section{Conclusion}

Side channel attacks exploit the boundary between the abstract notion of
a computer program and its real physical implementation. They use physical
measurements of time, power, vibrations, and radiation to learn what a computer
is doing. So far, most research has focused on applying these attacks against
cryptographic systems. The project proposed here is to explore the impact these
attacks have on everyday programs like web browsers, databases, and text
editors. Questions will be answered by trying to apply previously-known side
channel techniques to these programs. The outcome will be a greater
understanding of the impact side-channel attacks have in that context.

Throughout the project, there will be a focus on experimentation and
reproducibility of results.

\bibliographystyle{abbrv}
\bibliography{proposal}

\end{document}

