\documentclass{acm_proc_article-sp}

\begin{document}

\title{CPSC 502.04 Project Proposal}
\subtitle{Side Channel Attacks Against Non-Cryptographic Applications}

\numberofauthors{1}

\author{
\alignauthor
Taylor Hornby
}

\maketitle

\section{Introduction}

This is a proposal for a research project on applying side-channel attacks to
non-cryptographic applications. Section 2, describes what side-channel attacks
are. In section 3, we present the research questions that this project will
attempt to answer. Section 4 shows how the questions will be attacked. Section
5 concludes.

% TODO: Use actual references for the sections.

\section{Side Channel Attacks}

Computers are physical realizations of abstract machines. Indeed, the origins of
computation are rooted in purely mathematical designs like the Turing Machine.
Even von Neumann's architecture, upon which our modern systems are based, is an
abstract system (albeit one that lends itself easily to a real implementation).

When we program computers, it is easiest to think abstractly. We hide complexity
behind interfaces. Systems are organized into layers of abstraction with the
highest-level ideas and operations at the top, the bits and bytes further down,
and the physical implementation \textendash wires and voltages on them
\textendash still further down. The lower layers are available for us to use,
but we do not need to understand their internal function to get our work done.
Likewise, designers of the lower layers do need to know in advance all of the
things we will use their layer for.

Thinking about computers this way makes them manageable. It is impossible to
remember all of the details, but if the details can be summarized succinctly,
divided into layers and regions with boundaries, we can remember the summaries
and get on with our work without worrying about the details.

Seeing computers this way also causes problems. In particular, security
vulnerabilities arise when a user at one level misuses the functionality
provided by another level.

Seeing computers abstractly, we spend most of our effort thinking about explicit
communication paths, for they are the ones that matter. The user does not care
exactly how much power their computer is using, how much noise it makes, or how
long it takes to perform an operation, as long as it stays within reasonable
bounds. These variations are called side channels, and they tell us a little bit
about what is going on inside the computer.

Indeed, side channels sometimes leak sensitive information. Research
\cite{messerges1999power} has shown that by taking a close look at the power
usage signature of a smart card, we can tell which bits of its secret key are
1 and which are 0 (because processing the 1 bits uses more power than the
0 bits), thus extracting the secret key. Attacks like these, which use side
channels to gather information about a system are called side channel attacks.
Many more interesting examples are known:

\begin{itemize}
\item Genkin et al. \cite{genkin2013rsa} showed that by listening to the sounds
made by a computer performing RSA decryptions, it is possible to extract the
secret key.

\item SSH sends one packet per keystroke, immediately after the key is pressed.
Song et al. \cite{song2001timing} showed that some information about what the
user is typing can be inferred from the timing of the packets.

\item There is a lot of research \cite{bernstein2005cache, osvik2006cache,
weiss2012cache, aciiccmez2006cache} showing that it is very hard to write
a side-channel-free implementation of the AES cipher.

\item Timing side channels were shown to work even across a local network by
Brumley and Boneh \cite{brumley2005remote}.

\item More famously, the BREACH attack on SSL \cite{gluck2013breach} uses the
compression ratio as a side channel to learn secret tokens embedded in an
encrypted web page.

\end{itemize}

Side channels are a challenge because a computer program can be logically
correct yet still vulnerable to a side channel attack. The program can respond
to all inputs in all the right ways, always getting the right answer, but still
be leaking its secrets out through the time it takes to run or the amount of
power it uses.

Side channel attacks are not well understood. Indeed, the list above shows how
diverse and prevalent side channel attacks are. The next section presents the
open questions about side channels that this project will try to answer.

\section{Research Problem}

While many examples of side channels have been produced, they are still poorly
understood. This is mainly because we do not yet know everything that is
possible. To get a better understanding, we need to see more attacks, and in
particular, more \emph{general} attacks. It is hard to defend against
side-channel attacks without understanding where they can be found and how
reliable they are. If we develop defenses prematurely, they will not cover all
types of attacks.

Most side channels are applied in the context of cryptography. Cryptography is
especially easy to attack with side channels, because small information leaks,
like leaking a secret key, can be devastating to the entire system. Compounding
the issue, crypto usually makes heavy use of the few secret bits that it has,
making them more exposed to side channel attacks.

TODO: Cite counterexamples and respond to that.

We ask whether this focus on cryptography is because non-cryptographic things
are hard to attack, or if cryptography is just the low hanging fruit that gets
all of the attention. This is an important question, since if we have been
putting undue focus on cryptography, we may be overlooking significant and
exploitable flaws in the programs we rely on every day [EXAMPLE/ILLUSTRATE]
(WORDS: Dormant, undiscovered).

TODO: something about defense

TODO: motivation that side channels are hard to fix, because they are in the
hardware (so even if they are sev:lo they can be so expensive it makes it
worthwhile)

In particular, we ask the following questions:

\begin{enumerate}
\item What is the impact of side channel attacks on non-cryptographic
      applications?

\item Do all developers, not just the ones working on cryptoraphy, need to be
      educated about side channels and design their code to resist them, or is
      it ok to only consider side channels in security-critical code?

\item If the above two questions are answered in the affirmative, what can we do
      to reduce the risk? Are there automated technological defenses? Are there
      easy-to-follow coding guidelines that, when followed, make side channels
      less of a risk?

\item Given the broadened scope, can side channels be applied for good instead
      of bad? Can we use side channels as a means of identifying attacks?
\end{enumerate}

If this research finds that such attacks exist and are widespread, it will
motivate the community to broaden their horizons beyond crypto.

The next section discusses the methods we can use to answer these questions, and
what the result of the project will be if it is successful.

\section{Steps Forward}

% HERE, NOT above, note FLUSH+RELOAD as being particularly applicable to non-crypto attacks.
% 
% 
% Attack Non-crypto apps with FLUSH+RELOAD.
% Strong experimental / reproducible  (scientific) basis.
% Learn from the attacks to see if there are effective defenses.

The essence of this research problem is to determine whether side channels are
realistic threats against non-cryptographic applications. The best way to answer
this is to try to attack them.

TODO: Address project failure here too (i.e. what happens if no attacks are
found)?

\section{Conclusion}

Side channel attacks exploit the boundary between the abstract notion of
a computer program and its real physical implementation. They use physical
measurements of time, power, vibrations, and radiation to learn what a computer
is doing. So far, most research has focused on applying these attacks against
cryptographic systems. The project proposed here is to explore the impact these
attacks have on everyday programs like web browsers, databases, and text
editors. Questions will be answered by trying to apply previously-known side
channel techniques to these programs. The outcome will be a greater
understanding of the impact side-channel attacks have in that context.

Throughout the project, there will be a focus on experimentation and
reproducibility of results.

\bibliographystyle{abbrv}
\bibliography{proposal}

\end{document}

