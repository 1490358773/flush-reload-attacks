\documentclass{acm_proc_article-sp}

\usepackage{url}
\usepackage{microtype}

\begin{document}

% Without this, stuff overfills into the margin. TODO: Eventually, rewrite those
% paragraphs so that they don't.
\sloppy

\title{Side Channel Attacks Against Non-Cryptographic Applications\titlenote{Submitted September 23, 2014}}
\subtitle{CPSC 502.04 Project Proposal}

\numberofauthors{1}

\author{
\alignauthor
Taylor Hornby (Supervised by Dr. John Aycock)\titlenote{John Aycock reviewed and
edited a draft of this document before it was submitted.}\\
\email{th@defuse.ca}
}


\maketitle

\begin{abstract}
This is a proposal for an undergraduate research project. The project is to
apply side-channel attacks to non-cryptographic applications. Side channel
attacks are almost always applied to cryptography-related software and hardware,
so we ask whether significant side-channel vulnerabilities exist in
non-cryptographic systems. The question will be approached by attacking
non-cryptographic software with previously-known side-channel attack techniques.
This should either uncover serious vulnerabilities that have been overlooked, or
if not, at least give further insight into where and when side channel
information leakage is a threat.
\end{abstract}

\section{Introduction}
\label{sec:attacks}

Computers are physical realizations of abstract machines. Therefore, when we
program computers, it is easiest to think abstractly. Systems are organized into
layers of abstraction with the highest-level ideas and operations at the top,
the bits and bytes further down, and the physical implementation \textendash\
wires and the voltages on them \textendash\ still further down.

Complexity is hidden behind interfaces and APIs. We can use lower layers without
having to understand their inner workings. Likewise, we can write libraries
without having to consider and understand all of the possible uses.

Thinking about computers this way makes them manageable. Without abstraction, it
is impossible to remember all of the details. When the details are divided into
parts with clear boundaries, we can remember how each part is \emph{to be used},
without having to know how each part \emph{works}.

The divisions are not always perfect, and this often leads to security problems.
APIs are often misused and assumptions are often made about the internal
workings of a different part. An interesting boundary is the one between the
computer as an abstract machine, and the computer as a physical machine.

When we think about computers as abstract machines, it is natural to think in
terms of explicit communication paths. After all, the explicit communication
channels are the ones that matter: reading input and writing output. The user
does not care exactly how much power their computer is using, how much noise it
makes, or how long it takes to perform an operation, so long as the behavior
stays within reasonable bounds. Thinking physically, things are different. These
values all represent information about what is going on inside the computer.
Measurable variations of these values are called \emph{side channels}
\cite{kelsey1998side}.

Side channels sometimes leak information that is supposed to be kept secret. For
example, research has shown that by taking a close look at the amount of power
used by smart cards, we can steal the closely-guarded RSA secret keys inside
them \cite{messerges1999power}. This is possible because the device performs
a longer operation for key bits that are 1 and a shorter operation for key bits
that are is 0. So by looking at the amount of power it uses over time, it is
possible to tell which bits of the key are 0 and which are 1. Attacks like
these, which use side channels to learn secret information about a system are
called \emph{side channel attacks}. Many examples are known:

\begin{itemize}
\item Genkin et al. \cite{genkin2013rsa} showed that by listening to the sounds
made by a computer performing RSA decryptions, it was possible to extract the
secret key.

\item SSH sends one packet per keystroke, immediately after the key is pressed.
Song et al. \cite{song2001timing} showed that information about users' passwords
can be inferred from the timing of the packets.

\item There is a lot of research \cite{aciiccmez2006cache, bernstein2005cache,
    osvik2006cache, weiss2012cache} showing that it is very hard to write
    a side-channel-free implementation of the industry-standard AES cipher.

\item Timing side channels were shown to work even remotely across a local
network by Brumley and Boneh \cite{brumley2005remote}.

\item The BREACH attack on SSL \cite{gluck2013breach} uses variations in the
    compressibility of data as a side channel to learn secret tokens embedded in
    an encrypted web page.
\end{itemize}

Side channels present a challenge because a computer program can be logically
correct and still be vulnerable to side channel attacks. A program can respond
to all inputs in all the right ways, always getting the right answer, but still
leak secrets out through the time it takes to run or the amount of power it
uses.

The list above shows how diverse and widespread side channel attacks are.
Despite their prevalence, we still don't understand them very well. The next
section presents some open questions that this project will try to answer.

In this section we described what side-channel attacks are and why we care about
them. In Section \ref{sec:problem}, we present the research questions that this
project will focus on. Section \ref{sec:stepsforward} explains how the research
will proceed and what the results should look like. Section \ref{sec:conclusion}
concludes.

\section{Research Problem}
\label{sec:problem}

In the last section, we saw that side channel attacks are everywhere, and that
they can be devastating for a system's security. In order to build defenses
against side-channel attacks, we have to understand them. Our ignorance about
side channel attacks boils down to two central questions: Under what conditions
are side-channel attacks possible? and when a side-channel vulnerability exists,
how reliably can it be exploited?

Many examples of side-channel attacks have been presented in the literature.
Most research is focused on developing new attack vectors. Acoustic
cryptanalysis \cite{genkin2013rsa} and extracting keys from ground potential
fluctuations \cite{genkin2014get} are recent examples. These new types of attack
are almost universally applied to cryptography. This makes sense, because
leaking a small number of bits (i.e. the key) from a cryptographic application
weakens and may completely break its security. To compound the issue,
cryptography makes heavy, repeated, use of the few bits that it does need to
keep secret, making them more likely to influence the side channels.

It makes sense why cryptography is attacked, but it is unclear if this focus on
cryptography is because non-cryptographic programs can not be attacked, or if
cryptography is just the low-hanging fruit that gets attacked first. This is an
important question to resolve, since if non-cryptographic programs \emph{are}
just as vulnerable to side-channels, we might be overlooking serious
vulnerabilities in the programs we use every day. An attack against a regular
program can be just as devastating as an attack against an implementation of
cryptography. For example, if it is possible for one user to spy on another
user's keystrokes as they compose an email, that is just as serious an attack as
if the first user could steal the second user's PGP private key.

This project will try to answer that question. In particular, we ask:

\begin{enumerate}
\item What impact do side-channel attacks have on non-cryptographic systems?

\item Under what conditions are side channels a threat? Under what conditions
      are they not?

\item Does every developer have to be aware of side channels and design their
      code to resist attacks? Or is it enough to care about side channels only
      for sensitive applications like cryptography?

\item Are there automated technological defenses against side-channel attacks?
      Are there easy-to-follow coding guidelines that reduce the impact of side
      channels?

\item Can side channels be used for good? Could we, perhaps, use side channels
      monitoring as a mechanism to spot attacks?
\end{enumerate}

If this project finds that non-cryptographic attacks represent real threats, it
will motivate the research community to widen the scope of their attacks and
defenses. If the result is that side-channels cannot be applied to regular
programs, we can learn \emph{why} not, and begin to apply that knowledge to
build defenses.

The next section discusses the methods we will use to answer these questions,
and what the results of the project should look like.

\section{Steps Forward}
\label{sec:stepsforward}

The best way to find out whether non-cryptographic applications are vulnerable
to side channel attacks is to attack them.

The recently-discovered \textsc{Flush+Reload} \cite{yarom2013flush} side-channel
attack is particularly applicable. Using \textsc{Flush+Reload}, a spy process
running on a modern Intel processor can observe which code a victim process
executes over time. The attack is unusually effective, providing cache-line-size
granularity and a sample rate on the order of one megahertz.

The authors of the \textsc{Flush+Reload} attack originally applied it to extract
the secret RSA key from GnuPG. More recently, it has been applied to extract
secret keys from OpenSSL's ECDSA implementation \cite{yarom2014recovering}.
I have speculated \cite{hornby2013flush} that the attack would also be
devastating against non-cryptographic applications, since it is so general.

The project will begin by using \textsc{Flush+Reload} to attack
non-cryptographic programs like web browsers, databases, email clients, and text
editors. Because of the physical nature of side-channel attacks, there will be
a consistent focus on experiment and reproducibility.

If the hypothesis is correct, the project will result in a set of reliable
attacks against non-cryptographic applications. This will demonstrate that it is
worth looking for side-channel vulnerabilities in that space of programs, which
will hopefully incentivize other researchers to pay closer attention to these
threats.

If the hypothesis is wrong, and no significant attacks are found, then the
result will be some new knowledge about why attacking non-cryptographic
applications is hard.

A rough timeline for the project is as follows:

\begin{itemize}
    \item \textbf{October:} Finish the related work list with critiques assignment and
          reproduce the FLUSH+RELOAD attack against GnuPG.
      \item \textbf{November:} Begin attacking non-crypto applications with the
          FLUSH+RELOAD attack.
      \item \textbf{December:} Continue vulnerability-finding and start writing
          the final paper.
      \item \textbf{January:} Interim report. Some significant vulnerabilities
          will hopefully have been found by this time.
      \item \textbf{February-March:} Paper-writing and presentation preparation. Extra time
          will be used to explore other questions (e.g. defenses), and to find
          better attacks.
      \item \textbf{April:} Finish the final paper and give the presentation. If
          the paper is of good quality, submit to a conference (Current target
          is USENIX WOOT '15).
\end{itemize}

\section{Summary}
\label{sec:conclusion}

Side channel attacks exploit the boundary between the abstract notion of
a computer program and its real physical implementation. They use physical
measurements of time, power, vibrations, and radiation to learn what a computer
is doing. Most research has focused on applying these attacks to cryptographic
systems. The project proposed here is to explore the impact these attacks have
on everyday programs like web browsers, databases, and text editors. This
question will be explored by trying to apply previously-known side channel
techniques to these programs. The outcome will be a greater understanding of the
impact side-channel attacks have in that context.

\bibliographystyle{abbrv}
\bibliography{proposal}

\end{document}

