% TEMPLATE for Usenix papers, specifically to meet requirements of
%  USENIX '05
% originally a template for producing IEEE-format articles using LaTeX.
%   written by Matthew Ward, CS Department, Worcester Polytechnic Institute.
% adapted by David Beazley for his excellent SWIG paper in Proceedings,
%   Tcl 96
% turned into a smartass generic template by De Clarke, with thanks to
%   both the above pioneers
% use at your own risk.  Complaints to /dev/null.
% make it two column with no page numbering, default is 10 point

% Munged by Fred Douglis <douglis@research.att.com> 10/97 to separate
% the .sty file from the LaTeX source template, so that people can
% more easily include the .sty file into an existing document.  Also
% changed to more closely follow the style guidelines as represented
% by the Word sample file. 

% Note that since 2010, USENIX does not require endnotes. If you want
% foot of page notes, don't include the endnotes package in the 
% usepackage command, below.

% This version uses the latex2e styles, not the very ancient 2.09 stuff.
\documentclass[letterpaper,twocolumn,10pt]{article}
\usepackage{usenix,epsfig}

\usepackage{url}
\usepackage{microtype}
\usepackage{epsfig}
\usepackage{amssymb}
\usepackage{amsmath}
\usepackage{amsfonts}
\usepackage{draftwatermark}
\SetWatermarkText{Preprint -- not for distribution.}
\SetWatermarkScale{0.4}

\begin{document}

%don't want date printed
\date{}

%make title bold and 14 pt font (Latex default is non-bold, 16 pt)
\title{\Large \bf Distinguishing Inputs with the FLUSH+RELOAD Cache Side Channel}

%for single author (just remove % characters)
\author{
{\rm Taylor Hornby}\\
University of Calgary
\and
{\rm John Aycock}\\
University of Calgary
% copy the following lines to add more authors
% \and
% {\rm Name}\\
%Name Institution
} % end author

\maketitle

% Use the following at camera-ready time to suppress page numbers.
% Comment it out when you first submit the paper for review.
%\thispagestyle{empty}


\subsection*{Abstract}
Side channel attacks are typically used to break implementations of
cryptography. Recently, side channel attacks are being discovered in more
general settings that violate user privacy. We build on this work by showing
that the \textsc{Flush+Reload} L3 cache side channel from Yuval Yarom and
Katrina Falkner \cite{yarom2013flush} can be used to distinguish between inputs
to non-cryptographic programs. 

We present three attacks that can be carried out on a multi-user system. Our
first attack is against the Links command-line web browser. We show that an
attacker can determine which of the top 100 Wikipedia pages the user visited
(correct 94\% of the time). Our second attack is against the Poppler PDF
rendering library. We show that an attacker can determine which of a set of 127
PDFs were rendered by Poppler (correct 98\% of the time). Finally, we show how
an attacker can determine whether the TrueCrypt volume a user just mounted
contained a hidden volume or not (correct above 80\% of the time, but the attack
only works on half of our test systems). We also describe how new input
distinguishing attacks can be discovered semi-automatically.

\section{Introduction}
\label{sec:intro}

Side channels are a well-known category of vulnerabilities where an adversary is
able to learn information about their victim by observing implicit, rather than
explicit, channels of information. For example, information may be
unintentionally leaked out through radio signals, execution time, power usage,
or through the state of a computer processor's memory cache.

Side channels are most commonly used to extract secret keys from implementations
of cryptography. For example, Thomas Messerges et al.\ were able to extract the
secret RSA key from a smart card by looking at the amount of power it uses
\cite{messerges1999power}. More recently, Genkin et al.\ were able to extract
a secret RSA key just by listening to the sounds a computer makes as it is
decrypting data \cite{genkin2013rsa}. 

Side channel attacks are also being used to compromise user privacy. For
example, Liang Cai and Hao Chen have shown that keystrokes (including typed PIN
numbers) can be reliably recovered from accelerometer data on smartphones
\cite{cai2012practicality}. Shuo Chen et al.\ showed that observing a web
application's behavior reveals information about the user's input. They gave
real-world examples where they could determine which medical conditions the user
was searching for, what their income range was, and how they allocated their
investment funds \cite{chen2010side}.

In this paper, we study additional ways of using side channels to attack the
user's privacy. We show that the generic L3 cache side channel called
\textsc{Flush+Reload} \cite{yarom2013flush} can be applied in non-cryptography
settings to meaningfully violate confidentiality. We give three example attacks, where
each attack determines which of a set of inputs the user passed to a program.
Specifically, we show how an attacker on a shared system can (1) determine which
of the top 100 Wikipedia pages a user visited in the Links web browser, (2)
determine which of 127 PDF files a user passed to the \texttt{pdftops} command,
and (3) determine whether the TrueCrypt volume a user just mounted was a hidden
volume or not.

In addition to the three attacks, we also describe how the process of
discovering new attacks can be partially automated.

To the best of our knowledge, this is the first time that a generic cache-based
side channel has been used to compromise privacy by attacking
a non-cryptographic application. The attacks we present show that cache side
channels have implications beyond extracting secret keys from cryptography
software.

In the next section, we summarize the \textsc{Flush+Reload} side channel that
all of our novel attacks are based on. Section~\ref{sec:distinguishing}
describes how \textsc{Flush+Reload} can be used to distinguish inputs and how we
partially automated the process of discovering new attacks.
Section~\ref{sec:implementation} describes the software tools the attacks and
experiments are built from. Section~\ref{sec:experimentsetup} describes our
experimental setup. The three attacks are described and evaluated by experiment
in Section~\ref{sec:results}. Section~\ref{sec:relwork} highlights recent work
that uses side channels to violate privacy and explains how our contributions
fit in with that work. Finally, we list some ideas for future work in
Section~\ref{sec:future} and then conclude in Section~\ref{sec:conclusion}.

\section{The Flush+Reload Attack}
\label{sec:flushreload}

The attacks we present use the \textsc{Flush+Reload} attack by Yuval Yarom and
Katrina Falkner \cite{yarom2013flush}. The side channel was first described by
Bangerter et al.\ \cite{gullasch2011cache} where it was used on AES lookup
tables to extract keys and plaintext during encryption. Yarom and
Falkner realized the technique could be applied more generally to spy on the
code a process executes. It was given the name \textsc{Flush+Reload} and has
since broken GnuPG \cite{yarom2013flush} and OpenSSL \cite{benger2014ooh,
yarom2014recovering}.

In this section we give a brief explanation of the \textsc{Flush+Reload} attack
with enough detail to understand our attacks. For full details, refer to
Yarom and Falkner~\cite{yarom2013flush}.

\textsc{Flush+Reload} is a generic L3 (or last-level, in case the system does
not have an L3 cache) cache side channel attack. It takes advantage of
executable code page sharing between processes. If Alice is the first to run
a program, the operating system will load the program into physical memory. When
Bob comes along and runs the same program, instead of loading a second copy into
memory, the operating system will set Bob's page tables to use the copy that was
loaded into memory for Alice. The result is that both Alice and Bob are
using the same physical memory. If Bob can determine which cache lines of the
shared memory are present in the cache over time, he can tell which cache lines
Alice's process accesses over time. This is what the \textsc{Flush+Reload}
attack does.

Suppose an attacker knows their victim is running a certain program and would
like to see which code the victim's process is running. \textsc{Flush+Reload}
lets the attacker select a handful of cache lines to watch, called ``probes,''
which are specified as addresses in the program's executable code.
The attacker flushes those lines out of the cache, waits for a certain number of
clock cycles, then times how long it takes to read those lines. If the read is
fast (i.e. consistent with being in the L3 cache), it means the victim accessed
the line during the waiting period. If the read is slow, the victim did not
access the line. The attacker can repeat the process (flushing, then reloading)
to see which code the victim process is executing over time.

The attack relies on certain assumptions.  First, it
assumes only one instance of the spied-on program is running at
a time. If multiple instances are running, an access to a probed cache line by
any instance will trigger the probe. Second, only one attacking process can run at
a time on a CPU core. If a pair of attacking processes have to contend for CPU
time, they will miss measurements.

In summary, the attacker specifies a few (approximately 1 to 5, where adding more
probes makes the measurements less reliable) probe locations within a binary and
the amount of clock cycles to wait between measurements.
For example, the attack against GnuPG puts probes on the RSA modular
exponentiation routines and uses a waiting time of 2048 clock cycles
\cite{yarom2013flush}.
As output, the attacker
learns the sequence of probes that were accessed during the waiting periods.

\section{Using \textsc{Flush+Reload} to Distinguish Inputs}
\label{sec:distinguishing}

In the previous section we described how an attacker can use
\textsc{Flush+Reload} against a program.
In this
section, we give a novel procedure for distinguishing between a set of possible
inputs through the \textsc{Flush+Reload} side channel. Later, we present actual
attacks against Links and Poppler that use this technique.

In the scenario we are interested in, the attacker knows the victim is going to
run a program on some input. The attacker also knows that the input is one of a set
of inputs, all of which are available to the attacker. By spying on the program
with \textsc{Flush+Reload}, the attacker hopes to figure out which input in the
set the program was run on.

For example, suppose Alice has just been diagnosed with an illness and is using
Wikipedia to research it. The attacker, Mallory, knows Alice was just diagnosed,
and wants to find out which illness she has. Mallory knows that Alice will visit
one of, say, 100 Wikipedia pages that are about illnesses. Mallory's goal is to
find out, from the \textsc{Flush+Reload} probe sequence he observes, which of
the 100 pages Alice visited.

The first step in the attack is to figure out where to place the
\textsc{Flush+Reload} probes to best distinguish the input. This can be done by
sifting through the program's code and finding functions whose frequency and
order of execution are likely to depend heavily on the input. It is possible to
do this by hand, but some automation can make it easier, a topic we return to in
Section~\ref{sec:automate}. While the side channel allows probes to be placed on
any cache line, we only consider cache lines containing function entrypoints.

With the probes selected, the attack proceeds in three stages. The first stage
is a training stage. The attack is most successful when the training stage is
carried out on the victim's machine, but the attack still works at a lower
success rate when the attacker trains on a different system. Next, the actual
attack happens: the attacker spies on the victim as they execute the program on
one of the inputs. Finally, the attacker uses the training data and output from
the attack stage to determine which input was given to the program.

In the training stage, the attacker simply runs the \textsc{Flush+Reload} attack
tool against themselves as they run the program on every input in the set
multiple times. We refer to the number of times each input is sampled by $T$. As
$T$ increases, the training stage takes longer to perform, but the success rate
of the attack can increase. If there are $N$ different inputs, the attacker will
be left with $NT$ training samples. The training stage can be done either before
or after the attack stage, and the attacker can re-use a training set multiple
times, so they only need to train once to go through with many attacks and
recoveries.

In the attack stage, the attacker runs the \textsc{Flush+Reload} attack tool
against the victim as the victim passes one of the inputs to the program. The
attack tool records the observed probe sequence.

In the recovery stage, the attacker finds the probe sequence in the training set
that is closest to the one obtained from the victim. Closeness is measured using
the Levenshtein distance~\cite{levenshtein1966binary}, which is defined as the
smallest number of basic edits (single-character insertions, deletions, and
replacements) needed to bring one string to the other. The attacker computes the
Levenshtein distance between the victim's probe sequence and all of the training
probe sequences, and the one with the smallest Levenshtein distance is assumed
to correspond to the input the victim passed to the program.

Probe sequences are represented by strings of single-character probe names. An
example is given in Figure~\ref{figure:probetext}. Before computing the
Levenshtein distance, the probe sequences are first condensed by collapsing
sequentially repeated hits of the same probe down into a single hit of that
probe, and then the whole string is truncated to 1000 characters. This is done
to make the recovery stage faster, since the algorithm we use to compute the
Levenshtein distance has quadratic running time.

\begin{figure}
    \centering
\begin{verbatim}
BDBABABABABABABABABABABABCABABCBABCBCABC
ABABABCBABCABCBCBDABABACACADBABDBCBABCBA
BABDBCBABDBCABDBCABDBCABCBCBCBABCBCBABAB
DBABABDABDBCABDABABCABDBCBCBABCBCBABCBCB
ABCABCABDABDCBCABCABCBABDABCABDBCABDBCBA
\end{verbatim}
\caption{The first 200 bytes (of 24,984) of a condensed observed probe sequence
    while visiting the Facebook Wikipedia page in Links. From run links/0015.
    \texttt{A} corresponds to \texttt{kill\_html\_stack\_item()}, \texttt{B} to
    \texttt{html\_stack\_dup()}, \texttt{C} to \texttt{html\_a()}, and
\texttt{D} to \texttt{parse\_html()}.} \label{figure:probetext}
\end{figure}

\subsection{Using Automation to Find Probes}
\label{sec:automate}
\vspace*{-.25em} % fix widow

In our experience, the hardest part of creating a new input distinguishing
attack is finding the set of probes to spy on. We want to find a set of
functions whose order of execution best exhibits the differences between the
inputs. Statically, one way to do this is to read the program's code, understand
it, and then manually select some functions that are heavily involved in
processing the input. This takes time and effort. 

In this section, we present a method for partially automating probe discovery.
We describe a tool which takes a list of functions as input and finds a set of
them that are good at distinguishing the inputs. The tool requires symbols, but
not source code: as long as the attacker has access to a non-stripped binary,
they can use the tool to find a good probe set without having a deep
understanding of the program. Once a good set of probes is discovered, symbols
are no longer necessary, and the attack will work without them.

The tool first looks at the symbols in the binary to list all of the
functions. Next, some human intervention is needed to narrow down the list of
functions into a smaller set that are more likely to depend on the input. This
filtering can be as simple as keeping all functions whose names contain
``\texttt{html}'' to get the list of html-parsing functions, as we have done for
Links, or by looking for functions whose names begin with ``\texttt{Gfx::op}''
to get the list of functions responsible for PDF commands, as we have done for
Poppler.

Once the list of functions has been reduced to a manageable size (e.g. less
than 100 functions), each potential probe is tested individually and
automatically. A candidate probe is tested by placing a GDB breakpoint on the
first instruction in the function and counting the number of times it gets
executed as the program is run. This is done three times for each probe: twice
on one input, giving counts $c_1$ and $c_2$, and once on a second, different,
input, giving the count $c_3$. A score is given to each probe, equal to $|c_3
- c_1| - |c_2 - c_1|$. This rewards functions whose execution counts are stable
on the same input but vary on different inputs, meaning they likely depend on
the input. The list of candidate probes are sorted by their scores, and all but
the top 10 are rejected.

% TODO: We're using "function" and "probe" interchangably here.... we use
% "probe" in the rest of the paper. Should we just pick one, or should we keep
% it to reinforce the idea that we are probing functions?

The reduced and sorted list of functions is then used to generate all possible
sets of size 4. Sets that contain functions that are within 3 cache lines of
each other are immediately removed. If two probes are too close together, the
CPU's instruction prefetch may trigger one whenever another
one is actually executed, in which case there is no point having both probes.
The remaining sets are sorted by the sums of their rank in the input list (which
was sorted by score), and this sorted list is given to the user.
In our attacks, we always used the first probe set in this output list.

\section{Attack Implementation}
\label{sec:implementation}

The attacks are built on the following tools that we built.

\begin{itemize}
    \item A \textsc{Flush+Reload} attack tool, written in C, takes a binary and probe
        addresses with single-character names as arguments. When probes are hit,
        their single-character names are printed to standard output, with pipe
        characters separating the waiting periods. This tool is
        a re-implementation of the tool the original paper's
        (\cite{yarom2013flush}) authors provided us.

    \item A Ruby class that makes it easy for Ruby scripts to run the attack
        tool and monitor its output. It provides a callback whenever there is
        a burst of probe hits. This lets the attack script react to events (such
        as the entire execution of a command). Our attacks are implemented as
        Ruby scripts that use this class.

    \item A Ruby implementation of the semi-automated attack discovery procedure
        described in Section~\ref{sec:automate}. The tool takes as input a list
        of executable code addresses in a program and two different inputs. It
        uses the two inputs to test the quality of each candidate probe, and
        outputs a set of probes that are likely able to distinguish the inputs.

    \item A Ruby implementation of the attack procedure described in
          Section~\ref{sec:distinguishing}. There is one program for each stage
          in the attack.
\end{itemize}

The source code for all of these programs is available online; see
Appendix~\ref{sec:reproducing}. Next, we describe how we experimentally
evaluated our attacks using these tools.

\section{Experiment Setup}
\label{sec:experimentsetup}

We ran experiments on two systems. The first system is a laptop; the second
system is a dedicated server hosting a low-traffic website and operating as
a relay for the Tor network. The specifications of these systems are given in
Table \ref{table:specs}. We will refer to these systems as System 1 and System
2, respectively.

\begin{table*}
    \centering
\begin{tabular}{|c|c|c|}
    \hline
    & \textbf{System 1} & \textbf{System 2} \\
    \hline
    \textbf{Use} & Laptop & Web server \\
    \hline
    \textbf{OS} & Arch Linux (February 2015) & Debian Wheezy \\
    \hline
    \textbf{CPU} & Intel Core2 Duo P8700 2.53GHz & Intel Xeon E3-1245 V2 3.40GHz  \\
    \hline
    \textbf{RAM} & 4GiB DDR2 800MHz & 32GiB DDR3 1333MHz \\
    \hline
    \textbf{L1 Cache} & 64KiB Split & 256KiB Unified \\
    \hline
    \textbf{L2 Cache} & 3MiB Unified & 1MiB Unified \\
    \hline
    \textbf{L3 Cache} & None & 8MiB Unified \\
    \hline
\end{tabular}
\caption{System specifications. Cache specifications were obtained by the
\texttt{dmidecode} utility. System 1 does not have a L3
cache, but FLUSH+RELOAD works with its L2 cache as it is shared between cores.}
\label{table:specs}
\end{table*}

To test the input distinguishing attacks, we first perform the training stage,
taking $T$ samples of each input. Then, for every input in the set of size $N$,
the vulnerable program is run on the input $S$ times. The probe sequences from
each of the $SN$ runs are put through the recovery stage independently, and we
check how many of the recoveries are correct.

All of the experiments have been automated so that they are easy to repeat. The
source code and data from all experiment runs have been published so that other
researchers can verify and reproduce our work (see Appendix
\ref{sec:reproducing}).

All experiments were run with low system load. The experiments are all run
within the same user account for simplicity, but we confirmed independently that
the attacks work when the attacker is on one account and the spy is on another.

\section{Attacks}
\label{sec:results}

We present three attacks. The first attack lets an attacker tell whether which
of the top 100 Wikipedia pages the victim visited using the Links web browser.
The second attack does the same for the Poppler PDF rendering library,
distinguishing between 127 PDF transcripts of 2014 Canadian parliamentary
debates. The Links and Poppler attacks both use the input distinguishing
procedure exactly as described in Section~\ref{sec:distinguishing}. The third
attack, which is implemented differently for reasons explained later, determines whether a mounted TrueCrypt
volume contained a hidden volume or not.

\subsection{Links}

Links is a command-line web browser. We are interested in learning which web
page a user is visiting, out of a set of known pages. We found that it is
possible to reliably distinguish between the top 100 Wikipedia pages of
2013~\cite{wikitop2013}.

To find the Links probes, we ran the automatic probe finding tool on the HTML
parsing functions inside Links. The names of these functions all contain the
string ``html'', so we filtered the list of all functions for ``html'' and gave
the resulting list of 77 functions to the probe finding tool.
The ``Bird'' and
``Feather'' Wikipedia pages were chosen arbitrarily as inputs to
the probe-finding tool. The tool gave us the following probe set:

\begin{itemize}
\setlength{\itemsep}{0pt}
    \item \texttt{html\_stack\_item()}
    \item \texttt{html\_stack\_dup()}
    \item \texttt{html\_a()}
    \item \texttt{parse\_html()}
\end{itemize}

% FIXME: The endnotes after numbers look like exponents.

\begin{figure}[h]
    \centering
    % GNUPLOT: LaTeX picture
\setlength{\unitlength}{0.240900pt}
\ifx\plotpoint\undefined\newsavebox{\plotpoint}\fi
\sbox{\plotpoint}{\rule[-0.200pt]{0.400pt}{0.400pt}}%
\begin{picture}(944,944)(0,0)
\sbox{\plotpoint}{\rule[-0.200pt]{0.400pt}{0.400pt}}%
\put(131.0,131.0){\rule[-0.200pt]{4.818pt}{0.400pt}}
\put(111,131){\makebox(0,0)[r]{$0$}}
\put(863.0,131.0){\rule[-0.200pt]{4.818pt}{0.400pt}}
\put(131.0,228.0){\rule[-0.200pt]{4.818pt}{0.400pt}}
\put(111,228){\makebox(0,0)[r]{$10$}}
\put(863.0,228.0){\rule[-0.200pt]{4.818pt}{0.400pt}}
\put(131.0,324.0){\rule[-0.200pt]{4.818pt}{0.400pt}}
\put(111,324){\makebox(0,0)[r]{$20$}}
\put(863.0,324.0){\rule[-0.200pt]{4.818pt}{0.400pt}}
\put(131.0,421.0){\rule[-0.200pt]{4.818pt}{0.400pt}}
\put(111,421){\makebox(0,0)[r]{$30$}}
\put(863.0,421.0){\rule[-0.200pt]{4.818pt}{0.400pt}}
\put(131.0,517.0){\rule[-0.200pt]{4.818pt}{0.400pt}}
\put(111,517){\makebox(0,0)[r]{$40$}}
\put(863.0,517.0){\rule[-0.200pt]{4.818pt}{0.400pt}}
\put(131.0,614.0){\rule[-0.200pt]{4.818pt}{0.400pt}}
\put(111,614){\makebox(0,0)[r]{$50$}}
\put(863.0,614.0){\rule[-0.200pt]{4.818pt}{0.400pt}}
\put(131.0,710.0){\rule[-0.200pt]{4.818pt}{0.400pt}}
\put(111,710){\makebox(0,0)[r]{$60$}}
\put(863.0,710.0){\rule[-0.200pt]{4.818pt}{0.400pt}}
\put(131.0,807.0){\rule[-0.200pt]{4.818pt}{0.400pt}}
\put(111,807){\makebox(0,0)[r]{$70$}}
\put(863.0,807.0){\rule[-0.200pt]{4.818pt}{0.400pt}}
\put(131.0,903.0){\rule[-0.200pt]{4.818pt}{0.400pt}}
\put(111,903){\makebox(0,0)[r]{$80$}}
\put(863.0,903.0){\rule[-0.200pt]{4.818pt}{0.400pt}}
\put(165.0,131.0){\rule[-0.200pt]{0.400pt}{4.818pt}}
\put(165,90){\makebox(0,0){10}}
\put(165.0,883.0){\rule[-0.200pt]{0.400pt}{4.818pt}}
\put(234.0,131.0){\rule[-0.200pt]{0.400pt}{4.818pt}}
\put(234,90){\makebox(0,0){9}}
\put(234.0,883.0){\rule[-0.200pt]{0.400pt}{4.818pt}}
\put(302.0,131.0){\rule[-0.200pt]{0.400pt}{4.818pt}}
\put(302,90){\makebox(0,0){8}}
\put(302.0,883.0){\rule[-0.200pt]{0.400pt}{4.818pt}}
\put(370.0,131.0){\rule[-0.200pt]{0.400pt}{4.818pt}}
\put(370,90){\makebox(0,0){7}}
\put(370.0,883.0){\rule[-0.200pt]{0.400pt}{4.818pt}}
\put(439.0,131.0){\rule[-0.200pt]{0.400pt}{4.818pt}}
\put(439,90){\makebox(0,0){6}}
\put(439.0,883.0){\rule[-0.200pt]{0.400pt}{4.818pt}}
\put(507.0,131.0){\rule[-0.200pt]{0.400pt}{4.818pt}}
\put(507,90){\makebox(0,0){5}}
\put(507.0,883.0){\rule[-0.200pt]{0.400pt}{4.818pt}}
\put(575.0,131.0){\rule[-0.200pt]{0.400pt}{4.818pt}}
\put(575,90){\makebox(0,0){4}}
\put(575.0,883.0){\rule[-0.200pt]{0.400pt}{4.818pt}}
\put(644.0,131.0){\rule[-0.200pt]{0.400pt}{4.818pt}}
\put(644,90){\makebox(0,0){3}}
\put(644.0,883.0){\rule[-0.200pt]{0.400pt}{4.818pt}}
\put(712.0,131.0){\rule[-0.200pt]{0.400pt}{4.818pt}}
\put(712,90){\makebox(0,0){2}}
\put(712.0,883.0){\rule[-0.200pt]{0.400pt}{4.818pt}}
\put(780.0,131.0){\rule[-0.200pt]{0.400pt}{4.818pt}}
\put(780,90){\makebox(0,0){1}}
\put(780.0,883.0){\rule[-0.200pt]{0.400pt}{4.818pt}}
\put(849.0,131.0){\rule[-0.200pt]{0.400pt}{4.818pt}}
\put(849,90){\makebox(0,0){0}}
\put(849.0,883.0){\rule[-0.200pt]{0.400pt}{4.818pt}}
\put(131.0,131.0){\rule[-0.200pt]{0.400pt}{185.975pt}}
\put(131.0,131.0){\rule[-0.200pt]{181.157pt}{0.400pt}}
\put(883.0,131.0){\rule[-0.200pt]{0.400pt}{185.975pt}}
\put(131.0,903.0){\rule[-0.200pt]{181.157pt}{0.400pt}}
\put(30,517){\makebox(0,0){\rotatebox{90}{Number of pages}}}
\put(507,29){\makebox(0,0){Correctly recovered out of 10 trials}}
\put(148,131){\rule{8.4315pt}{165.257pt}}
\put(148.0,131.0){\rule[-0.200pt]{0.400pt}{165.016pt}}
\put(148.0,816.0){\rule[-0.200pt]{8.191pt}{0.400pt}}
\put(182.0,131.0){\rule[-0.200pt]{0.400pt}{165.016pt}}
\put(216,131){\rule{8.6724pt}{35.1714pt}}
\put(148.0,131.0){\rule[-0.200pt]{8.191pt}{0.400pt}}
\put(216.0,131.0){\rule[-0.200pt]{0.400pt}{34.930pt}}
\put(216.0,276.0){\rule[-0.200pt]{8.431pt}{0.400pt}}
\put(251.0,131.0){\rule[-0.200pt]{0.400pt}{34.930pt}}
\put(285,131){\rule{8.4315pt}{21.1992pt}}
\put(216.0,131.0){\rule[-0.200pt]{8.431pt}{0.400pt}}
\put(285.0,131.0){\rule[-0.200pt]{0.400pt}{20.958pt}}
\put(285.0,218.0){\rule[-0.200pt]{8.191pt}{0.400pt}}
\put(319.0,131.0){\rule[-0.200pt]{0.400pt}{20.958pt}}
\put(353,131){\rule{8.4315pt}{4.818pt}}
\put(285.0,131.0){\rule[-0.200pt]{8.191pt}{0.400pt}}
\put(353.0,131.0){\rule[-0.200pt]{0.400pt}{4.577pt}}
\put(353.0,150.0){\rule[-0.200pt]{8.191pt}{0.400pt}}
\put(387.0,131.0){\rule[-0.200pt]{0.400pt}{4.577pt}}
\put(558,131){\rule{8.4315pt}{2.6499pt}}
\put(353.0,131.0){\rule[-0.200pt]{8.191pt}{0.400pt}}
\put(558.0,131.0){\rule[-0.200pt]{0.400pt}{2.409pt}}
\put(558.0,141.0){\rule[-0.200pt]{8.191pt}{0.400pt}}
\put(592.0,131.0){\rule[-0.200pt]{0.400pt}{2.409pt}}
\put(627,131){\rule{8.4315pt}{2.6499pt}}
\put(558.0,131.0){\rule[-0.200pt]{8.191pt}{0.400pt}}
\put(627.0,131.0){\rule[-0.200pt]{0.400pt}{2.409pt}}
\put(627.0,141.0){\rule[-0.200pt]{8.191pt}{0.400pt}}
\put(661.0,131.0){\rule[-0.200pt]{0.400pt}{2.409pt}}
\put(695,131){\rule{8.4315pt}{2.6499pt}}
\put(627.0,131.0){\rule[-0.200pt]{8.191pt}{0.400pt}}
\put(695.0,131.0){\rule[-0.200pt]{0.400pt}{2.409pt}}
\put(695.0,141.0){\rule[-0.200pt]{8.191pt}{0.400pt}}
\put(729.0,131.0){\rule[-0.200pt]{0.400pt}{2.409pt}}
\put(695.0,131.0){\rule[-0.200pt]{8.191pt}{0.400pt}}
\put(131.0,131.0){\rule[-0.200pt]{0.400pt}{185.975pt}}
\put(131.0,131.0){\rule[-0.200pt]{181.157pt}{0.400pt}}
\put(883.0,131.0){\rule[-0.200pt]{0.400pt}{185.975pt}}
\put(131.0,903.0){\rule[-0.200pt]{181.157pt}{0.400pt}}
\end{picture}

    \vspace*{-2em}
    \caption{The distribution of success rates for individual pages. The $x$
        axis is the number of times the page was correctly identified out of ten
        trials; the $y$ axis is the number of pages that were correctly
        identified that many times. Most pages were correctly identified in all
        ten trials.}
    \vspace*{-1em}
    \label{figure:distribution}
\end{figure}

On System 1 with $T=10$ and $S=10$,\footnote{links/0014; interpretation of
these labels is explained in Appendix~\ref{sec:reproducing}.} the correct page was
identified $940$ times out of $1000$ (94\%). In this experiment,
the average time it took to recover the name of the page from the training
sequences and the victim sequence was $4.1$ seconds. All pages in the set were
correctly identified at least twice out of the ten runs, and most were correctly
identified all 10 times. The distribution is plotted in
Figure~\ref{figure:distribution}.

A second run\footnote{links/0013} on System 1 with $T=10$ and $S=1$ saw the
correct page being identified $95$ times out of $100$ (95\%).
Again it took $4.1$ seconds to perform a recovery, on average.

On System 2 with $T=10$ and $S=1$,\footnote{links/0015} the correct page was
identified $98$ times out of $100$. Recovery took $136$ seconds on average. The
recovery took much longer on System 2 despite having a faster processor because
System 2 was configured to use a pure Ruby implementation of a Levenshtein
distance algorithm instead of the fast native implementation that System 1 was
was configured to use. With $T=10$ and $S=10$,\footnote{links/0016} the correct
page was identified $977$ times out of $1000$ (97.7\%). As on System
1, most of the pages were correctly identified 9 or 10 times out of 10.

A plot of the Levenshtein distances involved in a successful recovery are shown
in Figure~\ref{figure:youtube}, and those from an unsuccessful recovery are
shown in Figure~\ref{figure:minaj}.

All of the Wikipedia pages in the set have distinct lengths; it is
reasonable to ask if this attack is only distinguishing the pages by their
lengths. We know this is not the case, however, because the probe sequences are
truncated to 1000 characters before the Levenshtein distance is computed, and in
all of the experiment runs mentioned above, most of the training samples have
lengths above 1000 and thus most of the comparisons are between constant-length
1000-character strings. It is therefore the order of the probe hits that matters, and
we are really learning information about the content of the page,
not just its length.

We were curious whether the attack would work if the training were done on
a different system. To simulate this, we used the training set from an
experiment\footnote{links/0014} on System 1 to recover victim samples from
another experiment\footnote{links/0016} on System 2. The result was that 75.7\%
of the pages were recovered. Using the training set from System 2 and the victim
samples from System 1, the recovery was successful 82.4\% of the time. So it is
best to train on the victim's machine, but the attack can still work when the
training is done on a different system.

Before we built the probe finding tool, we ran experiments with a different set
of probes that we found manually. We chose these probes by trial and error,
looking through the Links source code to find functions whose frequency and
order of execution should depend on the contents of Wikipedia pages. These are:

\begin{itemize}
\setlength{\itemsep}{0pt}
    \item \texttt{parse\_html()}
    \item \texttt{html\_stack\_dup()}
    \item \texttt{html\_h()}
    \item \texttt{html\_span()}
\end{itemize}

Note that two of the manually-selected probes were re-discovered by the
automation tool.

With these probes, we saw 76\% correct classification on System 1 with $T=5$ and
$S=1$.\footnote{links/0002. This experiment was run before we started truncating
to 1000 characters, so the full strings were compared.} On System 2 with $T=5$
and $S=10$\footnote{links/0010} we saw 88\% correctness. With $T=10$ and $S=1$
we saw 91\%.\footnote{links/0005. This experiment was also run before we started
truncating to 1000 characters.} This suggests that using more training samples
increases the success rate of the attack, and that the automatically discovered
probes are just as good or slightly better than the ones we found manually.

\begin{figure*}
    \centering
    % GNUPLOT: LaTeX picture
\setlength{\unitlength}{0.240900pt}
\ifx\plotpoint\undefined\newsavebox{\plotpoint}\fi
\sbox{\plotpoint}{\rule[-0.200pt]{0.400pt}{0.400pt}}%
\begin{picture}(1500,900)(0,0)
\sbox{\plotpoint}{\rule[-0.200pt]{0.400pt}{0.400pt}}%
\put(171.0,131.0){\rule[-0.200pt]{4.818pt}{0.400pt}}
\put(151,131){\makebox(0,0)[r]{ 0}}
\put(1419.0,131.0){\rule[-0.200pt]{4.818pt}{0.400pt}}
\put(171.0,222.0){\rule[-0.200pt]{4.818pt}{0.400pt}}
\put(151,222){\makebox(0,0)[r]{ 100}}
\put(1419.0,222.0){\rule[-0.200pt]{4.818pt}{0.400pt}}
\put(171.0,313.0){\rule[-0.200pt]{4.818pt}{0.400pt}}
\put(151,313){\makebox(0,0)[r]{ 200}}
\put(1419.0,313.0){\rule[-0.200pt]{4.818pt}{0.400pt}}
\put(171.0,404.0){\rule[-0.200pt]{4.818pt}{0.400pt}}
\put(151,404){\makebox(0,0)[r]{ 300}}
\put(1419.0,404.0){\rule[-0.200pt]{4.818pt}{0.400pt}}
\put(171.0,495.0){\rule[-0.200pt]{4.818pt}{0.400pt}}
\put(151,495){\makebox(0,0)[r]{ 400}}
\put(1419.0,495.0){\rule[-0.200pt]{4.818pt}{0.400pt}}
\put(171.0,586.0){\rule[-0.200pt]{4.818pt}{0.400pt}}
\put(151,586){\makebox(0,0)[r]{ 500}}
\put(1419.0,586.0){\rule[-0.200pt]{4.818pt}{0.400pt}}
\put(171.0,677.0){\rule[-0.200pt]{4.818pt}{0.400pt}}
\put(151,677){\makebox(0,0)[r]{ 600}}
\put(1419.0,677.0){\rule[-0.200pt]{4.818pt}{0.400pt}}
\put(171.0,768.0){\rule[-0.200pt]{4.818pt}{0.400pt}}
\put(151,768){\makebox(0,0)[r]{ 700}}
\put(1419.0,768.0){\rule[-0.200pt]{4.818pt}{0.400pt}}
\put(171.0,859.0){\rule[-0.200pt]{4.818pt}{0.400pt}}
\put(151,859){\makebox(0,0)[r]{ 800}}
\put(1419.0,859.0){\rule[-0.200pt]{4.818pt}{0.400pt}}
\put(171.0,131.0){\rule[-0.200pt]{0.400pt}{4.818pt}}
\put(171,90){\makebox(0,0){ 0}}
\put(171.0,839.0){\rule[-0.200pt]{0.400pt}{4.818pt}}
\put(298.0,131.0){\rule[-0.200pt]{0.400pt}{4.818pt}}
\put(298,90){\makebox(0,0){ 10}}
\put(298.0,839.0){\rule[-0.200pt]{0.400pt}{4.818pt}}
\put(425.0,131.0){\rule[-0.200pt]{0.400pt}{4.818pt}}
\put(425,90){\makebox(0,0){ 20}}
\put(425.0,839.0){\rule[-0.200pt]{0.400pt}{4.818pt}}
\put(551.0,131.0){\rule[-0.200pt]{0.400pt}{4.818pt}}
\put(551,90){\makebox(0,0){ 30}}
\put(551.0,839.0){\rule[-0.200pt]{0.400pt}{4.818pt}}
\put(678.0,131.0){\rule[-0.200pt]{0.400pt}{4.818pt}}
\put(678,90){\makebox(0,0){ 40}}
\put(678.0,839.0){\rule[-0.200pt]{0.400pt}{4.818pt}}
\put(805.0,131.0){\rule[-0.200pt]{0.400pt}{4.818pt}}
\put(805,90){\makebox(0,0){ 50}}
\put(805.0,839.0){\rule[-0.200pt]{0.400pt}{4.818pt}}
\put(932.0,131.0){\rule[-0.200pt]{0.400pt}{4.818pt}}
\put(932,90){\makebox(0,0){ 60}}
\put(932.0,839.0){\rule[-0.200pt]{0.400pt}{4.818pt}}
\put(1059.0,131.0){\rule[-0.200pt]{0.400pt}{4.818pt}}
\put(1059,90){\makebox(0,0){ 70}}
\put(1059.0,839.0){\rule[-0.200pt]{0.400pt}{4.818pt}}
\put(1185.0,131.0){\rule[-0.200pt]{0.400pt}{4.818pt}}
\put(1185,90){\makebox(0,0){ 80}}
\put(1185.0,839.0){\rule[-0.200pt]{0.400pt}{4.818pt}}
\put(1312.0,131.0){\rule[-0.200pt]{0.400pt}{4.818pt}}
\put(1312,90){\makebox(0,0){ 90}}
\put(1312.0,839.0){\rule[-0.200pt]{0.400pt}{4.818pt}}
\put(1439.0,131.0){\rule[-0.200pt]{0.400pt}{4.818pt}}
\put(1439,90){\makebox(0,0){ 100}}
\put(1439.0,839.0){\rule[-0.200pt]{0.400pt}{4.818pt}}
\put(171.0,131.0){\rule[-0.200pt]{0.400pt}{175.375pt}}
\put(171.0,131.0){\rule[-0.200pt]{305.461pt}{0.400pt}}
\put(30,495){\makebox(0,0){\rotatebox{90}{Levenshtein Distance}}}
\put(805,29){\makebox(0,0){Page}}
\put(171,338){\makebox(0,0){$+$}}
\put(184,359){\makebox(0,0){$+$}}
\put(196,400){\makebox(0,0){$+$}}
\put(209,412){\makebox(0,0){$+$}}
\put(222,385){\makebox(0,0){$+$}}
\put(234,380){\makebox(0,0){$+$}}
\put(247,432){\makebox(0,0){$+$}}
\put(260,483){\makebox(0,0){$+$}}
\put(272,434){\makebox(0,0){$+$}}
\put(285,341){\makebox(0,0){$+$}}
\put(298,431){\makebox(0,0){$+$}}
\put(310,425){\makebox(0,0){$+$}}
\put(323,365){\makebox(0,0){$+$}}
\put(336,417){\makebox(0,0){$+$}}
\put(349,361){\makebox(0,0){$+$}}
\put(361,393){\makebox(0,0){$+$}}
\put(374,450){\makebox(0,0){$+$}}
\put(387,402){\makebox(0,0){$+$}}
\put(399,345){\makebox(0,0){$+$}}
\put(412,380){\makebox(0,0){$+$}}
\put(425,400){\makebox(0,0){$+$}}
\put(437,409){\makebox(0,0){$+$}}
\put(450,439){\makebox(0,0){$+$}}
\put(463,453){\makebox(0,0){$+$}}
\put(475,353){\makebox(0,0){$+$}}
\put(488,405){\makebox(0,0){$+$}}
\put(501,480){\makebox(0,0){$+$}}
\put(513,371){\makebox(0,0){$+$}}
\put(526,783){\makebox(0,0){$+$}}
\put(539,424){\makebox(0,0){$+$}}
\put(551,409){\makebox(0,0){$+$}}
\put(564,396){\makebox(0,0){$+$}}
\put(577,429){\makebox(0,0){$+$}}
\put(589,351){\makebox(0,0){$+$}}
\put(602,391){\makebox(0,0){$+$}}
\put(615,417){\makebox(0,0){$+$}}
\put(627,346){\makebox(0,0){$+$}}
\put(640,471){\makebox(0,0){$+$}}
\put(653,452){\makebox(0,0){$+$}}
\put(666,350){\makebox(0,0){$+$}}
\put(678,434){\makebox(0,0){$+$}}
\put(691,360){\makebox(0,0){$+$}}
\put(704,359){\makebox(0,0){$+$}}
\put(716,402){\makebox(0,0){$+$}}
\put(729,361){\makebox(0,0){$+$}}
\put(742,403){\makebox(0,0){$+$}}
\put(754,471){\makebox(0,0){$+$}}
\put(767,371){\makebox(0,0){$+$}}
\put(780,349){\makebox(0,0){$+$}}
\put(792,460){\makebox(0,0){$+$}}
\put(805,356){\makebox(0,0){$+$}}
\put(818,439){\makebox(0,0){$+$}}
\put(830,466){\makebox(0,0){$+$}}
\put(843,462){\makebox(0,0){$+$}}
\put(856,444){\makebox(0,0){$+$}}
\put(868,389){\makebox(0,0){$+$}}
\put(881,420){\makebox(0,0){$+$}}
\put(894,450){\makebox(0,0){$+$}}
\put(906,387){\makebox(0,0){$+$}}
\put(919,565){\makebox(0,0){$+$}}
\put(932,419){\makebox(0,0){$+$}}
\put(944,392){\makebox(0,0){$+$}}
\put(957,376){\makebox(0,0){$+$}}
\put(970,420){\makebox(0,0){$+$}}
\put(983,392){\makebox(0,0){$+$}}
\put(995,346){\makebox(0,0){$+$}}
\put(1008,375){\makebox(0,0){$+$}}
\put(1021,285){\makebox(0,0){$+$}}
\put(1033,440){\makebox(0,0){$+$}}
\put(1046,404){\makebox(0,0){$+$}}
\put(1059,367){\makebox(0,0){$+$}}
\put(1071,409){\makebox(0,0){$+$}}
\put(1084,411){\makebox(0,0){$+$}}
\put(1097,381){\makebox(0,0){$+$}}
\put(1109,424){\makebox(0,0){$+$}}
\put(1122,376){\makebox(0,0){$+$}}
\put(1135,361){\makebox(0,0){$+$}}
\put(1147,329){\makebox(0,0){$+$}}
\put(1160,505){\makebox(0,0){$+$}}
\put(1173,343){\makebox(0,0){$+$}}
\put(1185,348){\makebox(0,0){$+$}}
\put(1198,416){\makebox(0,0){$+$}}
\put(1211,410){\makebox(0,0){$+$}}
\put(1223,388){\makebox(0,0){$+$}}
\put(1236,347){\makebox(0,0){$+$}}
\put(1249,371){\makebox(0,0){$+$}}
\put(1261,414){\makebox(0,0){$+$}}
\put(1274,499){\makebox(0,0){$+$}}
\put(1287,402){\makebox(0,0){$+$}}
\put(1300,426){\makebox(0,0){$+$}}
\put(1312,370){\makebox(0,0){$+$}}
\put(1325,403){\makebox(0,0){$+$}}
\put(1338,530){\makebox(0,0){$+$}}
\put(1350,437){\makebox(0,0){$+$}}
\put(1363,357){\makebox(0,0){$+$}}
\put(1376,342){\makebox(0,0){$+$}}
\put(1388,427){\makebox(0,0){$+$}}
\put(1401,382){\makebox(0,0){$+$}}
\put(1414,447){\makebox(0,0){$+$}}
\put(1426,346){\makebox(0,0){$+$}}
\put(171,406){\makebox(0,0){$\times$}}
\put(184,362){\makebox(0,0){$\times$}}
\put(196,410){\makebox(0,0){$\times$}}
\put(209,409){\makebox(0,0){$\times$}}
\put(222,409){\makebox(0,0){$\times$}}
\put(234,399){\makebox(0,0){$\times$}}
\put(247,458){\makebox(0,0){$\times$}}
\put(260,487){\makebox(0,0){$\times$}}
\put(272,418){\makebox(0,0){$\times$}}
\put(285,343){\makebox(0,0){$\times$}}
\put(298,428){\makebox(0,0){$\times$}}
\put(310,421){\makebox(0,0){$\times$}}
\put(323,368){\makebox(0,0){$\times$}}
\put(336,423){\makebox(0,0){$\times$}}
\put(349,377){\makebox(0,0){$\times$}}
\put(361,356){\makebox(0,0){$\times$}}
\put(374,441){\makebox(0,0){$\times$}}
\put(387,405){\makebox(0,0){$\times$}}
\put(399,345){\makebox(0,0){$\times$}}
\put(412,428){\makebox(0,0){$\times$}}
\put(425,375){\makebox(0,0){$\times$}}
\put(437,418){\makebox(0,0){$\times$}}
\put(450,430){\makebox(0,0){$\times$}}
\put(463,446){\makebox(0,0){$\times$}}
\put(475,372){\makebox(0,0){$\times$}}
\put(488,402){\makebox(0,0){$\times$}}
\put(501,473){\makebox(0,0){$\times$}}
\put(513,415){\makebox(0,0){$\times$}}
\put(526,781){\makebox(0,0){$\times$}}
\put(539,434){\makebox(0,0){$\times$}}
\put(551,388){\makebox(0,0){$\times$}}
\put(564,396){\makebox(0,0){$\times$}}
\put(577,398){\makebox(0,0){$\times$}}
\put(589,330){\makebox(0,0){$\times$}}
\put(602,385){\makebox(0,0){$\times$}}
\put(615,411){\makebox(0,0){$\times$}}
\put(627,356){\makebox(0,0){$\times$}}
\put(640,477){\makebox(0,0){$\times$}}
\put(653,457){\makebox(0,0){$\times$}}
\put(666,343){\makebox(0,0){$\times$}}
\put(678,388){\makebox(0,0){$\times$}}
\put(691,376){\makebox(0,0){$\times$}}
\put(704,368){\makebox(0,0){$\times$}}
\put(716,409){\makebox(0,0){$\times$}}
\put(729,366){\makebox(0,0){$\times$}}
\put(742,405){\makebox(0,0){$\times$}}
\put(754,465){\makebox(0,0){$\times$}}
\put(767,379){\makebox(0,0){$\times$}}
\put(780,374){\makebox(0,0){$\times$}}
\put(792,479){\makebox(0,0){$\times$}}
\put(805,370){\makebox(0,0){$\times$}}
\put(818,421){\makebox(0,0){$\times$}}
\put(830,480){\makebox(0,0){$\times$}}
\put(843,476){\makebox(0,0){$\times$}}
\put(856,450){\makebox(0,0){$\times$}}
\put(868,372){\makebox(0,0){$\times$}}
\put(881,429){\makebox(0,0){$\times$}}
\put(894,437){\makebox(0,0){$\times$}}
\put(906,367){\makebox(0,0){$\times$}}
\put(919,554){\makebox(0,0){$\times$}}
\put(932,421){\makebox(0,0){$\times$}}
\put(944,392){\makebox(0,0){$\times$}}
\put(957,370){\makebox(0,0){$\times$}}
\put(970,419){\makebox(0,0){$\times$}}
\put(983,408){\makebox(0,0){$\times$}}
\put(995,329){\makebox(0,0){$\times$}}
\put(1008,373){\makebox(0,0){$\times$}}
\put(1021,333){\makebox(0,0){$\times$}}
\put(1033,494){\makebox(0,0){$\times$}}
\put(1046,418){\makebox(0,0){$\times$}}
\put(1059,381){\makebox(0,0){$\times$}}
\put(1071,399){\makebox(0,0){$\times$}}
\put(1084,400){\makebox(0,0){$\times$}}
\put(1097,384){\makebox(0,0){$\times$}}
\put(1109,404){\makebox(0,0){$\times$}}
\put(1122,361){\makebox(0,0){$\times$}}
\put(1135,369){\makebox(0,0){$\times$}}
\put(1147,383){\makebox(0,0){$\times$}}
\put(1160,513){\makebox(0,0){$\times$}}
\put(1173,339){\makebox(0,0){$\times$}}
\put(1185,355){\makebox(0,0){$\times$}}
\put(1198,415){\makebox(0,0){$\times$}}
\put(1211,429){\makebox(0,0){$\times$}}
\put(1223,383){\makebox(0,0){$\times$}}
\put(1236,345){\makebox(0,0){$\times$}}
\put(1249,405){\makebox(0,0){$\times$}}
\put(1261,447){\makebox(0,0){$\times$}}
\put(1274,494){\makebox(0,0){$\times$}}
\put(1287,430){\makebox(0,0){$\times$}}
\put(1300,415){\makebox(0,0){$\times$}}
\put(1312,368){\makebox(0,0){$\times$}}
\put(1325,358){\makebox(0,0){$\times$}}
\put(1338,534){\makebox(0,0){$\times$}}
\put(1350,413){\makebox(0,0){$\times$}}
\put(1363,369){\makebox(0,0){$\times$}}
\put(1376,348){\makebox(0,0){$\times$}}
\put(1388,407){\makebox(0,0){$\times$}}
\put(1401,392){\makebox(0,0){$\times$}}
\put(1414,438){\makebox(0,0){$\times$}}
\put(1426,378){\makebox(0,0){$\times$}}
\sbox{\plotpoint}{\rule[-0.400pt]{0.800pt}{0.800pt}}%
\put(171,390){\makebox(0,0){$\ast$}}
\put(184,384){\makebox(0,0){$\ast$}}
\put(196,384){\makebox(0,0){$\ast$}}
\put(209,406){\makebox(0,0){$\ast$}}
\put(222,411){\makebox(0,0){$\ast$}}
\put(234,406){\makebox(0,0){$\ast$}}
\put(247,431){\makebox(0,0){$\ast$}}
\put(260,484){\makebox(0,0){$\ast$}}
\put(272,433){\makebox(0,0){$\ast$}}
\put(285,338){\makebox(0,0){$\ast$}}
\put(298,422){\makebox(0,0){$\ast$}}
\put(310,421){\makebox(0,0){$\ast$}}
\put(323,375){\makebox(0,0){$\ast$}}
\put(336,421){\makebox(0,0){$\ast$}}
\put(349,372){\makebox(0,0){$\ast$}}
\put(361,381){\makebox(0,0){$\ast$}}
\put(374,464){\makebox(0,0){$\ast$}}
\put(387,394){\makebox(0,0){$\ast$}}
\put(399,329){\makebox(0,0){$\ast$}}
\put(412,389){\makebox(0,0){$\ast$}}
\put(425,395){\makebox(0,0){$\ast$}}
\put(437,403){\makebox(0,0){$\ast$}}
\put(450,440){\makebox(0,0){$\ast$}}
\put(463,451){\makebox(0,0){$\ast$}}
\put(475,365){\makebox(0,0){$\ast$}}
\put(488,406){\makebox(0,0){$\ast$}}
\put(501,470){\makebox(0,0){$\ast$}}
\put(513,400){\makebox(0,0){$\ast$}}
\put(526,767){\makebox(0,0){$\ast$}}
\put(539,446){\makebox(0,0){$\ast$}}
\put(551,403){\makebox(0,0){$\ast$}}
\put(564,366){\makebox(0,0){$\ast$}}
\put(577,392){\makebox(0,0){$\ast$}}
\put(589,364){\makebox(0,0){$\ast$}}
\put(602,383){\makebox(0,0){$\ast$}}
\put(615,409){\makebox(0,0){$\ast$}}
\put(627,361){\makebox(0,0){$\ast$}}
\put(640,465){\makebox(0,0){$\ast$}}
\put(653,441){\makebox(0,0){$\ast$}}
\put(666,350){\makebox(0,0){$\ast$}}
\put(678,416){\makebox(0,0){$\ast$}}
\put(691,371){\makebox(0,0){$\ast$}}
\put(704,371){\makebox(0,0){$\ast$}}
\put(716,394){\makebox(0,0){$\ast$}}
\put(729,365){\makebox(0,0){$\ast$}}
\put(742,404){\makebox(0,0){$\ast$}}
\put(754,467){\makebox(0,0){$\ast$}}
\put(767,359){\makebox(0,0){$\ast$}}
\put(780,372){\makebox(0,0){$\ast$}}
\put(792,468){\makebox(0,0){$\ast$}}
\put(805,359){\makebox(0,0){$\ast$}}
\put(818,423){\makebox(0,0){$\ast$}}
\put(830,476){\makebox(0,0){$\ast$}}
\put(843,473){\makebox(0,0){$\ast$}}
\put(856,448){\makebox(0,0){$\ast$}}
\put(868,364){\makebox(0,0){$\ast$}}
\put(881,415){\makebox(0,0){$\ast$}}
\put(894,450){\makebox(0,0){$\ast$}}
\put(906,370){\makebox(0,0){$\ast$}}
\put(919,550){\makebox(0,0){$\ast$}}
\put(932,431){\makebox(0,0){$\ast$}}
\put(944,389){\makebox(0,0){$\ast$}}
\put(957,392){\makebox(0,0){$\ast$}}
\put(970,425){\makebox(0,0){$\ast$}}
\put(983,399){\makebox(0,0){$\ast$}}
\put(995,346){\makebox(0,0){$\ast$}}
\put(1008,381){\makebox(0,0){$\ast$}}
\put(1021,317){\makebox(0,0){$\ast$}}
\put(1033,464){\makebox(0,0){$\ast$}}
\put(1046,410){\makebox(0,0){$\ast$}}
\put(1059,383){\makebox(0,0){$\ast$}}
\put(1071,390){\makebox(0,0){$\ast$}}
\put(1084,418){\makebox(0,0){$\ast$}}
\put(1097,373){\makebox(0,0){$\ast$}}
\put(1109,424){\makebox(0,0){$\ast$}}
\put(1122,372){\makebox(0,0){$\ast$}}
\put(1135,364){\makebox(0,0){$\ast$}}
\put(1147,371){\makebox(0,0){$\ast$}}
\put(1160,520){\makebox(0,0){$\ast$}}
\put(1173,343){\makebox(0,0){$\ast$}}
\put(1185,359){\makebox(0,0){$\ast$}}
\put(1198,412){\makebox(0,0){$\ast$}}
\put(1211,398){\makebox(0,0){$\ast$}}
\put(1223,389){\makebox(0,0){$\ast$}}
\put(1236,353){\makebox(0,0){$\ast$}}
\put(1249,383){\makebox(0,0){$\ast$}}
\put(1261,449){\makebox(0,0){$\ast$}}
\put(1274,494){\makebox(0,0){$\ast$}}
\put(1287,384){\makebox(0,0){$\ast$}}
\put(1300,417){\makebox(0,0){$\ast$}}
\put(1312,367){\makebox(0,0){$\ast$}}
\put(1325,345){\makebox(0,0){$\ast$}}
\put(1338,528){\makebox(0,0){$\ast$}}
\put(1350,439){\makebox(0,0){$\ast$}}
\put(1363,376){\makebox(0,0){$\ast$}}
\put(1376,353){\makebox(0,0){$\ast$}}
\put(1388,401){\makebox(0,0){$\ast$}}
\put(1401,373){\makebox(0,0){$\ast$}}
\put(1414,436){\makebox(0,0){$\ast$}}
\put(1426,376){\makebox(0,0){$\ast$}}
\sbox{\plotpoint}{\rule[-0.500pt]{1.000pt}{1.000pt}}%
\put(171,359){\raisebox{-.8pt}{\makebox(0,0){$\Box$}}}
\put(184,376){\raisebox{-.8pt}{\makebox(0,0){$\Box$}}}
\put(196,396){\raisebox{-.8pt}{\makebox(0,0){$\Box$}}}
\put(209,415){\raisebox{-.8pt}{\makebox(0,0){$\Box$}}}
\put(222,406){\raisebox{-.8pt}{\makebox(0,0){$\Box$}}}
\put(234,400){\raisebox{-.8pt}{\makebox(0,0){$\Box$}}}
\put(247,439){\raisebox{-.8pt}{\makebox(0,0){$\Box$}}}
\put(260,492){\raisebox{-.8pt}{\makebox(0,0){$\Box$}}}
\put(272,430){\raisebox{-.8pt}{\makebox(0,0){$\Box$}}}
\put(285,379){\raisebox{-.8pt}{\makebox(0,0){$\Box$}}}
\put(298,426){\raisebox{-.8pt}{\makebox(0,0){$\Box$}}}
\put(310,405){\raisebox{-.8pt}{\makebox(0,0){$\Box$}}}
\put(323,370){\raisebox{-.8pt}{\makebox(0,0){$\Box$}}}
\put(336,419){\raisebox{-.8pt}{\makebox(0,0){$\Box$}}}
\put(349,357){\raisebox{-.8pt}{\makebox(0,0){$\Box$}}}
\put(361,367){\raisebox{-.8pt}{\makebox(0,0){$\Box$}}}
\put(374,429){\raisebox{-.8pt}{\makebox(0,0){$\Box$}}}
\put(387,480){\raisebox{-.8pt}{\makebox(0,0){$\Box$}}}
\put(399,345){\raisebox{-.8pt}{\makebox(0,0){$\Box$}}}
\put(412,393){\raisebox{-.8pt}{\makebox(0,0){$\Box$}}}
\put(425,406){\raisebox{-.8pt}{\makebox(0,0){$\Box$}}}
\put(437,410){\raisebox{-.8pt}{\makebox(0,0){$\Box$}}}
\put(450,433){\raisebox{-.8pt}{\makebox(0,0){$\Box$}}}
\put(463,448){\raisebox{-.8pt}{\makebox(0,0){$\Box$}}}
\put(475,350){\raisebox{-.8pt}{\makebox(0,0){$\Box$}}}
\put(488,410){\raisebox{-.8pt}{\makebox(0,0){$\Box$}}}
\put(501,481){\raisebox{-.8pt}{\makebox(0,0){$\Box$}}}
\put(513,408){\raisebox{-.8pt}{\makebox(0,0){$\Box$}}}
\put(526,776){\raisebox{-.8pt}{\makebox(0,0){$\Box$}}}
\put(539,423){\raisebox{-.8pt}{\makebox(0,0){$\Box$}}}
\put(551,415){\raisebox{-.8pt}{\makebox(0,0){$\Box$}}}
\put(564,383){\raisebox{-.8pt}{\makebox(0,0){$\Box$}}}
\put(577,359){\raisebox{-.8pt}{\makebox(0,0){$\Box$}}}
\put(589,339){\raisebox{-.8pt}{\makebox(0,0){$\Box$}}}
\put(602,372){\raisebox{-.8pt}{\makebox(0,0){$\Box$}}}
\put(615,411){\raisebox{-.8pt}{\makebox(0,0){$\Box$}}}
\put(627,351){\raisebox{-.8pt}{\makebox(0,0){$\Box$}}}
\put(640,472){\raisebox{-.8pt}{\makebox(0,0){$\Box$}}}
\put(653,440){\raisebox{-.8pt}{\makebox(0,0){$\Box$}}}
\put(666,378){\raisebox{-.8pt}{\makebox(0,0){$\Box$}}}
\put(678,381){\raisebox{-.8pt}{\makebox(0,0){$\Box$}}}
\put(691,370){\raisebox{-.8pt}{\makebox(0,0){$\Box$}}}
\put(704,398){\raisebox{-.8pt}{\makebox(0,0){$\Box$}}}
\put(716,394){\raisebox{-.8pt}{\makebox(0,0){$\Box$}}}
\put(729,368){\raisebox{-.8pt}{\makebox(0,0){$\Box$}}}
\put(742,401){\raisebox{-.8pt}{\makebox(0,0){$\Box$}}}
\put(754,466){\raisebox{-.8pt}{\makebox(0,0){$\Box$}}}
\put(767,419){\raisebox{-.8pt}{\makebox(0,0){$\Box$}}}
\put(780,413){\raisebox{-.8pt}{\makebox(0,0){$\Box$}}}
\put(792,468){\raisebox{-.8pt}{\makebox(0,0){$\Box$}}}
\put(805,375){\raisebox{-.8pt}{\makebox(0,0){$\Box$}}}
\put(818,426){\raisebox{-.8pt}{\makebox(0,0){$\Box$}}}
\put(830,463){\raisebox{-.8pt}{\makebox(0,0){$\Box$}}}
\put(843,467){\raisebox{-.8pt}{\makebox(0,0){$\Box$}}}
\put(856,445){\raisebox{-.8pt}{\makebox(0,0){$\Box$}}}
\put(868,359){\raisebox{-.8pt}{\makebox(0,0){$\Box$}}}
\put(881,404){\raisebox{-.8pt}{\makebox(0,0){$\Box$}}}
\put(894,465){\raisebox{-.8pt}{\makebox(0,0){$\Box$}}}
\put(906,373){\raisebox{-.8pt}{\makebox(0,0){$\Box$}}}
\put(919,532){\raisebox{-.8pt}{\makebox(0,0){$\Box$}}}
\put(932,405){\raisebox{-.8pt}{\makebox(0,0){$\Box$}}}
\put(944,399){\raisebox{-.8pt}{\makebox(0,0){$\Box$}}}
\put(957,374){\raisebox{-.8pt}{\makebox(0,0){$\Box$}}}
\put(970,460){\raisebox{-.8pt}{\makebox(0,0){$\Box$}}}
\put(983,401){\raisebox{-.8pt}{\makebox(0,0){$\Box$}}}
\put(995,366){\raisebox{-.8pt}{\makebox(0,0){$\Box$}}}
\put(1008,380){\raisebox{-.8pt}{\makebox(0,0){$\Box$}}}
\put(1021,301){\raisebox{-.8pt}{\makebox(0,0){$\Box$}}}
\put(1033,486){\raisebox{-.8pt}{\makebox(0,0){$\Box$}}}
\put(1046,419){\raisebox{-.8pt}{\makebox(0,0){$\Box$}}}
\put(1059,390){\raisebox{-.8pt}{\makebox(0,0){$\Box$}}}
\put(1071,380){\raisebox{-.8pt}{\makebox(0,0){$\Box$}}}
\put(1084,365){\raisebox{-.8pt}{\makebox(0,0){$\Box$}}}
\put(1097,380){\raisebox{-.8pt}{\makebox(0,0){$\Box$}}}
\put(1109,404){\raisebox{-.8pt}{\makebox(0,0){$\Box$}}}
\put(1122,386){\raisebox{-.8pt}{\makebox(0,0){$\Box$}}}
\put(1135,362){\raisebox{-.8pt}{\makebox(0,0){$\Box$}}}
\put(1147,331){\raisebox{-.8pt}{\makebox(0,0){$\Box$}}}
\put(1160,514){\raisebox{-.8pt}{\makebox(0,0){$\Box$}}}
\put(1173,352){\raisebox{-.8pt}{\makebox(0,0){$\Box$}}}
\put(1185,372){\raisebox{-.8pt}{\makebox(0,0){$\Box$}}}
\put(1198,407){\raisebox{-.8pt}{\makebox(0,0){$\Box$}}}
\put(1211,371){\raisebox{-.8pt}{\makebox(0,0){$\Box$}}}
\put(1223,389){\raisebox{-.8pt}{\makebox(0,0){$\Box$}}}
\put(1236,338){\raisebox{-.8pt}{\makebox(0,0){$\Box$}}}
\put(1249,390){\raisebox{-.8pt}{\makebox(0,0){$\Box$}}}
\put(1261,449){\raisebox{-.8pt}{\makebox(0,0){$\Box$}}}
\put(1274,486){\raisebox{-.8pt}{\makebox(0,0){$\Box$}}}
\put(1287,381){\raisebox{-.8pt}{\makebox(0,0){$\Box$}}}
\put(1300,432){\raisebox{-.8pt}{\makebox(0,0){$\Box$}}}
\put(1312,366){\raisebox{-.8pt}{\makebox(0,0){$\Box$}}}
\put(1325,361){\raisebox{-.8pt}{\makebox(0,0){$\Box$}}}
\put(1338,528){\raisebox{-.8pt}{\makebox(0,0){$\Box$}}}
\put(1350,440){\raisebox{-.8pt}{\makebox(0,0){$\Box$}}}
\put(1363,425){\raisebox{-.8pt}{\makebox(0,0){$\Box$}}}
\put(1376,358){\raisebox{-.8pt}{\makebox(0,0){$\Box$}}}
\put(1388,399){\raisebox{-.8pt}{\makebox(0,0){$\Box$}}}
\put(1401,369){\raisebox{-.8pt}{\makebox(0,0){$\Box$}}}
\put(1414,430){\raisebox{-.8pt}{\makebox(0,0){$\Box$}}}
\put(1426,395){\raisebox{-.8pt}{\makebox(0,0){$\Box$}}}
\sbox{\plotpoint}{\rule[-0.600pt]{1.200pt}{1.200pt}}%
\put(171,367){\makebox(0,0){$\blacksquare$}}
\put(184,359){\makebox(0,0){$\blacksquare$}}
\put(196,403){\makebox(0,0){$\blacksquare$}}
\put(209,410){\makebox(0,0){$\blacksquare$}}
\put(222,413){\makebox(0,0){$\blacksquare$}}
\put(234,394){\makebox(0,0){$\blacksquare$}}
\put(247,470){\makebox(0,0){$\blacksquare$}}
\put(260,465){\makebox(0,0){$\blacksquare$}}
\put(272,431){\makebox(0,0){$\blacksquare$}}
\put(285,340){\makebox(0,0){$\blacksquare$}}
\put(298,433){\makebox(0,0){$\blacksquare$}}
\put(310,422){\makebox(0,0){$\blacksquare$}}
\put(323,372){\makebox(0,0){$\blacksquare$}}
\put(336,417){\makebox(0,0){$\blacksquare$}}
\put(349,365){\makebox(0,0){$\blacksquare$}}
\put(361,347){\makebox(0,0){$\blacksquare$}}
\put(374,410){\makebox(0,0){$\blacksquare$}}
\put(387,406){\makebox(0,0){$\blacksquare$}}
\put(399,346){\makebox(0,0){$\blacksquare$}}
\put(412,386){\makebox(0,0){$\blacksquare$}}
\put(425,386){\makebox(0,0){$\blacksquare$}}
\put(437,414){\makebox(0,0){$\blacksquare$}}
\put(450,431){\makebox(0,0){$\blacksquare$}}
\put(463,460){\makebox(0,0){$\blacksquare$}}
\put(475,332){\makebox(0,0){$\blacksquare$}}
\put(488,399){\makebox(0,0){$\blacksquare$}}
\put(501,475){\makebox(0,0){$\blacksquare$}}
\put(513,392){\makebox(0,0){$\blacksquare$}}
\put(526,776){\makebox(0,0){$\blacksquare$}}
\put(539,420){\makebox(0,0){$\blacksquare$}}
\put(551,406){\makebox(0,0){$\blacksquare$}}
\put(564,395){\makebox(0,0){$\blacksquare$}}
\put(577,411){\makebox(0,0){$\blacksquare$}}
\put(589,338){\makebox(0,0){$\blacksquare$}}
\put(602,386){\makebox(0,0){$\blacksquare$}}
\put(615,419){\makebox(0,0){$\blacksquare$}}
\put(627,376){\makebox(0,0){$\blacksquare$}}
\put(640,480){\makebox(0,0){$\blacksquare$}}
\put(653,462){\makebox(0,0){$\blacksquare$}}
\put(666,348){\makebox(0,0){$\blacksquare$}}
\put(678,411){\makebox(0,0){$\blacksquare$}}
\put(691,363){\makebox(0,0){$\blacksquare$}}
\put(704,347){\makebox(0,0){$\blacksquare$}}
\put(716,394){\makebox(0,0){$\blacksquare$}}
\put(729,356){\makebox(0,0){$\blacksquare$}}
\put(742,393){\makebox(0,0){$\blacksquare$}}
\put(754,473){\makebox(0,0){$\blacksquare$}}
\put(767,411){\makebox(0,0){$\blacksquare$}}
\put(780,371){\makebox(0,0){$\blacksquare$}}
\put(792,482){\makebox(0,0){$\blacksquare$}}
\put(805,374){\makebox(0,0){$\blacksquare$}}
\put(818,423){\makebox(0,0){$\blacksquare$}}
\put(830,466){\makebox(0,0){$\blacksquare$}}
\put(843,479){\makebox(0,0){$\blacksquare$}}
\put(856,445){\makebox(0,0){$\blacksquare$}}
\put(868,408){\makebox(0,0){$\blacksquare$}}
\put(881,403){\makebox(0,0){$\blacksquare$}}
\put(894,458){\makebox(0,0){$\blacksquare$}}
\put(906,379){\makebox(0,0){$\blacksquare$}}
\put(919,582){\makebox(0,0){$\blacksquare$}}
\put(932,412){\makebox(0,0){$\blacksquare$}}
\put(944,389){\makebox(0,0){$\blacksquare$}}
\put(957,387){\makebox(0,0){$\blacksquare$}}
\put(970,417){\makebox(0,0){$\blacksquare$}}
\put(983,407){\makebox(0,0){$\blacksquare$}}
\put(995,353){\makebox(0,0){$\blacksquare$}}
\put(1008,380){\makebox(0,0){$\blacksquare$}}
\put(1021,299){\makebox(0,0){$\blacksquare$}}
\put(1033,467){\makebox(0,0){$\blacksquare$}}
\put(1046,411){\makebox(0,0){$\blacksquare$}}
\put(1059,367){\makebox(0,0){$\blacksquare$}}
\put(1071,415){\makebox(0,0){$\blacksquare$}}
\put(1084,419){\makebox(0,0){$\blacksquare$}}
\put(1097,388){\makebox(0,0){$\blacksquare$}}
\put(1109,411){\makebox(0,0){$\blacksquare$}}
\put(1122,400){\makebox(0,0){$\blacksquare$}}
\put(1135,363){\makebox(0,0){$\blacksquare$}}
\put(1147,339){\makebox(0,0){$\blacksquare$}}
\put(1160,492){\makebox(0,0){$\blacksquare$}}
\put(1173,338){\makebox(0,0){$\blacksquare$}}
\put(1185,365){\makebox(0,0){$\blacksquare$}}
\put(1198,422){\makebox(0,0){$\blacksquare$}}
\put(1211,428){\makebox(0,0){$\blacksquare$}}
\put(1223,386){\makebox(0,0){$\blacksquare$}}
\put(1236,347){\makebox(0,0){$\blacksquare$}}
\put(1249,389){\makebox(0,0){$\blacksquare$}}
\put(1261,427){\makebox(0,0){$\blacksquare$}}
\put(1274,486){\makebox(0,0){$\blacksquare$}}
\put(1287,401){\makebox(0,0){$\blacksquare$}}
\put(1300,412){\makebox(0,0){$\blacksquare$}}
\put(1312,361){\makebox(0,0){$\blacksquare$}}
\put(1325,360){\makebox(0,0){$\blacksquare$}}
\put(1338,533){\makebox(0,0){$\blacksquare$}}
\put(1350,438){\makebox(0,0){$\blacksquare$}}
\put(1363,374){\makebox(0,0){$\blacksquare$}}
\put(1376,373){\makebox(0,0){$\blacksquare$}}
\put(1388,440){\makebox(0,0){$\blacksquare$}}
\put(1401,369){\makebox(0,0){$\blacksquare$}}
\put(1414,447){\makebox(0,0){$\blacksquare$}}
\put(1426,371){\makebox(0,0){$\blacksquare$}}
\sbox{\plotpoint}{\rule[-0.200pt]{0.400pt}{0.400pt}}%
\put(171.0,131.0){\rule[-0.200pt]{0.400pt}{175.375pt}}
\put(171.0,131.0){\rule[-0.200pt]{305.461pt}{0.400pt}}
\end{picture}

    \caption{A successful recovery. The Levenshtein distance between the
    training samples and a recording of the victim visiting the YouTube
    Wikipedia page. The shortest distance is visible at mark 68 on the page axis
    which corresponds to a YouTube training sample. The outlier at mark 29
    corresponds to a disambiguation page that has a different format from the
usual Wikipedia page. The different shapes in a column represent the five
training samples of that page. The order on the page axis is not meaningful.}
    \label{figure:youtube}
\end{figure*}

\begin{figure*}
    \centering
    % GNUPLOT: LaTeX picture
\setlength{\unitlength}{0.240900pt}
\ifx\plotpoint\undefined\newsavebox{\plotpoint}\fi
\sbox{\plotpoint}{\rule[-0.200pt]{0.400pt}{0.400pt}}%
\begin{picture}(1500,900)(0,0)
\sbox{\plotpoint}{\rule[-0.200pt]{0.400pt}{0.400pt}}%
\put(171.0,131.0){\rule[-0.200pt]{4.818pt}{0.400pt}}
\put(151,131){\makebox(0,0)[r]{ 0}}
\put(1419.0,131.0){\rule[-0.200pt]{4.818pt}{0.400pt}}
\put(171.0,277.0){\rule[-0.200pt]{4.818pt}{0.400pt}}
\put(151,277){\makebox(0,0)[r]{ 100}}
\put(1419.0,277.0){\rule[-0.200pt]{4.818pt}{0.400pt}}
\put(171.0,422.0){\rule[-0.200pt]{4.818pt}{0.400pt}}
\put(151,422){\makebox(0,0)[r]{ 200}}
\put(1419.0,422.0){\rule[-0.200pt]{4.818pt}{0.400pt}}
\put(171.0,568.0){\rule[-0.200pt]{4.818pt}{0.400pt}}
\put(151,568){\makebox(0,0)[r]{ 300}}
\put(1419.0,568.0){\rule[-0.200pt]{4.818pt}{0.400pt}}
\put(171.0,713.0){\rule[-0.200pt]{4.818pt}{0.400pt}}
\put(151,713){\makebox(0,0)[r]{ 400}}
\put(1419.0,713.0){\rule[-0.200pt]{4.818pt}{0.400pt}}
\put(171.0,859.0){\rule[-0.200pt]{4.818pt}{0.400pt}}
\put(151,859){\makebox(0,0)[r]{ 500}}
\put(1419.0,859.0){\rule[-0.200pt]{4.818pt}{0.400pt}}
\put(171.0,131.0){\rule[-0.200pt]{0.400pt}{4.818pt}}
\put(171,90){\makebox(0,0){ 0}}
\put(171.0,839.0){\rule[-0.200pt]{0.400pt}{4.818pt}}
\put(298.0,131.0){\rule[-0.200pt]{0.400pt}{4.818pt}}
\put(298,90){\makebox(0,0){ 10}}
\put(298.0,839.0){\rule[-0.200pt]{0.400pt}{4.818pt}}
\put(425.0,131.0){\rule[-0.200pt]{0.400pt}{4.818pt}}
\put(425,90){\makebox(0,0){ 20}}
\put(425.0,839.0){\rule[-0.200pt]{0.400pt}{4.818pt}}
\put(551.0,131.0){\rule[-0.200pt]{0.400pt}{4.818pt}}
\put(551,90){\makebox(0,0){ 30}}
\put(551.0,839.0){\rule[-0.200pt]{0.400pt}{4.818pt}}
\put(678.0,131.0){\rule[-0.200pt]{0.400pt}{4.818pt}}
\put(678,90){\makebox(0,0){ 40}}
\put(678.0,839.0){\rule[-0.200pt]{0.400pt}{4.818pt}}
\put(805.0,131.0){\rule[-0.200pt]{0.400pt}{4.818pt}}
\put(805,90){\makebox(0,0){ 50}}
\put(805.0,839.0){\rule[-0.200pt]{0.400pt}{4.818pt}}
\put(932.0,131.0){\rule[-0.200pt]{0.400pt}{4.818pt}}
\put(932,90){\makebox(0,0){ 60}}
\put(932.0,839.0){\rule[-0.200pt]{0.400pt}{4.818pt}}
\put(1059.0,131.0){\rule[-0.200pt]{0.400pt}{4.818pt}}
\put(1059,90){\makebox(0,0){ 70}}
\put(1059.0,839.0){\rule[-0.200pt]{0.400pt}{4.818pt}}
\put(1185.0,131.0){\rule[-0.200pt]{0.400pt}{4.818pt}}
\put(1185,90){\makebox(0,0){ 80}}
\put(1185.0,839.0){\rule[-0.200pt]{0.400pt}{4.818pt}}
\put(1312.0,131.0){\rule[-0.200pt]{0.400pt}{4.818pt}}
\put(1312,90){\makebox(0,0){ 90}}
\put(1312.0,839.0){\rule[-0.200pt]{0.400pt}{4.818pt}}
\put(1439.0,131.0){\rule[-0.200pt]{0.400pt}{4.818pt}}
\put(1439,90){\makebox(0,0){ 100}}
\put(1439.0,839.0){\rule[-0.200pt]{0.400pt}{4.818pt}}
\put(171.0,131.0){\rule[-0.200pt]{0.400pt}{175.375pt}}
\put(171.0,131.0){\rule[-0.200pt]{305.461pt}{0.400pt}}
\put(30,495){\makebox(0,0){\rotatebox{90}{Levenshtein Distance}}}
\put(805,29){\makebox(0,0){Page}}
\put(171,457){\makebox(0,0){$+$}}
\put(184,495){\makebox(0,0){$+$}}
\put(196,600){\makebox(0,0){$+$}}
\put(209,620){\makebox(0,0){$+$}}
\put(222,464){\makebox(0,0){$+$}}
\put(234,563){\makebox(0,0){$+$}}
\put(247,546){\makebox(0,0){$+$}}
\put(260,690){\makebox(0,0){$+$}}
\put(272,627){\makebox(0,0){$+$}}
\put(285,463){\makebox(0,0){$+$}}
\put(298,598){\makebox(0,0){$+$}}
\put(310,575){\makebox(0,0){$+$}}
\put(323,547){\makebox(0,0){$+$}}
\put(336,601){\makebox(0,0){$+$}}
\put(349,531){\makebox(0,0){$+$}}
\put(361,445){\makebox(0,0){$+$}}
\put(374,582){\makebox(0,0){$+$}}
\put(387,611){\makebox(0,0){$+$}}
\put(399,488){\makebox(0,0){$+$}}
\put(412,561){\makebox(0,0){$+$}}
\put(425,540){\makebox(0,0){$+$}}
\put(437,582){\makebox(0,0){$+$}}
\put(450,619){\makebox(0,0){$+$}}
\put(463,613){\makebox(0,0){$+$}}
\put(475,507){\makebox(0,0){$+$}}
\put(488,547){\makebox(0,0){$+$}}
\put(501,660){\makebox(0,0){$+$}}
\put(513,486){\makebox(0,0){$+$}}
\put(539,577){\makebox(0,0){$+$}}
\put(551,508){\makebox(0,0){$+$}}
\put(564,533){\makebox(0,0){$+$}}
\put(577,533){\makebox(0,0){$+$}}
\put(589,478){\makebox(0,0){$+$}}
\put(602,555){\makebox(0,0){$+$}}
\put(615,552){\makebox(0,0){$+$}}
\put(627,514){\makebox(0,0){$+$}}
\put(640,658){\makebox(0,0){$+$}}
\put(653,593){\makebox(0,0){$+$}}
\put(666,549){\makebox(0,0){$+$}}
\put(678,609){\makebox(0,0){$+$}}
\put(691,499){\makebox(0,0){$+$}}
\put(704,533){\makebox(0,0){$+$}}
\put(716,489){\makebox(0,0){$+$}}
\put(729,521){\makebox(0,0){$+$}}
\put(742,575){\makebox(0,0){$+$}}
\put(754,654){\makebox(0,0){$+$}}
\put(767,501){\makebox(0,0){$+$}}
\put(780,496){\makebox(0,0){$+$}}
\put(792,639){\makebox(0,0){$+$}}
\put(805,578){\makebox(0,0){$+$}}
\put(818,593){\makebox(0,0){$+$}}
\put(830,600){\makebox(0,0){$+$}}
\put(843,582){\makebox(0,0){$+$}}
\put(856,514){\makebox(0,0){$+$}}
\put(868,424){\makebox(0,0){$+$}}
\put(881,593){\makebox(0,0){$+$}}
\put(894,584){\makebox(0,0){$+$}}
\put(906,533){\makebox(0,0){$+$}}
\put(932,671){\makebox(0,0){$+$}}
\put(944,534){\makebox(0,0){$+$}}
\put(957,476){\makebox(0,0){$+$}}
\put(970,677){\makebox(0,0){$+$}}
\put(983,528){\makebox(0,0){$+$}}
\put(995,547){\makebox(0,0){$+$}}
\put(1008,518){\makebox(0,0){$+$}}
\put(1021,483){\makebox(0,0){$+$}}
\put(1033,574){\makebox(0,0){$+$}}
\put(1046,530){\makebox(0,0){$+$}}
\put(1059,496){\makebox(0,0){$+$}}
\put(1071,616){\makebox(0,0){$+$}}
\put(1084,534){\makebox(0,0){$+$}}
\put(1097,555){\makebox(0,0){$+$}}
\put(1109,594){\makebox(0,0){$+$}}
\put(1122,478){\makebox(0,0){$+$}}
\put(1135,562){\makebox(0,0){$+$}}
\put(1147,534){\makebox(0,0){$+$}}
\put(1160,638){\makebox(0,0){$+$}}
\put(1173,552){\makebox(0,0){$+$}}
\put(1185,545){\makebox(0,0){$+$}}
\put(1198,526){\makebox(0,0){$+$}}
\put(1211,536){\makebox(0,0){$+$}}
\put(1223,539){\makebox(0,0){$+$}}
\put(1236,496){\makebox(0,0){$+$}}
\put(1249,534){\makebox(0,0){$+$}}
\put(1261,578){\makebox(0,0){$+$}}
\put(1274,690){\makebox(0,0){$+$}}
\put(1287,527){\makebox(0,0){$+$}}
\put(1300,514){\makebox(0,0){$+$}}
\put(1312,463){\makebox(0,0){$+$}}
\put(1325,451){\makebox(0,0){$+$}}
\put(1338,756){\makebox(0,0){$+$}}
\put(1350,680){\makebox(0,0){$+$}}
\put(1363,459){\makebox(0,0){$+$}}
\put(1376,539){\makebox(0,0){$+$}}
\put(1388,467){\makebox(0,0){$+$}}
\put(1401,409){\makebox(0,0){$+$}}
\put(1414,582){\makebox(0,0){$+$}}
\put(1426,492){\makebox(0,0){$+$}}
\put(171,594){\makebox(0,0){$\times$}}
\put(184,494){\makebox(0,0){$\times$}}
\put(196,614){\makebox(0,0){$\times$}}
\put(209,585){\makebox(0,0){$\times$}}
\put(222,514){\makebox(0,0){$\times$}}
\put(234,578){\makebox(0,0){$\times$}}
\put(247,565){\makebox(0,0){$\times$}}
\put(260,725){\makebox(0,0){$\times$}}
\put(272,607){\makebox(0,0){$\times$}}
\put(285,496){\makebox(0,0){$\times$}}
\put(298,585){\makebox(0,0){$\times$}}
\put(310,594){\makebox(0,0){$\times$}}
\put(323,549){\makebox(0,0){$\times$}}
\put(336,625){\makebox(0,0){$\times$}}
\put(349,533){\makebox(0,0){$\times$}}
\put(361,432){\makebox(0,0){$\times$}}
\put(374,569){\makebox(0,0){$\times$}}
\put(387,574){\makebox(0,0){$\times$}}
\put(399,517){\makebox(0,0){$\times$}}
\put(412,584){\makebox(0,0){$\times$}}
\put(425,537){\makebox(0,0){$\times$}}
\put(437,563){\makebox(0,0){$\times$}}
\put(450,629){\makebox(0,0){$\times$}}
\put(463,609){\makebox(0,0){$\times$}}
\put(475,534){\makebox(0,0){$\times$}}
\put(488,505){\makebox(0,0){$\times$}}
\put(501,661){\makebox(0,0){$\times$}}
\put(513,566){\makebox(0,0){$\times$}}
\put(539,578){\makebox(0,0){$\times$}}
\put(551,435){\makebox(0,0){$\times$}}
\put(564,531){\makebox(0,0){$\times$}}
\put(577,534){\makebox(0,0){$\times$}}
\put(589,482){\makebox(0,0){$\times$}}
\put(602,572){\makebox(0,0){$\times$}}
\put(615,550){\makebox(0,0){$\times$}}
\put(627,565){\makebox(0,0){$\times$}}
\put(640,648){\makebox(0,0){$\times$}}
\put(653,595){\makebox(0,0){$\times$}}
\put(666,523){\makebox(0,0){$\times$}}
\put(678,542){\makebox(0,0){$\times$}}
\put(691,542){\makebox(0,0){$\times$}}
\put(704,520){\makebox(0,0){$\times$}}
\put(716,492){\makebox(0,0){$\times$}}
\put(729,539){\makebox(0,0){$\times$}}
\put(742,577){\makebox(0,0){$\times$}}
\put(754,660){\makebox(0,0){$\times$}}
\put(767,510){\makebox(0,0){$\times$}}
\put(780,507){\makebox(0,0){$\times$}}
\put(792,657){\makebox(0,0){$\times$}}
\put(805,584){\makebox(0,0){$\times$}}
\put(818,593){\makebox(0,0){$\times$}}
\put(830,610){\makebox(0,0){$\times$}}
\put(843,603){\makebox(0,0){$\times$}}
\put(856,508){\makebox(0,0){$\times$}}
\put(868,397){\makebox(0,0){$\times$}}
\put(881,610){\makebox(0,0){$\times$}}
\put(894,534){\makebox(0,0){$\times$}}
\put(906,521){\makebox(0,0){$\times$}}
\put(919,820){\makebox(0,0){$\times$}}
\put(932,662){\makebox(0,0){$\times$}}
\put(944,539){\makebox(0,0){$\times$}}
\put(957,495){\makebox(0,0){$\times$}}
\put(970,668){\makebox(0,0){$\times$}}
\put(983,563){\makebox(0,0){$\times$}}
\put(995,526){\makebox(0,0){$\times$}}
\put(1008,530){\makebox(0,0){$\times$}}
\put(1021,473){\makebox(0,0){$\times$}}
\put(1033,728){\makebox(0,0){$\times$}}
\put(1046,575){\makebox(0,0){$\times$}}
\put(1059,476){\makebox(0,0){$\times$}}
\put(1071,611){\makebox(0,0){$\times$}}
\put(1084,448){\makebox(0,0){$\times$}}
\put(1097,569){\makebox(0,0){$\times$}}
\put(1109,558){\makebox(0,0){$\times$}}
\put(1122,495){\makebox(0,0){$\times$}}
\put(1135,566){\makebox(0,0){$\times$}}
\put(1147,480){\makebox(0,0){$\times$}}
\put(1160,633){\makebox(0,0){$\times$}}
\put(1173,518){\makebox(0,0){$\times$}}
\put(1185,539){\makebox(0,0){$\times$}}
\put(1198,526){\makebox(0,0){$\times$}}
\put(1211,639){\makebox(0,0){$\times$}}
\put(1223,536){\makebox(0,0){$\times$}}
\put(1236,480){\makebox(0,0){$\times$}}
\put(1249,558){\makebox(0,0){$\times$}}
\put(1261,619){\makebox(0,0){$\times$}}
\put(1274,686){\makebox(0,0){$\times$}}
\put(1287,658){\makebox(0,0){$\times$}}
\put(1300,486){\makebox(0,0){$\times$}}
\put(1312,445){\makebox(0,0){$\times$}}
\put(1325,473){\makebox(0,0){$\times$}}
\put(1338,754){\makebox(0,0){$\times$}}
\put(1350,655){\makebox(0,0){$\times$}}
\put(1363,416){\makebox(0,0){$\times$}}
\put(1376,491){\makebox(0,0){$\times$}}
\put(1388,512){\makebox(0,0){$\times$}}
\put(1401,447){\makebox(0,0){$\times$}}
\put(1414,566){\makebox(0,0){$\times$}}
\put(1426,459){\makebox(0,0){$\times$}}
\sbox{\plotpoint}{\rule[-0.400pt]{0.800pt}{0.800pt}}%
\put(171,475){\makebox(0,0){$\ast$}}
\put(184,517){\makebox(0,0){$\ast$}}
\put(196,591){\makebox(0,0){$\ast$}}
\put(209,595){\makebox(0,0){$\ast$}}
\put(222,571){\makebox(0,0){$\ast$}}
\put(234,588){\makebox(0,0){$\ast$}}
\put(247,553){\makebox(0,0){$\ast$}}
\put(260,713){\makebox(0,0){$\ast$}}
\put(272,623){\makebox(0,0){$\ast$}}
\put(285,450){\makebox(0,0){$\ast$}}
\put(298,588){\makebox(0,0){$\ast$}}
\put(310,582){\makebox(0,0){$\ast$}}
\put(323,543){\makebox(0,0){$\ast$}}
\put(336,638){\makebox(0,0){$\ast$}}
\put(349,547){\makebox(0,0){$\ast$}}
\put(361,422){\makebox(0,0){$\ast$}}
\put(374,565){\makebox(0,0){$\ast$}}
\put(387,604){\makebox(0,0){$\ast$}}
\put(399,495){\makebox(0,0){$\ast$}}
\put(412,600){\makebox(0,0){$\ast$}}
\put(425,555){\makebox(0,0){$\ast$}}
\put(437,563){\makebox(0,0){$\ast$}}
\put(450,604){\makebox(0,0){$\ast$}}
\put(463,616){\makebox(0,0){$\ast$}}
\put(475,595){\makebox(0,0){$\ast$}}
\put(488,504){\makebox(0,0){$\ast$}}
\put(501,651){\makebox(0,0){$\ast$}}
\put(513,507){\makebox(0,0){$\ast$}}
\put(539,587){\makebox(0,0){$\ast$}}
\put(551,453){\makebox(0,0){$\ast$}}
\put(564,480){\makebox(0,0){$\ast$}}
\put(577,552){\makebox(0,0){$\ast$}}
\put(589,457){\makebox(0,0){$\ast$}}
\put(602,574){\makebox(0,0){$\ast$}}
\put(615,553){\makebox(0,0){$\ast$}}
\put(627,559){\makebox(0,0){$\ast$}}
\put(640,658){\makebox(0,0){$\ast$}}
\put(653,591){\makebox(0,0){$\ast$}}
\put(666,512){\makebox(0,0){$\ast$}}
\put(678,547){\makebox(0,0){$\ast$}}
\put(691,510){\makebox(0,0){$\ast$}}
\put(704,523){\makebox(0,0){$\ast$}}
\put(716,473){\makebox(0,0){$\ast$}}
\put(729,540){\makebox(0,0){$\ast$}}
\put(742,579){\makebox(0,0){$\ast$}}
\put(754,660){\makebox(0,0){$\ast$}}
\put(767,501){\makebox(0,0){$\ast$}}
\put(780,488){\makebox(0,0){$\ast$}}
\put(792,655){\makebox(0,0){$\ast$}}
\put(805,572){\makebox(0,0){$\ast$}}
\put(818,588){\makebox(0,0){$\ast$}}
\put(830,610){\makebox(0,0){$\ast$}}
\put(843,581){\makebox(0,0){$\ast$}}
\put(856,508){\makebox(0,0){$\ast$}}
\put(868,429){\makebox(0,0){$\ast$}}
\put(881,579){\makebox(0,0){$\ast$}}
\put(894,572){\makebox(0,0){$\ast$}}
\put(906,520){\makebox(0,0){$\ast$}}
\put(919,826){\makebox(0,0){$\ast$}}
\put(932,677){\makebox(0,0){$\ast$}}
\put(944,526){\makebox(0,0){$\ast$}}
\put(957,505){\makebox(0,0){$\ast$}}
\put(970,680){\makebox(0,0){$\ast$}}
\put(983,547){\makebox(0,0){$\ast$}}
\put(995,565){\makebox(0,0){$\ast$}}
\put(1008,526){\makebox(0,0){$\ast$}}
\put(1021,479){\makebox(0,0){$\ast$}}
\put(1033,696){\makebox(0,0){$\ast$}}
\put(1046,523){\makebox(0,0){$\ast$}}
\put(1059,480){\makebox(0,0){$\ast$}}
\put(1071,611){\makebox(0,0){$\ast$}}
\put(1084,533){\makebox(0,0){$\ast$}}
\put(1097,577){\makebox(0,0){$\ast$}}
\put(1109,601){\makebox(0,0){$\ast$}}
\put(1122,489){\makebox(0,0){$\ast$}}
\put(1135,581){\makebox(0,0){$\ast$}}
\put(1147,505){\makebox(0,0){$\ast$}}
\put(1160,645){\makebox(0,0){$\ast$}}
\put(1173,527){\makebox(0,0){$\ast$}}
\put(1185,550){\makebox(0,0){$\ast$}}
\put(1198,523){\makebox(0,0){$\ast$}}
\put(1211,483){\makebox(0,0){$\ast$}}
\put(1223,531){\makebox(0,0){$\ast$}}
\put(1236,498){\makebox(0,0){$\ast$}}
\put(1249,534){\makebox(0,0){$\ast$}}
\put(1261,623){\makebox(0,0){$\ast$}}
\put(1274,696){\makebox(0,0){$\ast$}}
\put(1287,486){\makebox(0,0){$\ast$}}
\put(1300,536){\makebox(0,0){$\ast$}}
\put(1312,448){\makebox(0,0){$\ast$}}
\put(1325,520){\makebox(0,0){$\ast$}}
\put(1338,694){\makebox(0,0){$\ast$}}
\put(1350,677){\makebox(0,0){$\ast$}}
\put(1363,427){\makebox(0,0){$\ast$}}
\put(1376,517){\makebox(0,0){$\ast$}}
\put(1388,489){\makebox(0,0){$\ast$}}
\put(1401,384){\makebox(0,0){$\ast$}}
\put(1414,558){\makebox(0,0){$\ast$}}
\put(1426,467){\makebox(0,0){$\ast$}}
\sbox{\plotpoint}{\rule[-0.500pt]{1.000pt}{1.000pt}}%
\put(171,453){\raisebox{-.8pt}{\makebox(0,0){$\Box$}}}
\put(184,498){\raisebox{-.8pt}{\makebox(0,0){$\Box$}}}
\put(196,619){\raisebox{-.8pt}{\makebox(0,0){$\Box$}}}
\put(209,620){\raisebox{-.8pt}{\makebox(0,0){$\Box$}}}
\put(222,507){\raisebox{-.8pt}{\makebox(0,0){$\Box$}}}
\put(234,581){\raisebox{-.8pt}{\makebox(0,0){$\Box$}}}
\put(247,546){\raisebox{-.8pt}{\makebox(0,0){$\Box$}}}
\put(260,716){\raisebox{-.8pt}{\makebox(0,0){$\Box$}}}
\put(272,614){\raisebox{-.8pt}{\makebox(0,0){$\Box$}}}
\put(285,459){\raisebox{-.8pt}{\makebox(0,0){$\Box$}}}
\put(298,572){\raisebox{-.8pt}{\makebox(0,0){$\Box$}}}
\put(310,581){\raisebox{-.8pt}{\makebox(0,0){$\Box$}}}
\put(323,545){\raisebox{-.8pt}{\makebox(0,0){$\Box$}}}
\put(336,616){\raisebox{-.8pt}{\makebox(0,0){$\Box$}}}
\put(349,491){\raisebox{-.8pt}{\makebox(0,0){$\Box$}}}
\put(361,435){\raisebox{-.8pt}{\makebox(0,0){$\Box$}}}
\put(374,641){\raisebox{-.8pt}{\makebox(0,0){$\Box$}}}
\put(387,646){\raisebox{-.8pt}{\makebox(0,0){$\Box$}}}
\put(399,515){\raisebox{-.8pt}{\makebox(0,0){$\Box$}}}
\put(412,600){\raisebox{-.8pt}{\makebox(0,0){$\Box$}}}
\put(425,547){\raisebox{-.8pt}{\makebox(0,0){$\Box$}}}
\put(437,584){\raisebox{-.8pt}{\makebox(0,0){$\Box$}}}
\put(450,642){\raisebox{-.8pt}{\makebox(0,0){$\Box$}}}
\put(463,607){\raisebox{-.8pt}{\makebox(0,0){$\Box$}}}
\put(475,472){\raisebox{-.8pt}{\makebox(0,0){$\Box$}}}
\put(488,531){\raisebox{-.8pt}{\makebox(0,0){$\Box$}}}
\put(501,665){\raisebox{-.8pt}{\makebox(0,0){$\Box$}}}
\put(513,527){\raisebox{-.8pt}{\makebox(0,0){$\Box$}}}
\put(539,588){\raisebox{-.8pt}{\makebox(0,0){$\Box$}}}
\put(551,488){\raisebox{-.8pt}{\makebox(0,0){$\Box$}}}
\put(564,520){\raisebox{-.8pt}{\makebox(0,0){$\Box$}}}
\put(577,508){\raisebox{-.8pt}{\makebox(0,0){$\Box$}}}
\put(589,502){\raisebox{-.8pt}{\makebox(0,0){$\Box$}}}
\put(602,543){\raisebox{-.8pt}{\makebox(0,0){$\Box$}}}
\put(615,553){\raisebox{-.8pt}{\makebox(0,0){$\Box$}}}
\put(627,549){\raisebox{-.8pt}{\makebox(0,0){$\Box$}}}
\put(640,678){\raisebox{-.8pt}{\makebox(0,0){$\Box$}}}
\put(653,591){\raisebox{-.8pt}{\makebox(0,0){$\Box$}}}
\put(666,524){\raisebox{-.8pt}{\makebox(0,0){$\Box$}}}
\put(678,553){\raisebox{-.8pt}{\makebox(0,0){$\Box$}}}
\put(691,498){\raisebox{-.8pt}{\makebox(0,0){$\Box$}}}
\put(704,543){\raisebox{-.8pt}{\makebox(0,0){$\Box$}}}
\put(716,495){\raisebox{-.8pt}{\makebox(0,0){$\Box$}}}
\put(729,547){\raisebox{-.8pt}{\makebox(0,0){$\Box$}}}
\put(742,547){\raisebox{-.8pt}{\makebox(0,0){$\Box$}}}
\put(754,665){\raisebox{-.8pt}{\makebox(0,0){$\Box$}}}
\put(767,581){\raisebox{-.8pt}{\makebox(0,0){$\Box$}}}
\put(780,492){\raisebox{-.8pt}{\makebox(0,0){$\Box$}}}
\put(792,673){\raisebox{-.8pt}{\makebox(0,0){$\Box$}}}
\put(805,594){\raisebox{-.8pt}{\makebox(0,0){$\Box$}}}
\put(818,574){\raisebox{-.8pt}{\makebox(0,0){$\Box$}}}
\put(830,591){\raisebox{-.8pt}{\makebox(0,0){$\Box$}}}
\put(843,582){\raisebox{-.8pt}{\makebox(0,0){$\Box$}}}
\put(856,512){\raisebox{-.8pt}{\makebox(0,0){$\Box$}}}
\put(868,451){\raisebox{-.8pt}{\makebox(0,0){$\Box$}}}
\put(881,591){\raisebox{-.8pt}{\makebox(0,0){$\Box$}}}
\put(894,639){\raisebox{-.8pt}{\makebox(0,0){$\Box$}}}
\put(906,520){\raisebox{-.8pt}{\makebox(0,0){$\Box$}}}
\put(919,809){\raisebox{-.8pt}{\makebox(0,0){$\Box$}}}
\put(932,593){\raisebox{-.8pt}{\makebox(0,0){$\Box$}}}
\put(944,552){\raisebox{-.8pt}{\makebox(0,0){$\Box$}}}
\put(957,479){\raisebox{-.8pt}{\makebox(0,0){$\Box$}}}
\put(970,633){\raisebox{-.8pt}{\makebox(0,0){$\Box$}}}
\put(983,550){\raisebox{-.8pt}{\makebox(0,0){$\Box$}}}
\put(995,457){\raisebox{-.8pt}{\makebox(0,0){$\Box$}}}
\put(1008,533){\raisebox{-.8pt}{\makebox(0,0){$\Box$}}}
\put(1021,505){\raisebox{-.8pt}{\makebox(0,0){$\Box$}}}
\put(1033,715){\raisebox{-.8pt}{\makebox(0,0){$\Box$}}}
\put(1046,585){\raisebox{-.8pt}{\makebox(0,0){$\Box$}}}
\put(1059,491){\raisebox{-.8pt}{\makebox(0,0){$\Box$}}}
\put(1071,584){\raisebox{-.8pt}{\makebox(0,0){$\Box$}}}
\put(1084,459){\raisebox{-.8pt}{\makebox(0,0){$\Box$}}}
\put(1097,539){\raisebox{-.8pt}{\makebox(0,0){$\Box$}}}
\put(1109,566){\raisebox{-.8pt}{\makebox(0,0){$\Box$}}}
\put(1122,502){\raisebox{-.8pt}{\makebox(0,0){$\Box$}}}
\put(1135,569){\raisebox{-.8pt}{\makebox(0,0){$\Box$}}}
\put(1147,510){\raisebox{-.8pt}{\makebox(0,0){$\Box$}}}
\put(1160,627){\raisebox{-.8pt}{\makebox(0,0){$\Box$}}}
\put(1173,536){\raisebox{-.8pt}{\makebox(0,0){$\Box$}}}
\put(1185,559){\raisebox{-.8pt}{\makebox(0,0){$\Box$}}}
\put(1198,501){\raisebox{-.8pt}{\makebox(0,0){$\Box$}}}
\put(1211,460){\raisebox{-.8pt}{\makebox(0,0){$\Box$}}}
\put(1223,528){\raisebox{-.8pt}{\makebox(0,0){$\Box$}}}
\put(1236,491){\raisebox{-.8pt}{\makebox(0,0){$\Box$}}}
\put(1249,543){\raisebox{-.8pt}{\makebox(0,0){$\Box$}}}
\put(1261,606){\raisebox{-.8pt}{\makebox(0,0){$\Box$}}}
\put(1274,687){\raisebox{-.8pt}{\makebox(0,0){$\Box$}}}
\put(1287,504){\raisebox{-.8pt}{\makebox(0,0){$\Box$}}}
\put(1300,511){\raisebox{-.8pt}{\makebox(0,0){$\Box$}}}
\put(1312,453){\raisebox{-.8pt}{\makebox(0,0){$\Box$}}}
\put(1325,470){\raisebox{-.8pt}{\makebox(0,0){$\Box$}}}
\put(1338,759){\raisebox{-.8pt}{\makebox(0,0){$\Box$}}}
\put(1350,684){\raisebox{-.8pt}{\makebox(0,0){$\Box$}}}
\put(1363,489){\raisebox{-.8pt}{\makebox(0,0){$\Box$}}}
\put(1376,523){\raisebox{-.8pt}{\makebox(0,0){$\Box$}}}
\put(1388,486){\raisebox{-.8pt}{\makebox(0,0){$\Box$}}}
\put(1401,399){\raisebox{-.8pt}{\makebox(0,0){$\Box$}}}
\put(1414,609){\raisebox{-.8pt}{\makebox(0,0){$\Box$}}}
\put(1426,478){\raisebox{-.8pt}{\makebox(0,0){$\Box$}}}
\sbox{\plotpoint}{\rule[-0.600pt]{1.200pt}{1.200pt}}%
\put(171,470){\makebox(0,0){$\blacksquare$}}
\put(184,514){\makebox(0,0){$\blacksquare$}}
\put(196,597){\makebox(0,0){$\blacksquare$}}
\put(209,591){\makebox(0,0){$\blacksquare$}}
\put(222,507){\makebox(0,0){$\blacksquare$}}
\put(234,555){\makebox(0,0){$\blacksquare$}}
\put(247,582){\makebox(0,0){$\blacksquare$}}
\put(260,699){\makebox(0,0){$\blacksquare$}}
\put(272,595){\makebox(0,0){$\blacksquare$}}
\put(285,453){\makebox(0,0){$\blacksquare$}}
\put(298,601){\makebox(0,0){$\blacksquare$}}
\put(310,593){\makebox(0,0){$\blacksquare$}}
\put(323,574){\makebox(0,0){$\blacksquare$}}
\put(336,609){\makebox(0,0){$\blacksquare$}}
\put(349,537){\makebox(0,0){$\blacksquare$}}
\put(361,416){\makebox(0,0){$\blacksquare$}}
\put(374,553){\makebox(0,0){$\blacksquare$}}
\put(387,559){\makebox(0,0){$\blacksquare$}}
\put(399,499){\makebox(0,0){$\blacksquare$}}
\put(412,600){\makebox(0,0){$\blacksquare$}}
\put(425,523){\makebox(0,0){$\blacksquare$}}
\put(437,575){\makebox(0,0){$\blacksquare$}}
\put(450,641){\makebox(0,0){$\blacksquare$}}
\put(463,638){\makebox(0,0){$\blacksquare$}}
\put(475,511){\makebox(0,0){$\blacksquare$}}
\put(488,527){\makebox(0,0){$\blacksquare$}}
\put(501,668){\makebox(0,0){$\blacksquare$}}
\put(513,479){\makebox(0,0){$\blacksquare$}}
\put(539,593){\makebox(0,0){$\blacksquare$}}
\put(551,437){\makebox(0,0){$\blacksquare$}}
\put(564,536){\makebox(0,0){$\blacksquare$}}
\put(577,502){\makebox(0,0){$\blacksquare$}}
\put(589,518){\makebox(0,0){$\blacksquare$}}
\put(602,581){\makebox(0,0){$\blacksquare$}}
\put(615,566){\makebox(0,0){$\blacksquare$}}
\put(627,578){\makebox(0,0){$\blacksquare$}}
\put(640,655){\makebox(0,0){$\blacksquare$}}
\put(653,595){\makebox(0,0){$\blacksquare$}}
\put(666,520){\makebox(0,0){$\blacksquare$}}
\put(678,568){\makebox(0,0){$\blacksquare$}}
\put(691,495){\makebox(0,0){$\blacksquare$}}
\put(704,547){\makebox(0,0){$\blacksquare$}}
\put(716,496){\makebox(0,0){$\blacksquare$}}
\put(729,518){\makebox(0,0){$\blacksquare$}}
\put(742,540){\makebox(0,0){$\blacksquare$}}
\put(754,664){\makebox(0,0){$\blacksquare$}}
\put(767,572){\makebox(0,0){$\blacksquare$}}
\put(780,501){\makebox(0,0){$\blacksquare$}}
\put(792,662){\makebox(0,0){$\blacksquare$}}
\put(805,584){\makebox(0,0){$\blacksquare$}}
\put(818,595){\makebox(0,0){$\blacksquare$}}
\put(830,603){\makebox(0,0){$\blacksquare$}}
\put(843,611){\makebox(0,0){$\blacksquare$}}
\put(856,531){\makebox(0,0){$\blacksquare$}}
\put(868,572){\makebox(0,0){$\blacksquare$}}
\put(881,566){\makebox(0,0){$\blacksquare$}}
\put(894,579){\makebox(0,0){$\blacksquare$}}
\put(906,536){\makebox(0,0){$\blacksquare$}}
\put(932,671){\makebox(0,0){$\blacksquare$}}
\put(944,523){\makebox(0,0){$\blacksquare$}}
\put(957,492){\makebox(0,0){$\blacksquare$}}
\put(970,681){\makebox(0,0){$\blacksquare$}}
\put(983,563){\makebox(0,0){$\blacksquare$}}
\put(995,559){\makebox(0,0){$\blacksquare$}}
\put(1008,518){\makebox(0,0){$\blacksquare$}}
\put(1021,521){\makebox(0,0){$\blacksquare$}}
\put(1033,689){\makebox(0,0){$\blacksquare$}}
\put(1046,539){\makebox(0,0){$\blacksquare$}}
\put(1059,502){\makebox(0,0){$\blacksquare$}}
\put(1071,638){\makebox(0,0){$\blacksquare$}}
\put(1084,563){\makebox(0,0){$\blacksquare$}}
\put(1097,600){\makebox(0,0){$\blacksquare$}}
\put(1109,569){\makebox(0,0){$\blacksquare$}}
\put(1122,518){\makebox(0,0){$\blacksquare$}}
\put(1135,581){\makebox(0,0){$\blacksquare$}}
\put(1147,540){\makebox(0,0){$\blacksquare$}}
\put(1160,697){\makebox(0,0){$\blacksquare$}}
\put(1173,523){\makebox(0,0){$\blacksquare$}}
\put(1185,555){\makebox(0,0){$\blacksquare$}}
\put(1198,496){\makebox(0,0){$\blacksquare$}}
\put(1211,526){\makebox(0,0){$\blacksquare$}}
\put(1223,527){\makebox(0,0){$\blacksquare$}}
\put(1236,507){\makebox(0,0){$\blacksquare$}}
\put(1249,523){\makebox(0,0){$\blacksquare$}}
\put(1261,533){\makebox(0,0){$\blacksquare$}}
\put(1274,700){\makebox(0,0){$\blacksquare$}}
\put(1287,543){\makebox(0,0){$\blacksquare$}}
\put(1300,512){\makebox(0,0){$\blacksquare$}}
\put(1312,443){\makebox(0,0){$\blacksquare$}}
\put(1325,464){\makebox(0,0){$\blacksquare$}}
\put(1338,756){\makebox(0,0){$\blacksquare$}}
\put(1350,674){\makebox(0,0){$\blacksquare$}}
\put(1363,444){\makebox(0,0){$\blacksquare$}}
\put(1376,523){\makebox(0,0){$\blacksquare$}}
\put(1388,489){\makebox(0,0){$\blacksquare$}}
\put(1401,413){\makebox(0,0){$\blacksquare$}}
\put(1414,598){\makebox(0,0){$\blacksquare$}}
\put(1426,456){\makebox(0,0){$\blacksquare$}}
\sbox{\plotpoint}{\rule[-0.200pt]{0.400pt}{0.400pt}}%
\put(171.0,131.0){\rule[-0.200pt]{0.400pt}{175.375pt}}
\put(171.0,131.0){\rule[-0.200pt]{305.461pt}{0.400pt}}
\end{picture}

    \caption{A failed recovery. The Levenshtein distance between the training samples and
        a recording of the victim visiting the Nicki Minaj Wikipedia page. The
        shortest distance (97 on the page axis) corresponds to a training sample
        of the Eminem Wikipedia page. The Nicki Minaj training samples still
        stand out (55 on the page axis). The different shapes in a column
        represent the five training samples of that page. The order on the page
        axis is not meaningful.}
    \label{figure:minaj}
\end{figure*}

\subsection{Poppler}

Poppler is a PDF rendering library that gets used in software such as Evince and
LibreOffice. For ease of automation, we attacked the \texttt{pdftops} program,
which converts PDF files into PostScript files using Poppler. As our input set,
we used 127 transcripts of 2014 parliamentary debates made available by the
Canadian government~\cite{hansard}.

Unlike Links, we did not have a working attack against Poppler before we used
the probe finding tool. The only human input in the creation of this attack was
the idea to look at the set of functions that execute PDF commands, an
intuitively obvious thing to try.

We used the automatic probe finding tool to find the best probes amongst the
functions responsible for executing PDF commands. These functions, of which
there are 77, are easily identified because their names begin with ``\texttt{Gfx::op}.''
The probe finding tool returned the following set of probes after using test
inputs \texttt{HAN040-E.PDF} and \texttt{HAN050-E.PDF} from our input set.

\begin{itemize}
\setlength{\itemsep}{0pt}
    \item \texttt{Gfx::opShowSpaceText(Object*, int)}
    \item \texttt{Gfx::opTextMoveSet(Object*, int)}
    \item \texttt{Gfx::opSetFont(Object*, int)}
    \item \texttt{Gfx::opTextNextLine(Object*, int)}
\end{itemize}

On System 1 with $T=5$ and $S=10,$\footnote{poppler/0003} the correct PDF was
identified $1258$ times out of $1270$ (99.1\%). All of the PDFs were reliably
identifiable: all but one were recovered 9 or 10 times out of 10; the other one
was recovered 8 times. In a repeat run on same system with $T=5$ and
$S=1$,\footnote{poppler/0006} the correct PDF was identified $124$ times out of
$127$ (97.6\%).

On System 2 with $T=5$ and $S=1$,\footnote{poppler/0001} the correct PDF was
identified $126$ times out of $127$ (99.2\%). With $T=5$ and
$S=10$,\footnote{poppler/0007} the PDF was correctly identified $1260$ times out
of $1270$ (99.2\%). Again, all but one were identified 9 or 10 times out of 10,
except for one that was identified correctly only 8 times.

Using a training set\footnote{poppler/0003} created on System 1 we were able to
recover 69.6\% of probe sequences recorded on System 2.\footnote{poppler/0007}
Using the training set from System 2, we could recover 69.1\% of the probe
sequences recorded on System 1.

To test how well the probe finding tool works in the absence of any human
guidance, we tried running it on the list of all 5,397 functions in the
Poppler library. The result is the following set of probes:

\begin{itemize}
\setlength{\itemsep}{0pt}
    \item \texttt{gmallocn()}
    \item \texttt{PSOutputDev::writePSString(GooString*)}
    \item \texttt{PSOutputDev::drawString(GfxState*, GooString*)}
    \item \texttt{PSOutputDev::updateTextShift(GfxState*, double)}
\end{itemize}

We ran an experiment\footnote{poppler/0004} with these probes and $T=5$ and $S=1$
on System 1, and the result was that the correct PDF was only identified 21
times out of 127, only 16.5\%. This is much worse than with the other set
of probes, but is still better than random guessing. With these probes, the
\textsc{Flush+Reload} tool ran into an error condition, where the CPU's
\texttt{RDTSC} counter changes non-monotonically, much more frequently than it
did with the other probes, and that is probably the reason for the poorer
result.

\subsection{TrueCrypt}

TrueCrypt is a popular disk encryption utility that supports storing encrypted
filesystems in files called TrueCrypt volumes. TrueCrypt gives users the option
to place a hidden volume inside a normal volume. Given the passphrase to the
outer normal volume, it is not supposed to be possible to determine whether an
inner hidden volume exists. This is to protect the user in case they are coerced
into revealing their passphrase -- they can reveal the passphrase to the outer
volume and the contents of the hidden volume will be safe. 

Our attack watches the victim mount a TrueCrypt volume and determines whether
they mounted a hidden volume or a normal one. This attack is not implemented the
same way as the Links and Poppler attacks. Our automatic probe discovery tool
does not work against TrueCrypt, since our tool only supports debugging newly
launched processes, and TrueCrypt's volume mounting code runs in a background
process. Even if the required functionality were added to the tool, we would not
expect it to work with TrueCrypt since the difference in code execution between
mounting a normal volume and mounting a hidden volume is minimal.

\subsubsection{Attack Implementation}

TrueCrypt is written in C++ and has two classes defining the layout of normal
and hidden volumes. They are \texttt{VolumeLayout\-V2Normal} and
\texttt{VolumeLayout\-V2Hidden}, respectively. There are three other classes for
the operating system encryption layout and volume layouts from older versions,
making five volume layout classes in total.

When TrueCrypt mounts a volume, it instantiates all five layout classes and
tries to decrypt the volume using each one. It has to do this because, by
design, TrueCrypt volumes are supposed to be indistinguishable from random data,
so there is no way to tell which volume layout is the right one in advance; the
only way is to try to decrypt the volume and see if it works.

Each class implements a \texttt{GetDataSize()} method. This method is only
called on a layout object after the volume has been successfully decrypted using
that layout, making it a good candidate for a \textsc{Flush+Reload} probe.

At first we tried to place probes on both \texttt{VolumeLayout\-V2Normal}'s
\texttt{GetDataSize()} and \texttt{VolumeLayout\-V2Hidden}'s
\texttt{GetDataSize()}. This did not work because in the binary, the hidden
method immediately follows the normal method, and instruction prefetching
triggers the hidden probe when the normal method gets executed.

To work around that problem, we placed two \textsc{Flush+Reload} probes. The
first is placed on the entry point to the TrueCrypt binary, so that we can tell
when the user runs a TrueCrypt command. The second is placed in
\texttt{VolumeLayout\-V2Normal}'s \texttt{GetDataSize()}, which will get hit
once when the TrueCrypt binary is loaded, and then once again only if the volume
is normal.

To find out whether the volume is normal or hidden, the attacker records the
probe sequence as the victim mounts the volume. They check if the probe sequence
ends in the \texttt{VolumeLayout\-V2Normal} probe. If it does, the volume was
normal. If not, the volume was hidden (or what the attacker captured was not the
result of a mount command; we assume the attacker knows that the user is
mounting a volume).

\subsubsection{Experiment}

Our experiment creates two 1MB TrueCrypt volumes. One is a normal volume, the
other contains a hidden volume; both are protected by the same passphrase.

The experiment assumes the attacker knows the TrueCrypt command was run to mount
a volume, and not some other task (like unmounting a volume). This could be done
in practice by looking at the process list to see which command-line options
were passed to TrueCrypt.

The experiment starts the attack tool on the TrueCrypt binary. After waiting for
the tool to start, it randomly mounts either the normal or hidden volume with
TrueCrypt's command-line tool. It stops the attack tool and checks if the last
probe hit was the \texttt{GetDataSize()} probe. If it was, it guesses that the
volume is normal. If not, it guesses that the volume is hidden.

In one run\footnote{truecrypt/0001} of the experiment on System 1 with $S=500$,
the guess was right $416$ times, or 83\%.
(Recall that we call the
number of trials in a run of the experiment $S$.)
Of the $255$ trials with
normal volumes, the guess was right $184$ times. Of the $245$ trials with hidden
volumes, the guess was right $232$ times. So the error is skewed towards
mistakenly believing that the volume is hidden. By spying as the volume is
mounted multiple times, the confidence can be increased. In one
experiment\footnote{truecrypt/0005} where the attacker is allowed to watch the
volume get mounted 3 times and then take the majority of their decisions, they
decide correctly $93$ times out of $100$. In a longer
run\footnote{truecrypt/0006} with majority-of-three, the attacker decides
correctly $476$ out of $500$ times (95\%). $237/256$ (92.6\%) of the normal
samples were correct, $239/244$ (93.4\%) of the hidden samples were correct.

We could not reproduce the attack on System 2. We believe this is due to
differences in the way the two CPUs do instruction prefetching. The code for
normal volumes and hidden volumes is very close together, so the attack is
easily foiled by prefetching. System 2's processor seems to prefetch backwards
and read the normal volume \texttt{GetDataSize()} code when the hidden volume
\texttt{GetDataSize()} is executing. This makes it impossible to distinguish the
two cases on System 2 with our choice of probes. We were unable to find another
choice of probes that would allow it to work.

We only tried the attack with TrueCrypt's command-line interface. The attack
should extend to the graphical interface, since both interfaces are front ends
to the same volume mounting code. 

\section{Related Work}
\label{sec:relwork}


Our attacks are based on the \textsc{Flush+Reload} attack~\cite{yarom2013flush},
which in turn is based on work by Bangerter et al.\ where it was used to break
an implementation of AES \cite{gullasch2011cache}. It has since been applied to
GnuPG \cite{yarom2013flush} and OpenSSL \cite{benger2014ooh,
yarom2014recovering}. Our work is the first time it has been applied in
a non-cryptographic setting, to the best of our knowledge.

There is a vast body of literature on attacking cryptographic software and
hardware with side channels. There are far too many examples to list here; we
refer to the interesting cases of extracting RSA keys via power analysis
\cite{messerges1999power} and sound recording \cite{genkin2013rsa} as well as
the \textsc{Flush+Reload} attacks \cite{yarom2013flush, benger2014ooh,
yarom2014recovering}.

Our work shows that cache side channel attacks have privacy implications beyond
breaking cryptography software. We believe we are the first to apply a cache
side channel against non-cryptographic user software to compromise privacy. Cache
side channels have already been applied once in a non-cryptographic context: to
break kernel address space layout randomization \cite{hund2013practical}.

There is a growing body of work showing that other kinds of side channels are
successfully breaking privacy. Here we highlight some of that work.

\begin{itemize}
    \item[--] Some web applications leak information about the user's input through
          their behavior in communicating with the web server (even over
          encrypted connections) \cite{bortz2007exposing, chen2010side}.
      \item[--] Malicious web pages can read text inside an \texttt{iframe} by
          exploiting differences in the time it takes to run SVG
          filters~\cite{pixelperfect}. 
    \item[--] Timing variations in databases make it possible to extract
          indexed records \cite{futoransky2007nd2db}.
    \item[--] Variable bit rate encoding can leak words spoken over encrypted
        VoIP links \cite{white2011phonotactic}.
    \item[--] In an Android app, the UI state can be inferred through side channels
          in the GUI framework \cite{chen2014peeking}.
    \item[--] On Linux, keystroke timings can be learned by watching the size of
          a victim process's stack \cite{zhang2009peeping} and by observing an
          SSH connection \cite{song2001timing}. Knowing the timing of keystrokes
          reveals information about what is being typed.
    \item[--] Malicious web pages can use ``red pill'' side channels to find out if
          they are running in a virtual machine (i.e.\ for malware analysis)
          \cite{ho2014tick}.
    \item[--] An attacker can learn a victim's Internet traffic volume, as well as
          individual packet times by exploiting a side channel in router
          scheduling algorithms \cite{kadloor2010low}.
    \item[--] A smartphone app can infer what the user is typing from accelerometer
          and gyroscope measurements \cite{owusu2012accessory,
          cai2012practicality}.
    \item[--] Profiling a server's power use can reveal which virtual machines
        are running on it \cite{hlavacs2011energy}.
    \item[--] Text can be recovered from the sounds dot-matrix printers make
          \cite{backes2010acoustic}.
\end{itemize}

We believe our attacks represent first steps towards understanding the power and
limitations of using the \textsc{Flush+Reload} attack to compromise privacy, and
we hope they will motivate more research on the topic.

\section{Future Work}
\label{sec:future}

There are many more input distinguishing attacks waiting to be discovered. It
should be easy to use the tools we developed to find new attacks.

Quantitatively, our attacks only need a small amount of information from the
target programs. For TrueCrypt, we are only extracting one bit of information.
The Links and Poppler attacks only need to extract a number of bits logarithmic
in the number of inputs we want to distinguish between (6.6 bits for 100 pages,
7.0 for 127). We should find out how much information \textsc{Flush+Reload} can
extract from non-cryptographic programs. In particular, can
\textsc{Flush+Reload} extract previously-unknown user input from a program in
any plausible scenario?

Knowing that \textsc{Flush+Reload} has privacy implications to regular users and
not just users of cryptography, there will be a higher demand to know if one's
system is vulnerable. A database of vulnerable processors should be created, and
an easy-to-use test tool should be developed, so that users can determine if
their processors are vulnerable. With this knowledge, consumers can make better
purchasing decisions.

Some virtual machine hypervisors deduplicate pages between virtual machines. If
two physical pages have identical content, the hypervisor will change the page
tables so that both virtual machines access the same physical page. Two isolated
virtual machines on the same hardware can communicate by checking whether their
pages were deduplicated. The resulting covert channel can move 80 bits per
second, but with latencies of up to five minutes \cite{xiao2013security}. Using
\textsc{Flush+Reload} to communicate through the deduplicated pages could yield
similar data rates with less latency.

When memory deduplication is accomplished by matching page contents, an attacker
can force a deduplication to happen on any page whose contents they know.
Therefore, if an attacker knows or can predict some data inside another process,
they can spy on it with \textsc{Flush+Reload}. This might reveal more
information than spying on the program code alone.

\section{Conclusion}
\label{sec:conclusion}

Classically, side channel attacks are used to extract encryption keys from
software and hardware implementations of cryptography. More recently, side
channel attacks are being used to compromise privacy in more general settings.
We presented three attacks that extend this work, along with accompanying tools
and an automation framework.

Our attacks let an attacker (1) determine which of the top 100 Wikipedia pages
a victim visited with the Links web browser, (2) determine which of the 127
debates in the 2014 Canadian parliament a victim transcoded with the
\texttt{pdftops} command, and (3) determine whether a TrueCrypt volume contains
a hidden volume when it is mounted.

These attacks are not directly damaging on their own, but they are part of a growing
body of work that applies side channels in a much broader setting than breaking
cryptography. 

\section*{Acknowledgment}

The second author's research is supported in part by a grant from the
Natural Sciences and Engineering Research Council of Canada.

{\footnotesize \bibliographystyle{acm}
\bibliography{proposal}}

\appendix
\section{Reproducing this Work}
\label{sec:reproducing}

The attack tools, experiment implementations, and experiment data are all
available for download on the first author's website:
\url{https://defuse.ca/compromising-privacy-flush-reload.htm}.
For longevity, a second copy of everything is archived in the Internet
Archive, at
\url{https://archive.org/URL-TO-BE-DETERMINED}.

Throughout this paper we have referred to experiment runs by the name of the
experiment followed by a four-digit run number. The experiment name
corresponds to a directory name in the archive, and the run number
corresponds to a subdirectory of that directory. For example, the
truecrypt/0003 data can be found in
\texttt{experiments/truecrypt/runs/0003}.

\end{document}
